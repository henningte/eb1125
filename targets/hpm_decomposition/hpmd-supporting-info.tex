% Options for packages loaded elsewhere
\PassOptionsToPackage{unicode}{hyperref}
\PassOptionsToPackage{hyphens}{url}
%
\documentclass[
  12pt,
]{article}
\usepackage{amsmath,amssymb}
\usepackage{lmodern}
\usepackage{iftex}
\ifPDFTeX
  \usepackage[T1]{fontenc}
  \usepackage[utf8]{inputenc}
  \usepackage{textcomp} % provide euro and other symbols
\else % if luatex or xetex
  \usepackage{unicode-math}
  \defaultfontfeatures{Scale=MatchLowercase}
  \defaultfontfeatures[\rmfamily]{Ligatures=TeX,Scale=1}
\fi
% Use upquote if available, for straight quotes in verbatim environments
\IfFileExists{upquote.sty}{\usepackage{upquote}}{}
\IfFileExists{microtype.sty}{% use microtype if available
  \usepackage[]{microtype}
  \UseMicrotypeSet[protrusion]{basicmath} % disable protrusion for tt fonts
}{}
\makeatletter
\@ifundefined{KOMAClassName}{% if non-KOMA class
  \IfFileExists{parskip.sty}{%
    \usepackage{parskip}
  }{% else
    \setlength{\parindent}{0pt}
    \setlength{\parskip}{6pt plus 2pt minus 1pt}}
}{% if KOMA class
  \KOMAoptions{parskip=half}}
\makeatother
\usepackage{xcolor}
\IfFileExists{xurl.sty}{\usepackage{xurl}}{} % add URL line breaks if available
\IfFileExists{bookmark.sty}{\usepackage{bookmark}}{\usepackage{hyperref}}
\hypersetup{
  pdftitle={Supporting information to: Peat Oxic and Anoxic Controls of Sphagnum Decomposition Rates in the Holocene Peatland Model Decomposition Module Estimated from Litterbag Data},
  pdfauthor={Henning Teickner1,; Edzer Pebesma2; Klaus-Holger Knorr1},
  hidelinks,
  pdfcreator={LaTeX via pandoc}}
\urlstyle{same} % disable monospaced font for URLs
\usepackage[margin=1in]{geometry}
\usepackage{color}
\usepackage{fancyvrb}
\newcommand{\VerbBar}{|}
\newcommand{\VERB}{\Verb[commandchars=\\\{\}]}
\DefineVerbatimEnvironment{Highlighting}{Verbatim}{commandchars=\\\{\}}
% Add ',fontsize=\small' for more characters per line
\usepackage{framed}
\definecolor{shadecolor}{RGB}{248,248,248}
\newenvironment{Shaded}{\begin{snugshade}}{\end{snugshade}}
\newcommand{\AlertTok}[1]{\textcolor[rgb]{0.94,0.16,0.16}{#1}}
\newcommand{\AnnotationTok}[1]{\textcolor[rgb]{0.56,0.35,0.01}{\textbf{\textit{#1}}}}
\newcommand{\AttributeTok}[1]{\textcolor[rgb]{0.77,0.63,0.00}{#1}}
\newcommand{\BaseNTok}[1]{\textcolor[rgb]{0.00,0.00,0.81}{#1}}
\newcommand{\BuiltInTok}[1]{#1}
\newcommand{\CharTok}[1]{\textcolor[rgb]{0.31,0.60,0.02}{#1}}
\newcommand{\CommentTok}[1]{\textcolor[rgb]{0.56,0.35,0.01}{\textit{#1}}}
\newcommand{\CommentVarTok}[1]{\textcolor[rgb]{0.56,0.35,0.01}{\textbf{\textit{#1}}}}
\newcommand{\ConstantTok}[1]{\textcolor[rgb]{0.00,0.00,0.00}{#1}}
\newcommand{\ControlFlowTok}[1]{\textcolor[rgb]{0.13,0.29,0.53}{\textbf{#1}}}
\newcommand{\DataTypeTok}[1]{\textcolor[rgb]{0.13,0.29,0.53}{#1}}
\newcommand{\DecValTok}[1]{\textcolor[rgb]{0.00,0.00,0.81}{#1}}
\newcommand{\DocumentationTok}[1]{\textcolor[rgb]{0.56,0.35,0.01}{\textbf{\textit{#1}}}}
\newcommand{\ErrorTok}[1]{\textcolor[rgb]{0.64,0.00,0.00}{\textbf{#1}}}
\newcommand{\ExtensionTok}[1]{#1}
\newcommand{\FloatTok}[1]{\textcolor[rgb]{0.00,0.00,0.81}{#1}}
\newcommand{\FunctionTok}[1]{\textcolor[rgb]{0.00,0.00,0.00}{#1}}
\newcommand{\ImportTok}[1]{#1}
\newcommand{\InformationTok}[1]{\textcolor[rgb]{0.56,0.35,0.01}{\textbf{\textit{#1}}}}
\newcommand{\KeywordTok}[1]{\textcolor[rgb]{0.13,0.29,0.53}{\textbf{#1}}}
\newcommand{\NormalTok}[1]{#1}
\newcommand{\OperatorTok}[1]{\textcolor[rgb]{0.81,0.36,0.00}{\textbf{#1}}}
\newcommand{\OtherTok}[1]{\textcolor[rgb]{0.56,0.35,0.01}{#1}}
\newcommand{\PreprocessorTok}[1]{\textcolor[rgb]{0.56,0.35,0.01}{\textit{#1}}}
\newcommand{\RegionMarkerTok}[1]{#1}
\newcommand{\SpecialCharTok}[1]{\textcolor[rgb]{0.00,0.00,0.00}{#1}}
\newcommand{\SpecialStringTok}[1]{\textcolor[rgb]{0.31,0.60,0.02}{#1}}
\newcommand{\StringTok}[1]{\textcolor[rgb]{0.31,0.60,0.02}{#1}}
\newcommand{\VariableTok}[1]{\textcolor[rgb]{0.00,0.00,0.00}{#1}}
\newcommand{\VerbatimStringTok}[1]{\textcolor[rgb]{0.31,0.60,0.02}{#1}}
\newcommand{\WarningTok}[1]{\textcolor[rgb]{0.56,0.35,0.01}{\textbf{\textit{#1}}}}
\usepackage{longtable,booktabs,array}
\usepackage{calc} % for calculating minipage widths
% Correct order of tables after \paragraph or \subparagraph
\usepackage{etoolbox}
\makeatletter
\patchcmd\longtable{\par}{\if@noskipsec\mbox{}\fi\par}{}{}
\makeatother
% Allow footnotes in longtable head/foot
\IfFileExists{footnotehyper.sty}{\usepackage{footnotehyper}}{\usepackage{footnote}}
\makesavenoteenv{longtable}
\usepackage{graphicx}
\makeatletter
\def\maxwidth{\ifdim\Gin@nat@width>\linewidth\linewidth\else\Gin@nat@width\fi}
\def\maxheight{\ifdim\Gin@nat@height>\textheight\textheight\else\Gin@nat@height\fi}
\makeatother
% Scale images if necessary, so that they will not overflow the page
% margins by default, and it is still possible to overwrite the defaults
% using explicit options in \includegraphics[width, height, ...]{}
\setkeys{Gin}{width=\maxwidth,height=\maxheight,keepaspectratio}
% Set default figure placement to htbp
\makeatletter
\def\fps@figure{htbp}
\makeatother
\setlength{\emergencystretch}{3em} % prevent overfull lines
\providecommand{\tightlist}{%
  \setlength{\itemsep}{0pt}\setlength{\parskip}{0pt}}
\setcounter{secnumdepth}{5}
\newlength{\cslhangindent}
\setlength{\cslhangindent}{1.5em}
\newlength{\csllabelwidth}
\setlength{\csllabelwidth}{3em}
\newlength{\cslentryspacingunit} % times entry-spacing
\setlength{\cslentryspacingunit}{\parskip}
\newenvironment{CSLReferences}[2] % #1 hanging-ident, #2 entry spacing
 {% don't indent paragraphs
  \setlength{\parindent}{0pt}
  % turn on hanging indent if param 1 is 1
  \ifodd #1
  \let\oldpar\par
  \def\par{\hangindent=\cslhangindent\oldpar}
  \fi
  % set entry spacing
  \setlength{\parskip}{#2\cslentryspacingunit}
 }%
 {}
\usepackage{calc}
\newcommand{\CSLBlock}[1]{#1\hfill\break}
\newcommand{\CSLLeftMargin}[1]{\parbox[t]{\csllabelwidth}{#1}}
\newcommand{\CSLRightInline}[1]{\parbox[t]{\linewidth - \csllabelwidth}{#1}\break}
\newcommand{\CSLIndent}[1]{\hspace{\cslhangindent}#1}
\usepackage{float}
\usepackage[version=4]{mhchem}
\usepackage{booktabs}
\usepackage{multirow}
\usepackage{bm}
\usepackage{xr} \externaldocument[main-]{hpmd-paper}
\usepackage{tocloft}
\setlength{\cftsecnumwidth}{5ex}
\ifLuaTeX
  \usepackage{selnolig}  % disable illegal ligatures
\fi

\title{Supporting information to: Peat Oxic and Anoxic Controls of \emph{Sphagnum} Decomposition Rates in the Holocene Peatland Model Decomposition Module Estimated from Litterbag Data}
\author{Henning Teickner\textsuperscript{1,*} \and Edzer Pebesma\textsuperscript{2} \and Klaus-Holger Knorr\textsuperscript{1}}
\date{23 January, 2025}

\begin{document}
\maketitle

{
\setcounter{tocdepth}{2}
\tableofcontents
}
\textsuperscript{1} ILÖK, Ecohydrology \& Biogeochemistry Group, Institute of Landscape Ecology, University of Münster, 48149, Germany\\
\textsuperscript{2} IfGI, Spatiotemporal Modelling Lab, Institute for Geoinformatics, University of Münster, 48149, Germany

\textsuperscript{*} Correspondence: \href{mailto:henning.teickner@uni-muenster.de}{Henning Teickner \textless{}\href{mailto:henning.teickner@uni-muenster.de}{\nolinkurl{henning.teickner@uni-muenster.de}}\textgreater{}}

\renewcommand{\thefigure}{S\arabic{figure}} 
\renewcommand{\thetable}{S\arabic{table}}
\renewcommand{\thesection}{S\arabic{section}}
\renewcommand{\theequation}{S\arabic{equation}}

\hypertarget{sup-2}{%
\section{Prior choices and justification}\label{sup-2}}













\begin{table}[H]

\caption{\label{tab:sup-out-d-sdm-all-models-priors-1}Prior distributions of all Bayesian models and their justifications. ``Parameter name in code'' is the name for the parameter as used in our Stan models. ``Parameter name in text'' is the name of the corresponding parameter we use in the main text and figures. ``Equation in main text'' references the equation in the main text where the parameter occurs. When there is no value for ``Justification'', the prior was chosen based on prior predictive checks against the data. This prior predictive check tests whether the models can produce distributions of measured variables we expect based on prior knowledge. The results of these prior predictive checks are shown in supporting section \ref{sup-3}.}
\centering
\resizebox{\linewidth}{!}{
\begin{tabular}[t]{lllll>{\raggedright\arraybackslash}p{7cm}}
\toprule
Parameter name in code & Parameter name in text & Equation in main text & Unit & Prior distribution & Justification\\
\midrule
l\_2\_p1 &  &  & (g g$_\text{initial}^{-1}$) (logit scale) & normal(-3.5, l\_2\_p1\_p2) & Assumes an average initial leaching loss across all available litterbag data within (95\% confidence interval) (0.012, 0.068) g g$_\text{initial}^{-1}$\\
l\_2\_p2 &  &  & (g g$_\text{initial}^{-1}$) (logit scale) & normal(0, l\_2\_p2\_p2) & \\
l\_2\_p3 &  &  & (g g$_\text{initial}^{-1}$) (logit scale) & normal(0, l\_2\_p3\_p2) & \\
l\_2\_p4 &  &  & (g g$_\text{initial}^{-1}$) (logit scale) & normal(0, l\_2\_p4\_p2) & \\
k\_2\_p1 & $\beta_{k,1}$ & \ref{main-eq:model-link-2} & (yr$^{-1}$) (log scale) & normal(-2.9, k\_2\_p1\_p2) & Assumes an average initial decomposition rate across all available litterbag data within (95\% confidence interval) (0.024, 0.131) yr$^{-1}$\\
\addlinespace
k\_2\_p2 & $\beta_{k,2,\text{species}}$ & \ref{main-eq:model-link-2} & (yr$^{-1}$) (log scale) & normal(0, k\_2\_p2\_p2) & Centered at the standard value used in the HPM.\\
k\_2\_p3 & $\beta_{k,3,\text{species x study}}$ & \ref{main-eq:model-link-2} & (yr$^{-1}$) (log scale) & normal(0, k\_2\_p3\_p2) & Centered at the standard value used in the HPM.\\
k\_2\_p4 & $\beta_{k,4,\text{sample}}$ & \ref{main-eq:model-link-2} & (yr$^{-1}$) (log scale) & normal(0, k\_2\_p4\_p2) & Centered at the standard value used in the HPM.\\
phi\_2\_p2\_p1 &  &  & (-) (log scale) & normal(5, phi\_2\_p2\_p1\_p2) & \\
phi\_2\_p2\_p2 &  &  & (-) (log scale) & normal(0, phi\_2\_p2\_p2\_p2) & \\
\addlinespace
phi\_2\_p2\_p3 &  &  & (-) (log scale) & normal(0, phi\_2\_p2\_p3\_p2) & \\
phi\_2\_p2\_p4 &  &  & (-) (log scale) & normal(0, phi\_2\_p2\_p4\_p2) & \\
alpha\_2\_p1 &  &  & (-) (log scale) & normal(-0.2, 0.3) & Assumes an average $\alpha$ across all available litterbag data within (95\% confidence interval) (1.451, 2.473)\\
alpha\_2\_p2 &  &  & (-) (log scale) & normal(0, 0.3) & \\
alpha\_2\_p3 &  &  & (-) (log scale) & normal(0, 0.3) & \\
\addlinespace
alpha\_2\_p4 &  &  & (-) (log scale) & normal(0, 0.2) & \\
k\_2\_p1\_p2 &  &  & (yr$^{-1}$) (log scale) & half-normal(0, 0.4) & \\
k\_2\_p2\_p2 &  &  & (yr$^{-1}$) (log scale) & half-normal(0, 0.4) & \\
k\_2\_p3\_p2 &  &  & (yr$^{-1}$) (log scale) & half-normal(0, 0.4) & \\
k\_2\_p4\_p2 &  &  & (yr$^{-1}$) (log scale) & half-normal(0, 0.4) & \\
\addlinespace
phi\_2\_p2\_p1\_p2 &  &  & (-) (log scale) & half-normal(0, 0.3) & \\
phi\_2\_p2\_p2\_p2 &  &  & (-) (log scale) & half-normal(0, 0.3) & \\
phi\_2\_p2\_p3\_p2 &  &  & (-) (log scale) & half-normal(0, 0.3) & \\
phi\_2\_p2\_p4\_p2 &  &  & (-) (log scale) & half-normal(0, 0.3) & \\
l\_2\_p1\_p2 &  &  & (g g$_\text{initial}^{-1}$) (logit scale) & half-normal(0, 0.4) & \\
\addlinespace
l\_2\_p2\_p2 &  &  & (g g$_\text{initial}^{-1}$) (logit scale) & half-normal(0, 0.4) & \\
l\_2\_p3\_p2 &  &  & (g g$_\text{initial}^{-1}$) (logit scale) & half-normal(0, 0.4) & \\
l\_2\_p4\_p2 &  &  & (g g$_\text{initial}^{-1}$) (logit scale) & half-normal(0, 0.4) & \\
layer\_total\_porosity\_1 & $P$ & \ref{main-eq:hpm-modified-granberg-1} & L$_\text{pores}$ L$_\text{sample}^{-1}$ & beta(12, 3) & Centered at the standard value used in the HPM.\\
layer\_minimum\_degree\_of\_saturation\_at\_surface\_1 & $\theta_\text{0,min}$ & \ref{main-eq:hpm-modified-granberg-1} & L$_\text{water}$ L$_\text{pores}^{-1}$ & beta(0.9, 17.1) & Centered at the standard value used in the HPM.\\
\addlinespace
layer\_water\_table\_depth\_to\_surface\_1 &  &  & cm & normal(average reported WTD, 3) & The average was set to the average water table depths reported in the litterbag studies.\\
hpm\_k\_2\_p1 & $\alpha_{\mu_k}$ & \ref{main-eq:model-link-1} & (-) & gamma(20, 1) & Centered at the standard value used in the HPM.\\
m69\_p1 & $W_{opt}$ & \ref{main-eq:hpm-moisture-modifier-1} & L$_\text{water}$ L$_\text{pores}^{-1}$ & beta(13.5, 16.5) & Centered at the standard value used in the HPM.\\
m69\_p2 & $c_{1}$ & \ref{main-eq:hpm-moisture-modifier-1} & (-) & gamma(20, 8.66) & Centered at the standard value used in the HPM.\\
m68\_p1 & $f_{min}$ & \ref{main-eq:hpm-moisture-modifier-2} & (yr$^{-1}$) & gamma(5, 5000) & Centered at the standard value used in the HPM.\\
\addlinespace
m68\_p2 & $c_{2}$ & \ref{main-eq:hpm-moisture-modifier-2} & (cm) & gamma(5, 16.67) & Centered at the standard value used in the HPM.\\
m68\_p3\_2\_p1 & $k_{0,i}$ & \ref{main-eq:hpm-decomposition-rate-1} & (yr$^{-1}$) (log scale) & normal(-2.2, 0.3) & Assumes a maximum potential initial decomposition rate across all species within (95\% confidence interval) (0.061, 0.2) yr$^{-1}$\\
hpm\_l\_2\_p1 & $\beta_{l,1}$ & \ref{main-eq:hpm-hpm-l-2} & (g g$_\text{initial}^{-1}$) (logit scale) & normal(-2.2, 0.3) & Centered at the standard value used in the HPM.\\
hpm\_l\_2\_p3 & $\beta_{l,2}$ & \ref{main-eq:hpm-hpm-l-2} & (g g$_\text{initial}^{-1}$ L$_\text{water}^{-1}$ L$_\text{pores}$) (logit scale) & normal(0, 0.5) & Centered at the standard value used in the HPM.\\
hpm\_l\_2\_p4 & $\phi_l$ & \ref{main-eq:hpm-hpm-l-2} & (-) & gamma(10, 0.25) & Centered at the standard value used in the HPM.\\
\bottomrule
\end{tabular}}
\end{table}

\clearpage

\hypertarget{sup-12}{%
\section{Further Information on Bayesian Data Analysis}\label{sup-12}}

\hypertarget{monte-carlo-standard-errors}{%
\paragraph*{Monte Carlo Standard Errors}\label{monte-carlo-standard-errors}}
\addcontentsline{toc}{paragraph}{Monte Carlo Standard Errors}

Monte Carlo standard errors (MSCE) (\protect\hyperlink{ref-Vehtari.2021}{Vehtari et al., 2021}) for the median were at most 0.012 yr\(^{-1}\) for \(k_0\), 0.363 mass-\% for \(l_0\), 0.043 for \(\alpha\), 0.401 mass-\% for the remaining mass, 0.001 L\(_\text{water}\) L\(_\text{pores}^{-1}\) for \(W_{opt}\), 0.004 for \(c_{1}\), 0.001 yr\(^{-1}\) for \(f_{min}\), 0.002 m for \(c_{2}\), 0.003 yr\(^{-1}\) for \(k_0\) predicted by the HPM modifications, and 0.342 mass-\% for \(l_0\) predicted by HPM-leaching. For the 2.5\% and 97.5\% quantiles, MCSE were at most 0.088 yr\(^{-1}\) for \(k_0\), 0.646 mass-\% for \(l_0\), 0.147 for \(\alpha\), 2.742 mass-\% for the remaining mass, 0.004 L\(_\text{water}\) L\(_\text{pores}^{-1}\) for \(W_{opt}\), 0.006 for \(c_{1}\), 0.005 yr\(^{-1}\) for \(f_{min}\), 0.007 m for \(c_{2}\), 0.003 yr\(^{-1}\) for \(k_0\) predicted by the HPM modifications, and 0.293 mass-\% for \(l_0\) predicted by HPM-leaching.

\hypertarget{power-scaling}{%
\paragraph*{Power-scaling}\label{power-scaling}}
\addcontentsline{toc}{paragraph}{Power-scaling}

Power-scaling exponentiates prior (to analyze prior sensitivity) or likelihood (to analyze likelihood sensitivity) distributions by different constants \(\alpha>0\), where \(\alpha>1\) means that the scaled component gets more important relative to the other component, and \(\alpha<1\) means it gets less important (\protect\hyperlink{ref-Kallioinen.2024}{Kallioinen et al., 2024}). We varied \(\alpha\) from \(0.99\) to \(1.01\) (default option) and identified sensitivity with the cumulative Jensen-Shannon distance and a threshold of 0.05, as suggested in Kallioinen et al. (\protect\hyperlink{ref-Kallioinen.2024}{2024}).

The power-scaling sensitivity analysis indicates a weak likelihood for all peat properties for most litterbag experiments, indicating that, not surprisingly, remaining masses alone do not give much information about peat properties. For \(W_{opt}\), \(c_{1}\), and \(c_{2}\) the analysis suggested a prior-data conflict which supports our finding that parameter values different from the standard values are more compatible with the data. For \(k_{0,i}\), the analysis suggested a prior-data conflict for most species, and similar for the parameters with which we modeled how initial leaching losses depend on the degree of saturation. We did not attempt to resolve these conflicts, either because we know from our previous study that the data provide only uncertain information (\protect\hyperlink{ref-Teickner.2025}{Teickner et al., 2025}) which makes prior-data conflicts more likely, or because we wanted to use HPM standard parameter values as prior information. A future update of our study with more accurate data may address these challenges.

\hypertarget{software}{%
\paragraph*{Software}\label{software}}
\addcontentsline{toc}{paragraph}{Software}

All computations were done in R (4.2.0) (\protect\hyperlink{ref-RCoreTeam.2022}{R Core Team, 2022}). We computed prior and posterior predictive checks with the bayesplot package (1.9.0) (\protect\hyperlink{ref-Gabry.2022}{Gabry and Mahr, 2022}) (supporting section \ref{sup-3}). Data were handled with tidyverse packages (\protect\hyperlink{ref-Wickham.2019}{Wickham et al., 2019}), MCMC samples with the posterior (1.5.0) (\protect\hyperlink{ref-Burkner.2023}{Bürkner et al., 2023}) and tidybayes (3.0.2) (\protect\hyperlink{ref-Kay.2022}{Kay, 2022b}) packages. Graphics were created with ggplot2 (3.4.4) (\protect\hyperlink{ref-Wickham.2016}{Wickham, 2016}), ggdist (3.1.1) (\protect\hyperlink{ref-Kay.2022a}{Kay, 2022a}) and patchwork (1.1.1) (\protect\hyperlink{ref-Pedersen.2020}{Pedersen, 2020}).

\hypertarget{sup-3}{%
\section{Prior and posterior predictive checks}\label{sup-3}}



\begin{figure}[H]

{\centering \includegraphics[width=1\linewidth]{figures/hpmd_plot_ppc_prior_m} 

}

\caption{Density estimate of 100 sets of remaining masses sampled from the prior distribution of each model (light blue lines) versus density estimate of the measured remaining masses from the litterbag studies.}\label{fig:sup-hpmd-plot-ppc-prior-m}
\end{figure}



\begin{figure}[H]

{\centering \includegraphics[width=1\linewidth]{figures/hpmd_plot_ppc_posterior_m} 

}

\caption{Density estimate of 100 sets of remaining masses sampled from the posterior distribution of each model (light blue lines) versus density estimate of the measured remaining masses from the litterbag studies.}\label{fig:sup-hpmd-plot-ppc-posterior-m}
\end{figure}



\begin{figure}[H]

{\centering \includegraphics[width=1\linewidth]{figures/hpmd_plot_ppc_prior_phi} 

}

\caption{Density estimate of 100 sets of remaining mass errors (converted to precision) sampled from the prior distribution of each model (light blue lines) versus density estimate of the measured remaining mass errors from the litterbag studies. The x axis is log scaled.}\label{fig:sup-hpmd-plot-ppc-prior-phi}
\end{figure}



\begin{figure}[H]

{\centering \includegraphics[width=1\linewidth]{figures/hpmd_plot_ppc_posterior_phi} 

}

\caption{Density estimate of 100 sets of remaining mass errors (converted to precision) sampled from the posterior distribution of each model (light blue lines) versus density estimate of the measured remaining mass errors from the litterbag studies. The x axis is log scaled.}\label{fig:sup-hpmd-plot-ppc-posterior-phi}
\end{figure}



\begin{figure}[H]

{\centering \includegraphics[width=1\linewidth]{figures/hpmd_plot_ppc_prior_hpm_k_2} 

}

\caption{Density estimate of 100 sets of decomposition rates (\(k_0\)) predicted by the HPM modifications sampled from the prior distribution of each model (light blue lines) versus density estimate of the decomposition rates estimated from the litterbag studies.}\label{fig:sup-hpmd-plot-ppc-prior-hpm-k-2}
\end{figure}



\begin{figure}[H]

{\centering \includegraphics[width=1\linewidth]{figures/hpmd_plot_ppc_posterior_hpm_k_2} 

}

\caption{Density estimate of 100 sets of decomposition rates (\(k_0\)) predicted by the HPM modifications sampled from the posterior distribution of each model (light blue lines) versus density estimate of the decomposition rates estimated from the litterbag studies.}\label{fig:sup-hpmd-plot-ppc-posterior-hpm-k-2}
\end{figure}



\begin{figure}[H]

{\centering \includegraphics[width=1\linewidth]{figures/hpmd_plot_ppc_prior_hpm_l_2} 

}

\caption{Density estimate of 100 sets of initial leaching losses (\(l_0\)) predicted by HPM-leaching sampled from the prior distribution (light blue lines) versus density estimate of the initial leaching loss estimated from the litterbag studies.}\label{fig:sup-hpmd-plot-ppc-prior-hpm-l-2}
\end{figure}



\begin{figure}[H]

{\centering \includegraphics[width=1\linewidth]{figures/hpmd_plot_ppc_posterior_hpm_l_2} 

}

\caption{Density estimate of 100 sets of initial leaching losses (\(l_0\)) predicted by HPM-leaching sampled from the posterior distribution (light blue lines) versus density estimate of the initial leaching loss estimated from the litterbag studies.}\label{fig:sup-hpmd-plot-ppc-posterior-hpm-l-2}
\end{figure}

\hypertarget{sup-4}{%
\section{\texorpdfstring{\(k_{0,i}\) estimates in HPM-all and in HPM-leaching}{k\_\{0,i\} estimates in HPM-all and in HPM-leaching}}\label{sup-4}}



\begin{figure}[H]

{\centering \includegraphics[width=0.6\linewidth]{figures/hpmd_plot_6} 

}

\caption{\(k_{0,i}\) estimates in HPM-all and in HPM-leaching for each \emph{Sphagnum} species. Points are average values and error bars are 95\% confidence intervals. \emph{Sphagnum} spec. are samples which have been identified only to the genus level and there are two values here because we defined two separate species in the HPM to estimate maximum possible decomposition rates separately for initial peat samples collected from 10 or 20 cm depth in Prevost et al. (\protect\hyperlink{ref-Prevost.1997}{1997}).}\label{fig:sup-hpmd-plot-6}
\end{figure}

\hypertarget{sup-5}{%
\section{Marginal posterior distributions of HPM parameters in HPM-all and HPM-leaching}\label{sup-5}}



\begin{figure}[H]

{\centering \includegraphics[width=1\linewidth]{figures/hpmd_plot_4_3} 

}

\caption{Marginal posterior distributions of HPM decomposition model parameters as estimated by HPM-all. (a) \(k_0\) estimated for each species. Species were assigned to HPM microhabitats as described in the Methods section in the main text. (b) other HPM parameters (see Tab. \ref{main-tab:hpmd-m-tab-hpm-parameters-standard-values} in the main text for details). Vertical black lines are the standard parameter values from Frolking et al. (\protect\hyperlink{ref-Frolking.2010}{2010}). \emph{Sphagnum} spec. are samples which have been identified only to the genus level.}\label{fig:sup-hpmd-plot-4-3}
\end{figure}



\begin{figure}[H]

{\centering \includegraphics[width=1\linewidth]{figures/hpmd_plot_4_4} 

}

\caption{Marginal posterior distributions of HPM decomposition model parameters as estimated by HPM-leaching. (a) \(k_0\) estimated for each species. Species were assigned to HPM microhabitats as described in the Methods section in the main text. (b) other HPM parameters (see Tab. \ref{main-tab:hpmd-m-tab-hpm-parameters-standard-values} in the main text for details). Vertical black lines are the standard parameter values from Frolking et al. (\protect\hyperlink{ref-Frolking.2010}{2010}). \emph{Sphagnum} spec. are samples which have been identified only to the genus level.}\label{fig:sup-hpmd-plot-4-4}
\end{figure}



\begin{figure}[H]

{\centering \includegraphics[width=1\linewidth]{figures/hpmd_plot_5} 

}

\caption{Marginal posterior distributions of HPM decomposition model parameters as estimated by HPM-leaching during the cross-validation. During the cross-validation, one of the cross-validation folds was left out each time and the model was refitted, producing a marginal posterior distribution for each parameter and cross-validation block. (a) \(k_{0,i}\) estimated for each species for which data were removed during the cross-validation. (b) other HPM parameters (see Tab. \ref{main-tab:hpmd-m-tab-hpm-parameters-standard-values} in the main text for details).}\label{fig:sup-hpmd-plot-5}
\end{figure}

\hypertarget{sup-8}{%
\section{\texorpdfstring{\(k_0\) predicted by the HPM versus water table depth below the litter for different studies and species}{k\_0 predicted by the HPM versus water table depth below the litter for different studies and species}}\label{sup-8}}



\begin{figure}[H]

{\centering \includegraphics[width=1\linewidth]{figures/hpmd_plot_8_1} 

}

\caption{\(k_0\) estimated with LDM-standard (Predicted with HPM = No) and predicted by the HPM decomposition module with standard parameter values (HPM-standard, Predicted with HPM = Yes) versus reported average water table depths below the litterbags for different species and studies (negative values represent litterbags placed below the water table, positive values represent litterbags placed above the water table in the unsaturated zone). Points represent average estimates and error bars 95\% posterior intervals. Lines are predictions of linear models fitted to the average estimates. \emph{Sphagnum} spec. are samples which have been identified only to the genus level. Only data for species with at least three replicates are shown. Error bars exceeding 0.5 yr\(^{-1}\) are clipped.}\label{fig:sup-hpmd-plot-8-1}
\end{figure}



\begin{figure}[H]

{\centering \includegraphics[width=1\linewidth]{figures/hpmd_plot_8_4} 

}

\caption{\(k_0\) estimated with the litterbag decomposition model in HPM-leaching from the litterbag data (Predicted with HPM = No) and predicted by the HPM decomposition module with parameter values estimated from the litterbag data (HPM-leaching, Predicted with HPM = Yes) versus estimated average water table depths below the litterbags for different species and studies (negative values represent litterbags placed below the water table, positive values represent litterbags placed above the water table in the unsaturated zone). Points represent average estimates and error bars 95\% posterior intervals. Lines are predictions of linear models fitted to the average estimates. \emph{Sphagnum} spec. are samples which have been identified only to the genus level. Only data for species with at least three replicates are shown. Error bars exceeding 0.5 yr\(^{-1}\) are clipped.}\label{fig:sup-hpmd-plot-8-4}
\end{figure}

\hypertarget{sup-9}{%
\section{\texorpdfstring{Depth profiles of predicted decomposition rates with \(W_{opt}\) estimated by HPM-leaching or set to its standard value for \emph{S. fallax}}{Depth profiles of predicted decomposition rates with W\_\{opt\} estimated by HPM-leaching or set to its standard value for S. fallax}}\label{sup-9}}



\begin{figure}[H]

{\centering \includegraphics[width=0.6\linewidth]{figures/hpmd_simulation_1_plot_1_5} 

}

\caption{Decomposition rates predicted with HPM-leaching (\(k_{0,\text{modified}}(\text{HPM-leaching})\)) for \emph{S. fallax} (hollows), using either the standard value for \(W_{opt}\) or the \(W_{opt}\) value estimated by HPM-leaching versus depth of the litter below the peat surface. The horizontal line is the average water table depth.}\label{fig:sup-hpmd-simulation-1-plot-5}
\end{figure}

\hypertarget{sup-10}{%
\section{Results for HPM-outlier}\label{sup-10}}



\begin{figure}[H]

{\centering \includegraphics[width=1\linewidth]{figures/hpmd_plot_4_9} 

}

\caption{Marginal posterior distributions of HPM decomposition model parameters as estimated by HPM-outlier. (a) \(k_0\) estimated for each species. Species were assigned to HPM microhabitats as described in the Methods section in the main text. (b) other HPM parameters. Vertical black lines are the standard parameter values from Frolking et al. (\protect\hyperlink{ref-Frolking.2010}{2010}). \emph{Sphagnum} spec. are samples which have been identified only to the genus level.}\label{fig:sup-hpmd-plot-4-9}
\end{figure}

\hypertarget{sup-11}{%
\section{Prediction uncertainties of HPM-leaching}\label{sup-11}}

To illustrate that the HPM decomposition module implies large uncertainties if its parameters are estimated from available litterbag data, we simulate decomposition of \emph{S. fallax} and \emph{S. fuscum} litter during 50 years, either incubated at 10 cm depth under a degree of saturation of 0.6 L\(_\text{water}\) L\(_\text{pores}^{-1}\), or 20 cm below the water table. The results are shown in Fig. \ref{fig:sup-hpmd-simulation-1-plot-6}.



\begin{figure}[H]

{\centering \includegraphics[width=1\linewidth]{figures/hpmd_simulation_1_plot_1_6} 

}

\caption{Fraction of initial mass remaining of \emph{S. fuscum} and \emph{S. fallax} versus incubation duration as predicted by HPM-leaching, assuming average species \(\alpha\) and uncertainty of remaining masses averaged across litterbag experiments. (a) Shows predicted fractions of initial mass remaining and (b) predicted fractions of initial mass remaining for one individual sample. Samples are either incubated in the saturated zone 20 cm below the water table, or in the unsaturated zone 10 cm above the water table. Shaded areas are 50, 80, and 95\% confidence and prediction intervals, respectively.}\label{fig:sup-hpmd-simulation-1-plot-6}
\end{figure}

\hypertarget{sup-7}{%
\section{\texorpdfstring{R code to predict \(k_0\), \(l_0\), and remaining masses with HPM-leaching}{R code to predict k\_0, l\_0, and remaining masses with HPM-leaching}}\label{sup-7}}

HPM-leaching and functions to predict \(k_0\) and \(l_0\) for different species and water table levels are available via the R package hpmdpredict (\protect\hyperlink{ref-Teickner.2025b}{Teickner and Knorr, 2025}). To make predictions, one first has to define some variables like the incubation duration. Here, we predict remaining masses and initial leaching losses for \emph{S. fuscum} incubated at a degree of saturation of 0.6 L\(_\text{water}\) L\(_\text{pores}^{-1}\) during the first five years.

\begin{Shaded}
\begin{Highlighting}[]
\NormalTok{d }\OtherTok{\textless{}{-}} 
\NormalTok{  tibble}\SpecialCharTok{::}\FunctionTok{tibble}\NormalTok{(}
    \AttributeTok{incubation\_duration =} \FunctionTok{seq}\NormalTok{(}\AttributeTok{from =} \DecValTok{0}\NormalTok{, }\AttributeTok{to =} \DecValTok{5}\NormalTok{, }\AttributeTok{length.out =} \DecValTok{30}\NormalTok{),}
    \AttributeTok{m0 =} \DecValTok{1}\NormalTok{,}
    \AttributeTok{layer\_degree\_of\_saturation\_1 =} \FloatTok{0.6}\NormalTok{,}
    \AttributeTok{layer\_water\_table\_depth\_to\_surface\_1 =} \DecValTok{20}\NormalTok{,}
    \AttributeTok{sample\_depth\_lower =} \DecValTok{10}\NormalTok{,}
    \AttributeTok{hpm\_taxon\_rank\_value =} \StringTok{"Sphagnum fuscum"}
\NormalTok{  )}
\end{Highlighting}
\end{Shaded}

Next, one can pass this data frame to \texttt{hpmd\_predict\_fit\_4()} which makes the predictions.

\begin{Shaded}
\begin{Highlighting}[]
\FunctionTok{library}\NormalTok{(hpmdpredict)}
\NormalTok{d }\OtherTok{\textless{}{-}}\NormalTok{ hpmdpredict}\SpecialCharTok{::}\FunctionTok{hpmd\_predict\_fit\_4}\NormalTok{(}\AttributeTok{newdata =}\NormalTok{ d)}
\end{Highlighting}
\end{Shaded}

To illustrate the result, we plot predicted remaining masses versus incubation time:

\begin{Shaded}
\begin{Highlighting}[]
\FunctionTok{library}\NormalTok{(ggplot2)}
\FunctionTok{library}\NormalTok{(ggdist)}

\NormalTok{d }\SpecialCharTok{|}\ErrorTok{\textgreater{}}
  \FunctionTok{ggplot}\NormalTok{(}\FunctionTok{aes}\NormalTok{(}\AttributeTok{ydist =}\NormalTok{ mass\_relative\_mass }\SpecialCharTok{*} \DecValTok{100}\NormalTok{, }\AttributeTok{x =}\NormalTok{ incubation\_duration)) }\SpecialCharTok{+}
  \FunctionTok{stat\_lineribbon}\NormalTok{() }\SpecialCharTok{+}
  \FunctionTok{scale\_fill\_brewer}\NormalTok{() }\SpecialCharTok{+}
  \FunctionTok{labs}\NormalTok{(}
    \AttributeTok{y =} \StringTok{"Fraction of initial mass (\%)"}\NormalTok{,}
    \AttributeTok{x =} \StringTok{"Incubation duration (yr)"}
\NormalTok{  )}
\end{Highlighting}
\end{Shaded}

\begin{center}\includegraphics[width=0.8\linewidth]{hpmd-supporting-info_files/figure-latex/sup-hpmd-hpmdpredict-3-1} \end{center}

Further information are available from the package documentation.

\hypertarget{references}{%
\section*{References}\label{references}}
\addcontentsline{toc}{section}{References}

\hypertarget{refs}{}
\begin{CSLReferences}{0}{0}
\leavevmode\vadjust pre{\hypertarget{ref-Burkner.2023}{}}%
Bürkner, P.-C., Gabry, J., Kay, M., and Vehtari, A.: {posterior}: {Tools} for working with posterior distributions, 2023.

\leavevmode\vadjust pre{\hypertarget{ref-Frolking.2010}{}}%
Frolking, S., Roulet, N. T., Tuittila, E., Bubier, J. L., Quillet, A., Talbot, J., and Richard, P. J. H.: A new model of {Holocene} peatland net primary production, decomposition, water balance, and peat accumulation, Earth System Dynamics, 1, 1--21, \url{https://doi.org/10.5194/esd-1-1-2010}, 2010.

\leavevmode\vadjust pre{\hypertarget{ref-Gabry.2022}{}}%
Gabry, J. and Mahr, T.: {bayesplot}: {Plotting} for bayesian models, 2022.

\leavevmode\vadjust pre{\hypertarget{ref-Kallioinen.2024}{}}%
Kallioinen, N., Paananen, T., Bürkner, P.-C., and Vehtari, A.: Detecting and diagnosing prior and likelihood sensitivity with power-scaling, Statistics and Computing, 34, 57, \url{https://doi.org/10.1007/s11222-023-10366-5}, 2024.

\leavevmode\vadjust pre{\hypertarget{ref-Kay.2022a}{}}%
Kay, M.: {ggdist}: {Visualizations} of distributions and uncertainty, Manual, \url{https://doi.org/10.5281/zenodo.3879620}, 2022a.

\leavevmode\vadjust pre{\hypertarget{ref-Kay.2022}{}}%
Kay, M.: {tidybayes}: {Tidy} data and geoms for {Bayesian} models, \url{https://doi.org/10.5281/zenodo.1308151}, 2022b.

\leavevmode\vadjust pre{\hypertarget{ref-Liu.2019}{}}%
Liu, H. and Lennartz, B.: Hydraulic properties of peat soils along a bulk density gradient-{A} meta study, Hydrological Processes, 33, 101--114, \url{https://doi.org/10.1002/hyp.13314}, 2019.

\leavevmode\vadjust pre{\hypertarget{ref-Pedersen.2020}{}}%
Pedersen, T. L.: {patchwork}: {The} composer of plots, 2020.

\leavevmode\vadjust pre{\hypertarget{ref-Prevost.1997}{}}%
Prevost, M., Belleau, P., and Plamondon, A. P.: Substrate conditions in a treed peatland: {Responses} to drainage, {É}coscience, 4, 543--554, \url{https://doi.org/10.1080/11956860.1997.11682434}, 1997.

\leavevmode\vadjust pre{\hypertarget{ref-RCoreTeam.2022}{}}%
R Core Team: R: {A} language and environment for statistical computing, Manual, R Foundation for Statistical Computing, Vienna, Austria, 2022.

\leavevmode\vadjust pre{\hypertarget{ref-Teickner.2025b}{}}%
Teickner, H. and Knorr, K.-H.: {hpmdpredict}: {Predictions} with model {HPMe-leaching} from {Teickner} et al. (2024), \url{https://doi.org/10.5281/ZENODO.14724314}, 2025.

\leavevmode\vadjust pre{\hypertarget{ref-Teickner.2025}{}}%
Teickner, H., Pebesma, E., and Knorr, K.-H.: A synthesis of {\emph{Sphagnum}} litterbag experiments: Initial leaching losses bias decomposition rate estimates, Biogeosciences, 22, 417--433, \url{https://doi.org/10.5194/bg-22-417-2025}, 2025.

\leavevmode\vadjust pre{\hypertarget{ref-Vehtari.2021}{}}%
Vehtari, A., Gelman, A., Simpson, D., Carpenter, B., and Bürkner, P.-C.: Rank-{Normalization}, {Folding}, and {Localization}: {An Improved R{\^{}}} for {Assessing Convergence} of {MCMC} (with {Discussion}), Bayesian Analysis, 16, \url{https://doi.org/10.1214/20-BA1221}, 2021.

\leavevmode\vadjust pre{\hypertarget{ref-Wickham.2016}{}}%
Wickham, H.: {ggplot2}: {Elegant} graphics for data analysis, 2nd ed. 2016., Springer International Publishing : Imprint: Springer, Cham, \url{https://doi.org/10.1007/978-3-319-24277-4}, 2016.

\leavevmode\vadjust pre{\hypertarget{ref-Wickham.2019}{}}%
Wickham, H., Averick, M., Bryan, J., Chang, W., McGowan, L., François, R., Grolemund, G., Hayes, A., Henry, L., Hester, J., Kuhn, M., Pedersen, T., Miller, E., Bache, S., Müller, K., Ooms, J., Robinson, D., Seidel, D., Spinu, V., Takahashi, K., Vaughan, D., Wilke, C., Woo, K., and Yutani, H.: Welcome to the {Tidyverse}, Journal of Open Source Software, 4, 1686, \url{https://doi.org/10.21105/joss.01686}, 2019.

\end{CSLReferences}

\end{document}
