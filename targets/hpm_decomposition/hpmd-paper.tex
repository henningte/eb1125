% Options for packages loaded elsewhere
\PassOptionsToPackage{unicode}{hyperref}
\PassOptionsToPackage{hyphens}{url}
%
\documentclass[
  12pt,
]{article}
\usepackage{amsmath,amssymb}
\usepackage{lmodern}
\usepackage{iftex}
\ifPDFTeX
  \usepackage[T1]{fontenc}
  \usepackage[utf8]{inputenc}
  \usepackage{textcomp} % provide euro and other symbols
\else % if luatex or xetex
  \usepackage{unicode-math}
  \defaultfontfeatures{Scale=MatchLowercase}
  \defaultfontfeatures[\rmfamily]{Ligatures=TeX,Scale=1}
\fi
% Use upquote if available, for straight quotes in verbatim environments
\IfFileExists{upquote.sty}{\usepackage{upquote}}{}
\IfFileExists{microtype.sty}{% use microtype if available
  \usepackage[]{microtype}
  \UseMicrotypeSet[protrusion]{basicmath} % disable protrusion for tt fonts
}{}
\makeatletter
\@ifundefined{KOMAClassName}{% if non-KOMA class
  \IfFileExists{parskip.sty}{%
    \usepackage{parskip}
  }{% else
    \setlength{\parindent}{0pt}
    \setlength{\parskip}{6pt plus 2pt minus 1pt}}
}{% if KOMA class
  \KOMAoptions{parskip=half}}
\makeatother
\usepackage{xcolor}
\IfFileExists{xurl.sty}{\usepackage{xurl}}{} % add URL line breaks if available
\IfFileExists{bookmark.sty}{\usepackage{bookmark}}{\usepackage{hyperref}}
\hypersetup{
  pdftitle={Underestimation of Anaerobic Decomposition Rates in Sphagnum Litterbag Experiments by the Holocene Peatland Model Depends on Initial Leaching Losses},
  pdfauthor={Henning Teickner1,2,; Edzer Pebesma2; Klaus-Holger Knorr1},
  pdfkeywords={Holocene Peatland Model, peatland, litterbag, decomposition, anoxic, oxic, Bayesian data analysis},
  hidelinks,
  pdfcreator={LaTeX via pandoc}}
\urlstyle{same} % disable monospaced font for URLs
\usepackage[margin=1in]{geometry}
\usepackage{longtable,booktabs,array}
\usepackage{calc} % for calculating minipage widths
% Correct order of tables after \paragraph or \subparagraph
\usepackage{etoolbox}
\makeatletter
\patchcmd\longtable{\par}{\if@noskipsec\mbox{}\fi\par}{}{}
\makeatother
% Allow footnotes in longtable head/foot
\IfFileExists{footnotehyper.sty}{\usepackage{footnotehyper}}{\usepackage{footnote}}
\makesavenoteenv{longtable}
\usepackage{graphicx}
\makeatletter
\def\maxwidth{\ifdim\Gin@nat@width>\linewidth\linewidth\else\Gin@nat@width\fi}
\def\maxheight{\ifdim\Gin@nat@height>\textheight\textheight\else\Gin@nat@height\fi}
\makeatother
% Scale images if necessary, so that they will not overflow the page
% margins by default, and it is still possible to overwrite the defaults
% using explicit options in \includegraphics[width, height, ...]{}
\setkeys{Gin}{width=\maxwidth,height=\maxheight,keepaspectratio}
% Set default figure placement to htbp
\makeatletter
\def\fps@figure{htbp}
\makeatother
\setlength{\emergencystretch}{3em} % prevent overfull lines
\providecommand{\tightlist}{%
  \setlength{\itemsep}{0pt}\setlength{\parskip}{0pt}}
\setcounter{secnumdepth}{5}
\newlength{\cslhangindent}
\setlength{\cslhangindent}{1.5em}
\newlength{\csllabelwidth}
\setlength{\csllabelwidth}{3em}
\newlength{\cslentryspacingunit} % times entry-spacing
\setlength{\cslentryspacingunit}{\parskip}
\newenvironment{CSLReferences}[2] % #1 hanging-ident, #2 entry spacing
 {% don't indent paragraphs
  \setlength{\parindent}{0pt}
  % turn on hanging indent if param 1 is 1
  \ifodd #1
  \let\oldpar\par
  \def\par{\hangindent=\cslhangindent\oldpar}
  \fi
  % set entry spacing
  \setlength{\parskip}{#2\cslentryspacingunit}
 }%
 {}
\usepackage{calc}
\newcommand{\CSLBlock}[1]{#1\hfill\break}
\newcommand{\CSLLeftMargin}[1]{\parbox[t]{\csllabelwidth}{#1}}
\newcommand{\CSLRightInline}[1]{\parbox[t]{\linewidth - \csllabelwidth}{#1}\break}
\newcommand{\CSLIndent}[1]{\hspace{\cslhangindent}#1}
\usepackage{float}
\usepackage[version=4]{mhchem}
\usepackage{booktabs}
\usepackage{bm}
\usepackage{xr} \externaldocument{hpmd-supporting-info}
\usepackage{lineno}
\linenumbers
\ifLuaTeX
  \usepackage{selnolig}  % disable illegal ligatures
\fi

\title{Underestimation of Anaerobic Decomposition Rates in \emph{Sphagnum} Litterbag Experiments by the Holocene Peatland Model Depends on Initial Leaching Losses}
\author{Henning Teickner\textsuperscript{1,2,*} \and Edzer Pebesma\textsuperscript{2} \and Klaus-Holger Knorr\textsuperscript{1}}
\date{24 May, 2024}

\begin{document}
\maketitle
\begin{abstract}
Testing peatland models is necessary to improve understanding and predictions of future peatland carbon dynamics. The Holocene Peatland Model (HPM) is widely applied, but currently insufficiently tested. Here, we test whether the HPM can predict decomposition of available \emph{Sphagnum} litterbag data along a gradient from oxic to anoxic conditions.\\
Large uncertainties in available litterbag data allow predictions of the HPM with standard parameter values to fit decomposition rates estimated from litterbags by adjusting initial leaching losses and decomposition rates estimated from the litterbag data within the range of their uncertainties. Therefore, improved tests of the HPM rely on future litterbag experiments that allow a more accurate estimation of initial leaching losses and decomposition rates.\\
Our analysis indicates that the HPM underestimates anaerobic decomposition rates for several species and assumes a too steep decrease of decomposition rates from oxic to anoxic conditions. This may be caused by not considering effects of water table fluctuations on aerobic and anaerobic decomposition rates.\\
Whether the estimated parameter values may be an easy fix to account for effects of water table fluctuations in long-term predictions needs further investigation. Based on previous sensitivity analyses of the HPM, the updated parameter estimates can cause differences in predicted C accumulation up to 100 kg.
\end{abstract}

\textsuperscript{1} ILÖK, Ecohydrology \& Biogeochemistry Group, Institute of Landscape Ecology, University of Münster, 48149, Germany\\
\textsuperscript{2} IfGI, Spatiotemporal Modelling Lab, Institute for Geoinformatics, University of Münster, 48149, Germany

\textsuperscript{*} Correspondence: \href{mailto:henning.teickner@uni-muenster.de}{Henning Teickner \textless{}\href{mailto:henning.teickner@uni-muenster.de}{\nolinkurl{henning.teickner@uni-muenster.de}}\textgreater{}}

Keywords: Holocene Peatland Model, peatland, litterbag, decomposition, anoxic, oxic, Bayesian data analysis

\hypertarget{introduction}{%
\section{Introduction}\label{introduction}}

Decomposition is one of the major controls of how much carbon (C) peatlands can store. Compared to other ecosystems, northern peatlands usually have small decomposition rates because of cold temperatures, high water table levels, acidic pH value, and litter that does not decompose fast even under more conditions facilitating microbial decomposition (\protect\hyperlink{ref-vanBreemen.1995}{van Breemen, 1995}; \protect\hyperlink{ref-Rydin.2013}{Rydin et al., 2013}). These slow decomposition rates caused northern peatlands to accumulate at least 400 Gt C (\protect\hyperlink{ref-Yu.2012}{Yu, 2012}; \protect\hyperlink{ref-Nichols.2019}{Nichols and Peteet, 2019}) during the Holocene and changes in the controls of decomposition rates may cause them to loose considerable amounts of C to the atmosphere under climate and land use changes (\protect\hyperlink{ref-Frolking.2011}{Frolking et al., 2011}; \protect\hyperlink{ref-Loisel.2017}{Loisel et al., 2017}). Peatland models are used to better understand past C accumulation and to predict future changes in peat C stocks, but because of the long time scales which have to be considered, they are difficult to test and currently are insufficiently tested.

How to test long-term peatland models is an open problem. Past studies have compared site-adapted simulations of peat height, age, C and N stocks, macrofossil composition, and water table level predicted by peatland models against peat core data (e.g., Frolking et al. (\protect\hyperlink{ref-Frolking.2010}{2010}), Tuittila et al. (\protect\hyperlink{ref-Tuittila.2013}{2013}), Treat et al. (\protect\hyperlink{ref-Treat.2021}{2021}), Zhao et al. (\protect\hyperlink{ref-Zhao.2022}{2022})), and have shown that existing peatland models can reproduce observed patterns to some extent.

These tests suffer from two problems. First, they test entire peatland models against observed data and thus can identify the parameter values or model equations that cause observed discrepancies less reliably. Second, there often are large uncertainties on both sides of the test; peatland models have large uncertainties in parameter values and model structure and these may produce a range of predictions as illustrated by uncertainty analyses (e.g. Quillet et al. (\protect\hyperlink{ref-Quillet.2013}{2013a}), Quillet et al. (\protect\hyperlink{ref-Quillet.2013a}{2013b})) and model intercomparisons (e.g. Zhao et al. (\protect\hyperlink{ref-Zhao.2022}{2022})). Observed data also has uncertainty from measurements, peat dating, or simply missing data, for example for past precipitation. Large uncertainties can make tests inconclusive, no matter how much data we use. An alternative which avoids some of these problems is to test only some part of a model while taking into account relevant uncertainty sources.

Such a test could address the decomposition module of a peatland model. For example, in the Holocene Peatland Model (HPM) (\protect\hyperlink{ref-Frolking.2010}{Frolking et al., 2010}), we only need to know litter species, peat water content, peat porosity, water table depth, and only five parameters to predict decomposition rates. The predictions can be compared to decomposition rates estimated from litterbag data and therefore future litterbag studies can directly test whether discrepancies are replicable and identify the factors causing the discrepancies. Admittedly, such a test is restricted to short time ranges and not representative for long-term decomposition rates, but future tests with different scope will benefit from the reduced parameter uncertainties and can consider where the model fails already on short time scales.

A test of decomposition modules is relevant because of the importance of decomposition for long-term C accumulation in peatlands. Previous sensitivity analyses of the HPM and applications to peat cores suggest that the anoxia scale length (\(c_2\)), the parameter controlling how anaerobic decomposition rates are limited by electron acceptor depletion and accumulation of decomposition products, can result in a doubling of accumulated C, depending on climate conditions (\protect\hyperlink{ref-Quillet.2013a}{Quillet et al., 2013b}; \protect\hyperlink{ref-Kurnianto.2015}{Kurnianto et al., 2015}). A test of only the HPM decomposition module can provide better estimates for \(c_2\) and may therefore help to reduce uncertainties in predicted C accumulation rates.

Currently, litterbag experiments are not as extensively used for testing peatland models as they could and only a fraction of the information available from litterbag experiments is used to develop models. The HPM uses litterbag data to define average decomposition rates of moss plant functional types, but parameters for environmental controls of decomposition are assumptions which appear to be informed at most qualitatively by litterbag experiments, and it is not tested whether the HPM successfully fits available litterbag data (\protect\hyperlink{ref-Frolking.2010}{Frolking et al., 2010}). This is also the case for other dynamic peatland models, e.g. Frolking et al. (\protect\hyperlink{ref-Frolking.2001}{2001}), Bauer (\protect\hyperlink{ref-Bauer.2004}{2004}), Heijmans et al. (\protect\hyperlink{ref-Heijmans.2008}{2008}), Heinemeyer et al. (\protect\hyperlink{ref-Heinemeyer.2010}{2010}), Morris et al. (\protect\hyperlink{ref-Morris.2012}{2012}), Chaudhary et al. (\protect\hyperlink{ref-Chaudhary.2018}{2018}), Bona et al. (\protect\hyperlink{ref-Bona.2020}{2020}).

One reason why such tests have been difficult is that suitable litterbag raw data to test peatland models are scarce. Bona et al. (\protect\hyperlink{ref-Bona.2018}{2018}) developed a Peatland Productivity and Decomposition Parameter Database, but it contains only data from studies older than 2010 and no error estimates for remaining masses in litterbag data. Since decomposition rates have been estimated with different litterbag decomposition models in previous studies, their values are not directly comparable and therefore raw data are necessary to obtain estimates directly comparable to predictions from a certain peatland model (\protect\hyperlink{ref-Yu.2001}{Yu et al., 2001}; \protect\hyperlink{ref-Teickner.2024}{Teickner et al., 2024}). Recently, we used available \emph{Sphagnum} litterbag data to estimate decomposition rates which can be directly compared to decomposition rates predicted by the HPM (\protect\hyperlink{ref-Teickner.2024}{Teickner et al., 2024}).

Even though tests of only a part of a model are less uncertain than tests of whole models, there still is a risk that they are dominated by uncertainties. Remaining masses in litterbag experiments are often very variable, even under controlled environmental conditions (e.g. Bengtsson et al. (\protect\hyperlink{ref-Bengtsson.2018}{2018})), and for many litterbag experiments, a range of decomposition rates may produce similar predictions for remaining masses if a litterbag decomposition model compatible with the HPM is used (\protect\hyperlink{ref-Teickner.2024}{Teickner et al., 2024}). Finally, also only five model parameters, as in the case of the HPM decomposition module, can make predictions uncertain. These uncertainties have to be taken into account to check whether litterbag data are compatible with the peatland model. A possible way to do this is to combine the HPM decomposition module, the litterbag decomposition model from our previous study, and available litterbag experiments into one model and use Bayesian data analysis (\protect\hyperlink{ref-Gelman.2014}{Gelman et al., 2014}) to estimate uncertainties of data and parameters.

If such a test suggests that decomposition rates predicted by the HPM do not fit estimates from litterbag experiments even if main uncertainty sources are considered, we have identified a discrepancy worth considering in more detail. We can then identify how the estimated parameter values differ from the standard values and analyze whether previous sensitivity analyses of the HPM suggest that these discrepancies may have larger effects on the predicted C accumulation.

Our aim is to test the HPM decmposition module against decomposition rates estimated from available \emph{Sphagnum} litterbag experiments. Specifically, we want to:

\begin{enumerate}
\def\labelenumi{\arabic{enumi}.}
\item
  Test whether the HPM can predict litterbag decomposition rates for different \emph{Sphagnum} species along the gradient from oxic to anoxic conditions.
\item
  Test whether HPM parameters estimated from litterbag data are compatible with the values originally proposed in the HPM (standard parameter values) (Tab. \ref{tab:hpmd-m-tab-hpm-parameters-standard-values}).
\end{enumerate}

We test the following hypotheses:

\begin{enumerate}
\def\labelenumi{\arabic{enumi}.}
\item
  The HPM can successfully predict decomposition rates estimated from litterbag data under oxic and anoxic conditions.
\item
  HPM parameter values (\(k_{i,0}\), \(W_{opt}\), \(c_1\), \(f_{min}\), \(c_2\)) estimated from litterbag experiments are compatible with the standard values.
\end{enumerate}

To address these aims, we developed a model that combines the HPM decomposition module and our previous \emph{Sphagnum} litterbag decomposition model, which estimates decomposition rates in available litterbag experiments while considering initial leaching losses (\protect\hyperlink{ref-Teickner.2024}{Teickner et al., 2024}). Estimated decomposition rates of this model can be directly compared to decomposition rates predicted by the HPM because the formula to compute remaining masses from decomposition rates is the same.

We only test the decomposition module of the HPM, but our results are valuable also for other peatland models that parameterize their decomposition modules from litterbag experiments because they also require a correct representation of how decomposition rates are controlled by the water table level. Our test identified discrepancies between the HPM and litterbag data that could give novel insights into processes controlling anaerobic decomposition rates in future litterbag experiments.\\


\begin{table}[!h]

\caption{\label{tab:hpmd-m-tab-hpm-parameters-standard-values}Standard values of parameters of the decomposition module in the Holocene Peatland Model (\protect\hyperlink{ref-Frolking.2010}{Frolking et al., 2010}).}
\centering
\resizebox{\linewidth}{!}{
\begin{tabular}[t]{lr>{\raggedright\arraybackslash}p{12cm}}
\toprule
HPM parameter & Standard value & Description\\
\midrule
$W_{opt}$ (L$_\mathrm{water}$ L$_\mathrm{pores}^{-1}$) & 0.450 & Optimum degree of saturation for aerobic decomposition.\\
$c_1$ (-) & 2.310 & Curvature of the relation of the aerobic decomposition rate to the degree of saturation (larger values imply a steeper decrease of decomposition rates for degrees of saturation diverging from $W_{opt}$).\\
$f_{min}$ (yr$^{-1}$) & 0.001 & Minimum anaerobic decomposition rate.\\
$c_2$ (m) & 0.300 & Anoxia scale length. Represents limitation of anaerobic decomposition rates with increasing distance below the annual average water table depth due to end product accumulation and limitation of available electron acceptors. Larger values mean that anaerobic decomposition rates decrease less strongly with depth below the average annual water table level.\\
$k_{0, \text{hollow}}$ (yr$^{-1}$) & 0.130 & Maximum possible decomposition rate for hollow $Sphagnum$ species.\\
\addlinespace
$k_{0, \text{lawn}}$ (yr$^{-1}$) & 0.080 & Maximum possible decomposition rate for lawn $Sphagnum$ species.\\
$k_{0, \text{hummock}}$ (yr$^{-1}$) & 0.060 & Maximum possible decomposition rate for hummock $Sphagnum$ species.\\
\bottomrule
\end{tabular}}
\end{table}

\hypertarget{sdm-003-methods}{%
\section{Methods}\label{sdm-003-methods}}

\hypertarget{sdm-003-methods-1}{%
\subsection{\texorpdfstring{\emph{Sphagnum} litterbag data}{Sphagnum litterbag data}}\label{sdm-003-methods-1}}

To test the HPM against litterbag data, we use the Peatland Decomposition Database (\protect\hyperlink{ref-Teickner.2024c}{Teickner and Knorr, 2024b}). In this study, we use data from Bartsch and Moore (\protect\hyperlink{ref-Bartsch.1985}{1985}), Vitt (\protect\hyperlink{ref-Vitt.1990}{1990}), Johnson and Damman (\protect\hyperlink{ref-Johnson.1991}{1991}), Szumigalski and Bayley (\protect\hyperlink{ref-Szumigalski.1996}{1996}), Prevost et al. (\protect\hyperlink{ref-Prevost.1997}{1997}), Scheffer et al. (\protect\hyperlink{ref-Scheffer.2001}{2001}), Thormann et al. (\protect\hyperlink{ref-Thormann.2001}{2001}), Asada and Warner (\protect\hyperlink{ref-Asada.2005b}{2005}), Trinder et al. (\protect\hyperlink{ref-Trinder.2008}{2008}), Breeuwer et al. (\protect\hyperlink{ref-Breeuwer.2008}{2008}), Straková et al. (\protect\hyperlink{ref-Strakova.2010}{2010}), Hagemann and Moroni (\protect\hyperlink{ref-Hagemann.2015}{2015}), Golovatskaya and Nikonova (\protect\hyperlink{ref-Golovatskaya.2017}{2017}), and Mäkilä et al. (\protect\hyperlink{ref-Makila.2018}{2018}) to estimate litterbag decomposition rates and predicted \(k_0\) were tested against \(k_0\) estimated from Johnson and Damman (\protect\hyperlink{ref-Johnson.1991}{1991}), Szumigalski and Bayley (\protect\hyperlink{ref-Szumigalski.1996}{1996}), Prevost et al. (\protect\hyperlink{ref-Prevost.1997}{1997}), Straková et al. (\protect\hyperlink{ref-Strakova.2010}{2010}), Golovatskaya and Nikonova (\protect\hyperlink{ref-Golovatskaya.2017}{2017}), and Mäkilä et al. (\protect\hyperlink{ref-Makila.2018}{2018}) because only these studies reported water table depths required to make predictions with the HPM. Samples originally classified as \emph{Sphagnum magellanicum} are here classified as \emph{Sphagnum magellanicum aggr.} (\protect\hyperlink{ref-Hassel.2018}{Hassel et al., 2018}).

\hypertarget{sdm-003-methods-2}{%
\subsection{Prediction of litterbag decomposition rates with the Holocene Peatland Model}\label{sdm-003-methods-2}}

To predict decomposition rates, the HPM decomposition module needs as inputs the litter type in terms of the HPM plant functional types, the fraction of mass already lost due to previous decomposition, the depth of the litter below the peat surface, the water table depth, and the peat degree of saturation (\protect\hyperlink{ref-Frolking.2010}{Frolking et al., 2010}).

Predicting decomposition rates for the available litterbag data is not straightforward because the HPM does not consider specific features of litterbag experiments, because it does not specify how to assign species to plant functional types, and because required variables such as the degree of saturation are not reported in the litterbag studies and therefore need to be estimated. In addition, we need to link decomposition rates estimated from litterbag data to the decomposition rates predicted by the HPM and this requires to link remaining masses in litterbag experiments to decomposition rates.

The only variables that can be directly linked are the depth of the litter below the peat surface, water table depths (both reported in litterbag experiments). All other variables need additional assumptions that we describe in the following subsections.

\hypertarget{sdm-003-methods-3}{%
\subsubsection{Remaining masses and decomposition rates}\label{sdm-003-methods-3}}

In a previous study, we estimated \(k_0\) for the litterbag data using the decomposition equation of the HPM (equation (7) in Frolking et al. (\protect\hyperlink{ref-Frolking.2010}{2010})) and in addition considering initial leaching losses to avoid bias of \(k_0\) estimates (\protect\hyperlink{ref-Teickner.2024}{Teickner et al., 2024}):

\begin{equation}
m(t) = \begin{cases}
m_0 & \mathrm{if}~t=0\\
\frac{m_0 - l_0}{(1 + (\alpha - 1) k_0 t)^{\frac{1}{\alpha - 1}}} & \mathrm{if}~t>0\\
\end{cases}
\label{eq:decomposition-solution-2-with-leaching-1}
\end{equation}

Where \(m(t)\) is the remaining mass at time \(t\), \(m_0\) is the mass at time \(t=0\) (the initial mass), \(l_0\) is the initial mass loss due to leaching and respiration of soluble compounds, \(k_0\) is the decomposition rate of litter with no prior decomposition, \(\alpha\) controls how the decomposition rate decreases as the fraction of remaining mass decreases and is assumed to describe how decomposition rates decrease with decreasing litter quality over time (\protect\hyperlink{ref-Frolking.2001}{Frolking et al., 2001}).

\hypertarget{sdm-003-methods-4}{%
\subsubsection{\texorpdfstring{Assignment of \emph{Sphagnum} species to PFT}{Assignment of Sphagnum species to PFT}}\label{sdm-003-methods-4}}

The HPM defines maximum possible decomposition rates (\(k_{i,0}\)) for three \emph{Sphagnum} PFT (hollow, lawn, and hummock species), but not how to assign species to them. We assigned individual \emph{Sphagnum} species to the three PFT by comparing their niche water table depths with the optimal water table depth for net primary production defined in the HPM. Specifically, we defined fixed average annual water table depth intervals for the PFT: hollow (\textless5 cm), lawn (\(\ge5\) cm and \(<15\) cm), hummock (\(\ge15\) cm). Then, we used niche water table depths and standard deviations from Johnson et al. (\protect\hyperlink{ref-Johnson.2015}{2015}) to assign \emph{Sphagnum} species to these three microhabitats. Using only average values and the microhabitat water table depth thresholds resulted in unintuitive assignments, such as assigning \emph{S. fallax} to hummocks. To avoid such obvious misclassifications, we defined rules to assign species to HPM microhabitats based on the probability a species would occur in the three niche water table depth intervals. To compute the probabilities, we assumed a normal distribution (\protect\hyperlink{ref-Johnson.2015}{Johnson et al., 2015}):

\begin{enumerate}
\def\labelenumi{\arabic{enumi}.}
\item
  Species with a probability of occurrence \(\ge15\)\% in the intervals of all three PFT were classified as lawn species.
\item
  In all other cases, species were assigned to the PFT for which their probability of occurrence was largest.
\end{enumerate}

Litterbag data from Prevost et al. (\protect\hyperlink{ref-Prevost.1997}{1997}) are incubations of peat samples where the species is unknown. Based on descriptions in the paper, it is likely that the peat was formed by hummock species. In addition, decomposition rate estimates for these samples are small. For these reasons, we assigned these samples to the hummock PFT of the HPM.

When estimating parameters of the HPM from the litterbag data (see section \ref{sdm-003-methods-8}), we also estimated the maximum possible decomposition rate (\(k_{i,0}\)). \emph{Sphagnum} species differ in their decomposition rate and the PFT of the HPM are a simplification which may cause misfits of the HPM to litterbag data. We therefore estimated \(k_{i,0}\) for individual \emph{Sphagnum} species in models HPMe-LE-peat and HPMe-LE-peat-l0.

\hypertarget{sdm-003-methods-5}{%
\subsubsection{Degree of saturation}\label{sdm-003-methods-5}}

We estimated the degree of saturation with the modified Granberg model (ModGberg model) (\protect\hyperlink{ref-Granberg.1999}{Granberg et al., 1999}; \protect\hyperlink{ref-Kettridge.2007}{Kettridge and Baird, 2007}) from total porosity, the water table depth, and the positions of the litterbags during the incubation. The total porosity was not reported in any study and therefore we assumed an average value of 80\% with a standard deviation of 10\%, roughly based on values reported for low-density \emph{Sphagnum} peat (\protect\hyperlink{ref-Liu.2019}{Liu and Lennartz, 2019}).

\hypertarget{sdm-003-methods-6}{%
\subsubsection{Fraction of mass lost during previous decomposition}\label{sdm-003-methods-6}}

The HPM assumes that decomposition rates decrease the more of the initial mass has already been decomposed (\protect\hyperlink{ref-Frolking.2001}{Frolking et al., 2001}; \protect\hyperlink{ref-Frolking.2010}{Frolking et al., 2010}). All litterbag data we use here, except samples from Prevost et al. (\protect\hyperlink{ref-Prevost.1997}{1997}), are from \emph{Sphagnum} samples collected from the surface of peatlands and therefore can be expected to have not experienced mass loss due to decomposition at the start of the experiments. Prevost et al. (\protect\hyperlink{ref-Prevost.1997}{1997}) incubated \emph{Sphagnum} peat collected from different depths below the surface and these samples probably have already experienced some decomposition, however it is difficult to estimate how much. To avoid this problem, we estimated \(k_{i,0}\) separately for samples from different depths in Prevost et al. (\protect\hyperlink{ref-Prevost.1997}{1997}), implicitly assuming that these are two different PFT with different maximum possible decomposition rate.

\hypertarget{sdm-003-methods-7}{%
\subsection{Testing the HPM against litterbag data}\label{sdm-003-methods-7}}

\hypertarget{sdm-003-methods-8}{%
\subsubsection{Model versions}\label{sdm-003-methods-8}}

To test different aspects of the HPM and the additional assumptions we introduce, we computed several models which differ in whether HPM parameters were fixed to their standard values or estimated from data, whether peat properties (porosity, water content, minimum water content at the surface) are estimated from data or not, whether the litterbag decomposition model and the HPM decomposition module were estimated in two separate Bayesian models or one combined model, and whether the HPM decomposition module was extended to also predict \(l_0\) or not (Tab. \ref{tab:m-litterbag-synthesis-models}).

The first model (HPMf) does not estimate any parameters of the HPM (except for \(\alpha\)) and does not estimate peat properties from the litterbag data and therefore corresponds to the HPM decomposition module with standard parameter values. Values of \(k_0\) are predicted independently from the litterbag decomposition model.

Each subsequent model combines the HPM decomposition module and the litterbag decomposition model into one Bayesian model. Each of these models estimates an additional set of parameters from the litterbag data relative to the previous model (Tab. \ref{tab:m-litterbag-synthesis-models}). First, only the peat properties (HPMf-LE-peat) are estimated, and second all HPM parameters (\(k_{i,0}\), \(c_1\), \(W_{opt}\), \(f_{min}\), \(c_2\)) (HPMe-LE-peat). Finally, HPMe-LE-peat-l0 extends HPMe-LE-peat by adding formulas to model how \(l_0\) depends on the degree of saturation, similar to how the HPM predicts \(k_0\).

It is important to note that combining the litterbag decomposition model and the HPM decomposition module into one Bayesian model does not only estimate HPM parameters from the litterbag data, but it also adjusts the decomposition rates estimated from litterbag data to the HPM: The HPM serves as prior in the combined model and Bayesian probability theory estimates what parameter values are compatible with the data \emph{and} the combined model. This is exactly what we want because there is uncertainty both in the remaining masses reported in litterbag experiments and in HPM parameters. If HPM parameter estimates from the combined model are not compatible with standard values used in the original model (Tab. \ref{tab:hpmd-m-tab-hpm-parameters-standard-values}) even if we adjust them to the HPM within the range allowed by the uncertainties, this is a discrepancy worth testing in future experiments.

HPMf-LE-peat tested whether the HPM can be made compatible with available litterbag data when the HPM decomposition module and the decomposition model for litterbag data are combined and when uncertain peat properties are estimated from data.

HPMe-LE-peat estimates what HPM parameter values are compatible with available litterbag data and therefore allows to test whether the standard parameter values are extreme relative to these estimates. Values of \(k_{i,0}\) were estimated for each species separately, as described in section \ref{sdm-003-methods-4}.

HPMe-LE-peat-l0 was computed because decomposition rates estimated from available litterbag experiments are sensitive to initial leaching losses (\protect\hyperlink{ref-Yu.2001}{Yu et al., 2001}; \protect\hyperlink{ref-Lind.2022}{Lind et al., 2022}; \protect\hyperlink{ref-Teickner.2024}{Teickner et al., 2024}). It is therefore interesting to see whether litterbag decomposition rates are adjusted differently in HPMe-LE-peat-l0 --- when initial leaching losses are constrained --- compared to HPMe-LE-peat --- when initial leaching losses can be estimated more independently for each replicate. Based on previous experiments with tea bags it is reasonable to assume that there is some relation between initial leaching losses and the degree of saturation (\protect\hyperlink{ref-Lind.2022}{Lind et al., 2022}).

To check whether outliers in the litterbag data could influence our results, we computed one additional model, HPMe-LE-peat-l0-outlier, with the same structure as HPMe-LE-peat-l0, but estimated without littebag experiments identified as outliers. Litterbag experiments were defined as outliers if the reported average remaining mass of any litterbag (batch) during the experiment had a posterior probability \(>99\)\% to be different from the remaining mass predicted by the litterbag decomposition model alone. This procedure identified experiments as outliers where remaining masses increased over time, where litterbags collected at intermediate time points had unexpectedly low remaining masses, or where initial leaching losses were retarded to later time points, presumably because of freezing after the start of the experiment (\protect\hyperlink{ref-Teickner.2024}{Teickner et al., 2024}). In total, 5 litterbag experiments were identified as outliers. Results for HPMe-LE-peat-l0-outlier are shown in supporting information \ref{sup-10} and HPM parameter estimates agree with the other models where HPM parameters were estimated.

Strictly, we do not test the decomposition module in the HPM, but the combination of the decomposition model in the HPM and the modified Granberg model, assuming that uncertainties in water table depths are negligible and that we accounted sufficiently for uncertainties in total porosity. This ambiguity has to be accepted when combining heterogeneous litterbag data where some variables have to be estimated. Litterbag experiments where the degree of saturation is measured would be needed to avoid this ambiguity.











\begin{table}[!h]

\caption{\label{tab:m-litterbag-synthesis-models}Overview of HPM modifications computed in this study. Complete formulas for the models are shown in supporting information \ref{sup-1}.}
\centering
\resizebox{\linewidth}{!}{
\begin{tabular}[t]{l>{\raggedright\arraybackslash}p{12cm}}
\toprule
Model & Description\\
\midrule
HPMf & Decomposition model from the Holocene Peatland model with default parameter values (\protect\hyperlink{ref-Frolking.2010}{Frolking et al., 2010}). The model is run with peat water contents estimated with the modified Granberg model, using water table depths and litterbag depths reported from the litterbag studies, and assuming a fixed peat porosity, and minimum peat water content at the surface. Litterbag decomposition rate estimates are from the litterbag decomposition model in Teickner et al. (\protect\hyperlink{ref-Teickner.2024}{2024}).\\
HPMf-LE-peat & The same as HPMf, but combined with the litterbag decomposition model into one Bayesian model. Water table depths, peat porosity, and minimum peat water content at the surface were estimated from data.\\
HPMe-LE-peat & The same as HPMf-LE-peat, but now also parameters from the HPM decomposition model ($k_{i,0}$, $W_{opt}$, $f_{min}$, $c_1$, $c_2$) are estimated from the litterbag data.\\
HPMe-LE-peat-l0 & The same as HPMe-LE-peat, but now also an average initial leaching loss for each species and, across all species, a factor by which this average leaching loss increases or decreases as the peat degree of saturation increases are estimated.\\
HPMe-LE-peat-l0-outlier & The same as HPMe-LE-peat-l0, but computed without litterbag experiments that were identified as outliers and for which the HPM decomposition module predicts decomposition rates (see the text for details).\\
\bottomrule
\end{tabular}}
\end{table}

\hypertarget{sdm-003-methods-9}{%
\subsubsection{Tests of model fits to litterbag decomposition rates and comparison between estimated and standard HPM parameter values}\label{sdm-003-methods-9}}

For each model, we computed the difference of the decomposition rate predicted by the HPM and estimated from the litterbag data for each litterbag replicate and from this the average. We then computed the posterior probability that this average difference is different from zero. A small probability indicates a misfit of the model to available litterbag data. We also tested the same difference for specific species because graphical checks indicated that the decomposition rate prediction skill of the HPM depends on species.

For HPMe-LE-peat and HPMe-LE-peat-l0, we computed the posterior probability that that the HPM parameter values estimated from litterbag data (\(k_{i,0}\), \(c_1\), \(W_{opt}\), \(c_2\), \(f_{min}\)) differ from the standard parameter values (Tab. \ref{tab:hpmd-m-tab-hpm-parameters-standard-values}).

\hypertarget{sdm-003-methods-11}{%
\subsubsection{Predictive accuracy of the modified Holocene Peatland Model in comparison to the original model}\label{sdm-003-methods-11}}

To test whether HPMe-LE-peat-l0 has not only a better fit to available litterbag data, but also a better predictive accuracy for novel data than the model with standard parameter values (HPMf), we compared how well both can predict the one-pool decomposition rates from litterbag experiments.

HPM parameters of HPMf are not estimated from data and therefore we could compute the root mean square error of prediction (RMSE\(_\text{test}\)) directly with \(k_0\) predicted by HPMf and estimated with the litterbag decomposition model. HPM parameters of HPMe-LE-peat-l0 are estimated from the litterbag data and we therefore used cross-validation (CV) to estimate RMSE\(_\text{test}\).

Since decomposition rates form the same species and study usually are not independent, we defined blocks which were used as CV-folds. Each fold consists of the data from one study, except those values that were measured for \emph{Sphagnum} species for which only this study had data (we want to estimate the predictive accuracy not for new species). Data for species with data from one study only were always used for model training and not part of the testing folds. This procedure resulted in 5 folds. HPMf and HPMe-LE-peat-l0 were tested against the same data.

In the text, RMSE\(_\text{train}\) is the RMSE computed with the data a model was estimated with (for HPMf, the data the litterbag decomposition model was estimated with), and RMSE\(_\text{test}\) is the RMSE computed with independent test data.

\hypertarget{sdm-003-methods-13}{%
\subsection{Bayesian data analysis}\label{sdm-003-methods-13}}

All models listed in Tab. \ref{tab:m-litterbag-synthesis-models} were computed with Bayesian statistics to account for relevant uncertainty sources and include relevant prior knowledge (for example that \emph{Sphagnum} decomposition rates are unlikely to be larger than 0.5 yr\(^{-1}\)). Bayesian computations were performed using Markov Chain Monte Carlo (MCMC) sampling with Stan (2.32.2) (\protect\hyperlink{ref-StanDevelopmentTeam.2021a}{Stan Development Team, 2021b}) and rstan (2.32.5) (\protect\hyperlink{ref-StanDevelopmentTeam.2021b}{Stan Development Team, 2021a}) using the NUTS sampler (\protect\hyperlink{ref-Hoffman.2014}{Hoffman and Gelman, 2014}), with four chains, 4000 total iterations per chain, and 2000 warmup iterations per chain. None of the models had divergent transitions, the minimum bulk effective sample size was larger than 400, and the largest improved \(\hat{R}\) was 1.01, indicating that all chains converged (\protect\hyperlink{ref-Vehtari.2021}{Vehtari et al., 2021}). All models used the same priors for the same parameters and prior choices are listed and justified in supporting Tab. \ref{tab:sup-out-d-sdm-all-models-priors-1}.

We used power-scaling of the prior and likelihood distributions as implemented in the priorsense package (0.0.0.9000) (\protect\hyperlink{ref-Kallioinen.2024}{Kallioinen et al., 2024}) to analyze the relative sensitivity of the posterior distribution to small perturbations of the prior and likelihood in HPMe-LE-peat-l0 for HPM parameters and peat properties. This is a computationally nonexpensive way to check whether the data provide information about a parameter and where prior and data may provide conflicting information (\protect\hyperlink{ref-Kallioinen.2024}{Kallioinen et al., 2024}). Results of this analysis and further information on the data analysis are shown in supporting information \ref{sup-12}.

\hypertarget{results}{%
\section{Results}\label{results}}

\hypertarget{fit-and-predictive-accuracy-of-the-different-versions-of-the-holocene-peatland-model-to-available-litterbag-data}{%
\subsection{Fit and predictive accuracy of the different versions of the Holocene Peatland Model to available litterbag data}\label{fit-and-predictive-accuracy-of-the-different-versions-of-the-holocene-peatland-model-to-available-litterbag-data}}

The HPM decomposition module with standard parameter values (HPMf) fitted decomposition rates estimated from litterbag data to variable degrees (Tab. \ref{tab:out-tab-sdm-all-models-rmse-1}). All other models had an improved overall fit (smaller RMSE\(_\text{train}\)) to the data (Tab. \ref{tab:out-tab-sdm-all-models-rmse-1}, Fig. \ref{fig:out-p-hpm-mm27-2-mm29-1-mm30-1-p1}). Despite better fitting the data, HPMe-LE-peat-l0 did not predict \(k_0\) better in the cross-validation than HPMf, as indicated by a large RMSE\(_\text{test}\) (Tab. \ref{tab:out-tab-sdm-all-models-rmse-1}).

Errors of HPMf differed between species (Fig. \ref{fig:out-p-hpm-mm27-2-mm29-1-mm30-1-p2}). They were particularly small for \emph{S. fuscum} (RMSE\(_\text{train}\) = 0.02 yr\(^{-1} \pm\) 0.004, data from 5 studies) as well as \emph{Sphagnum} spec. samples from Prevost et al. (\protect\hyperlink{ref-Prevost.1997}{1997}) (RMSE\(_\text{train}\) = 0.02 yr\(^{-1} \pm\) 0). All rates were underestimated for \emph{S. angustifolium} (RMSE\(_\text{train}\) = 0.23 yr\(^{-1} \pm\) 0.09, data from 3 studies).



\begin{table}[!h]

\caption{\label{tab:out-tab-sdm-all-models-rmse-1}Training and testing RMSE for decomposition rates as predicted by different versions of the Holocene Peatland Model (see Tab. \ref{tab:m-litterbag-synthesis-models} for a description of the models) and number of misfits. RMSE\(_\text{train}(k_0)\) is the root mean square error of model predictions for litterbag replicates used during model computation. RMSE\(_\text{test}(k_0)\) is the RMSE for litterbag replicates used in blocked cross-validation. Where no RMSE\(_\text{test}(k_0)\) is given, it was not computed for these models. Values are averages and lower and upper bounds of central 95\% uncertainty intervals (yr\(^{-1}\)). Misfits counts the number of litterbag experiments for which \(k_0\) predicted by the HPM modification differed from \(k_0\) as estimated from the litterbag decomposition model with a posterior probability of at least 99\%. In total, \(k_0\) was predicted with the HPM modifications for 53 litterbag experiments (RMSE\(_\text{train}(k_0)\)) or 29 (RMSE\(_\text{test}(k_0)\)).}
\centering
\begin{tabular}[t]{lllr}
\toprule
Model & RMSE$_\text{train}(k_0)$ & RMSE$_\text{test}(k_0)$ & Misfits\\
\midrule
HPMf & 0.105 (0.051, 0.191) & 0.136 (0.06, 0.252) & 13\\
HPMf-LE-peat & 0.02 (0.013, 0.029) &  & 0\\
HPMe-LE-peat & 0.014 (0.008, 0.021) &  & 0\\
HPMe-LE-peat-l0 & 0.022 (0.012, 0.039) & 0.088 (0.038, 0.179) & 0\\
HPMe-LE-peat-l0-outlier & 0.021 (0.013, 0.032) &  & 0\\
\bottomrule
\end{tabular}
\end{table}



\begin{figure}[H]

{\centering \includegraphics[width=1\linewidth]{figures/hpmd_plot_1} 

}

\caption{Comparison of \(k_0\) estimated by the litterbag decomposition model versus \(k_0\) predicted by different modifications of the HPM decomposition module (Tab. \ref{tab:m-litterbag-synthesis-models}). Points are colored according to the microhabitat classification of \emph{Sphagnum species} (see the Methods section for details). Error bars exceeding 0.5 yr\(^{-1}\) are clipped.}\label{fig:out-p-hpm-mm27-2-mm29-1-mm30-1-p1}
\end{figure}

\hypertarget{differences-in-model-behavior-of-the-holocene-peatland-model-and-its-modifications}{%
\subsection{Differences in model behavior of the Holocene Peatland Model and its modifications}\label{differences-in-model-behavior-of-the-holocene-peatland-model-and-its-modifications}}

\hypertarget{the-hpm-with-standard-parameter-values-can-fit-litterbag-data-due-to-large-uncertainties-in-available-litterbag-data.}{%
\paragraph*{The HPM with standard parameter values can fit litterbag data due to large uncertainties in available litterbag data.}\label{the-hpm-with-standard-parameter-values-can-fit-litterbag-data-due-to-large-uncertainties-in-available-litterbag-data.}}
\addcontentsline{toc}{paragraph}{The HPM with standard parameter values can fit litterbag data due to large uncertainties in available litterbag data.}

HPMf-LE-peat suggests that it is possible to fit remaining masses in litterbag experiments without changing the standard HPM parameter values, simply by adjusting \(k_0\) and \(l_0\) estimates from the litterbag decomposition model such that they fit the HPM predictions. Fig. \ref{fig:out-p-hpm-mm27-2-mm29-1-mm30-1-p1} shows that HPMf-LE-peat can reproduce these adjusted \(k_0\) estimates. Fig. \ref{fig:out-p-hpm-mm27-2-mm29-1-mm30-1-p2} shows that this better fit is mainly achieved by adjusting \(k_0\) estimates from the litterbag decomposition model (mainly decreased) to the HPM and not because of differences in peat properties estimated from the litterbag data. In combination with the improved fit of HPMf-LE-peat, this indicates that uncertainties in the litterbag data are large enough to make the HPM compatible with the litterbag decomposition rates by varying the magnitude of decomposition rates and initial leaching losses, even though the standard HPM parameters are not necessarily (most) compatible with the data. This indicates that a better test of the HPM decomposition module requires more accurate estimates of initial leaching losses.

\hypertarget{estimates-for-k_0-l_0-and-k_i0-differ-between-modifications-of-the-hpm.}{%
\paragraph*{\texorpdfstring{Estimates for \(k_0\), \(l_0\), and \(k_{i,0}\) differ between modifications of the HPM.}{Estimates for k\_0, l\_0, and k\_\{i,0\} differ between modifications of the HPM.}}\label{estimates-for-k_0-l_0-and-k_i0-differ-between-modifications-of-the-hpm.}}
\addcontentsline{toc}{paragraph}{Estimates for \(k_0\), \(l_0\), and \(k_{i,0}\) differ between modifications of the HPM.}

The two modifications of the HPM where HPM parameters were estimated from litterbag data (HPMe-LE-peat, HPMe-LE-peat-l0) also differed in the magnitude of \(l_0\) and \(k_0\) estimates, as well as the maximum possible initial decomposition rate for each species (\(k_{0,i}\)). However, they had very similar estimates for the other HPM parameters (\(c_1\), \(W_{opt}\), \(f_{min}\), \(c_2\)).

HPMe-LE-peat estimated larger initial leaching losses and smaller decomposition rates than the litterbag decomposition model from Teickner et al. (\protect\hyperlink{ref-Teickner.2024}{2024}) alone, similar to HPMf-LE-peat (Fig. \ref{fig:out-sdm-all-models-p2}). This is particularly the case for \emph{S. angustifolium}, for which the separate litterbag decomposition model estimated much larger average decomposition rates and smaller initial leaching losses than the litterbag decomposition model in HPMe-LE-peat (Fig. \ref{fig:out-p-hpm-mm27-2-mm29-1-mm30-1-p2}). In contrast to this, initial leaching losses and smaller decomposition rates estimated by HPMe-LE-peat-l0 were more similar to estimates of the separate litterbag decomposition model from Teickner et al. (\protect\hyperlink{ref-Teickner.2024}{2024}) (Fig. \ref{fig:out-sdm-all-models-p2}). This indicates again that a better test of the HPM is possible when \(l_0\) can be estimated more accurately.

In line with this, the maximum possible decomposition rates for the species differ between the HPM modifications. HPMe-LE-peat-l0 estimates a larger average maximum possible decomposition rate, particularly for \emph{S. angustifolium}, than the other models (Fig. \ref{fig:out-p-hpm-mm27-2-mm29-1-mm30-1-p2} and supporting Fig. \ref{fig:sup-hpmd-plot-6}).

\hypertarget{estimates-for-c_1-w_opt-f_min-c_2-are-similar-between-modifications-of-the-hpm.}{%
\paragraph*{\texorpdfstring{Estimates for \(c_1\), \(W_{opt}\), \(f_{min}\), \(c_2\) are similar between modifications of the HPM.}{Estimates for c\_1, W\_\{opt\}, f\_\{min\}, c\_2 are similar between modifications of the HPM.}}\label{estimates-for-c_1-w_opt-f_min-c_2-are-similar-between-modifications-of-the-hpm.}}
\addcontentsline{toc}{paragraph}{Estimates for \(c_1\), \(W_{opt}\), \(f_{min}\), \(c_2\) are similar between modifications of the HPM.}

In contrast to estimates for \(k_0\), \(l_0\), and \(k_{i,0}\), the other HPM parameters had similar estimates for HPMe-LE-peat and HPMe-LE-peat-l0 and as a consequence relative differences of decomposition rates along the water table depth gradient are very similar between all models (Fig. \ref{fig:out-p-hpm-mm27-2-mm29-1-mm30-1-p2}). Estimates for \(f_{min}\) did not differ much to the prior value and the power-scaling sensitivity analysis indicates a weak influence of the data (supporting information \ref{sup-12}) and therefore that available litterbag data provide only little information about minimum decomposition rates under anoxic conditions.

\hypertarget{litterbag-data-do-not-indicate-a-clear-relation-of-l_0-to-the-degree-of-saturation-in-hpme-le-peat-l0.}{%
\paragraph*{\texorpdfstring{Litterbag data do not indicate a clear relation of \(l_0\) to the degree of saturation in HPMe-LE-peat-l0.}{Litterbag data do not indicate a clear relation of l\_0 to the degree of saturation in HPMe-LE-peat-l0.}}\label{litterbag-data-do-not-indicate-a-clear-relation-of-l_0-to-the-degree-of-saturation-in-hpme-le-peat-l0.}}
\addcontentsline{toc}{paragraph}{Litterbag data do not indicate a clear relation of \(l_0\) to the degree of saturation in HPMe-LE-peat-l0.}

HPMe-LE-peat-l0 suggests that both positive and negative relations of \(l_0\) are compatible with available litterbag data (95\% confidence intervals for the slope (logit scale): (-0.28, 0.15), supporting Fig. \ref{fig:sup-hpmd-plot-7}). In contrast to HPMf-LE-peat and HPMe-LE-peat, it estimates on average smaller initial leaching losses, more similar to estimates of the litterbag decomposition model not combined with the HPM (\protect\hyperlink{ref-Teickner.2024}{Teickner et al., 2024}) (Fig. \ref{fig:out-sdm-all-models-p2}).



\begin{figure}[H]

{\centering \includegraphics[width=1\linewidth]{figures/hpmd_plot_2} 

}

\caption{\(k_0\) estimated from the litterbag data (Predicted with HPM = No) and predicted by different versions of the HPM decomposition module (Predicted with HPM = Yes) (HPMf, HPMf-LE-peat, HPMe-LE-peat, or HPMe-LE-peat-l0) versus reported (HPMf) or estimated (HPMf-LE-peat, HPMe-LE-peat, or HPMe-LE-peat-l0) average water table depths below the litterbags. Points represent average estimates and error bars 95\% posterior intervals. Lines are predictions of linear models fitted to the average estimates. \emph{Sphagnum} spec. are samples that have been identified only to the genus level. Only data for species with at least three replicates are shown.}\label{fig:out-p-hpm-mm27-2-mm29-1-mm30-1-p2}
\end{figure}



\begin{figure}[H]

{\centering \includegraphics[width=1\linewidth]{figures/hpmd_plot_3} 

}

\caption{Plot of \(l_0\) (a) or \(k_0\) (b) as predicted by litterbag decomposition models combined with different modifications of the HPM (see Tab. \ref{tab:m-litterbag-synthesis-models}) versus estimates of the litterbag decomposition model from Teickner et al. (\protect\hyperlink{ref-Teickner.2024}{2024}) for the same data. Litterbag experiments for which the HPM decomposition module could make predictions (water table depths reported in the studies) and to which the HPM parameters were fitted are shown as white points. Estimates for experiments from Hagemann and Moroni (\protect\hyperlink{ref-Hagemann.2015}{2015}) are not shown because these always had large estimates for \(k_0\), were not directly tested against the HPM, and would make it difficult to represent the pattern for samples for which the HPM predicted \(k_0\).}\label{fig:out-sdm-all-models-p2}
\end{figure}

\hypertarget{comparison-between-standard-hpm-parameter-values-defined-in-frolking.2010-and-estimated-from-litterbag-data}{%
\subsection{\texorpdfstring{Comparison between standard HPM parameter values defined in Frolking et al. (\protect\hyperlink{ref-Frolking.2010}{2010}) and estimated from litterbag data}{Comparison between standard HPM parameter values defined in Frolking et al. (2010) and estimated from litterbag data}}\label{comparison-between-standard-hpm-parameter-values-defined-in-frolking.2010-and-estimated-from-litterbag-data}}

Figure \ref{fig:out-mm30-hpm-test-p2} shows marginal posterior densities of the maximum possible decomposition rate for each species and the four other HPM parameters for HPMe-LE-peat, with standard parameter values as defined in Frolking et al. (\protect\hyperlink{ref-Frolking.2010}{2010}) indicated by vertical lines. For both HPMe-LE-peat and HPMe-LE-peat-l0, the range of parameter estimates contains the standard values, but there are large posterior probabilities that \(c_2\) (\(P_\text{HPMe-LE-peat}(c_2>0.3~\mathrm{m}) = 1\) and \(P_\text{HPMe-LE-peat-l0}(c_2>0.3~\mathrm{m}) = 1\)) and \(W_{opt}\) (\(P_\text{HPMe-LE-peat}(W_{opt}>0.45~\mathrm{L}_\mathrm{water}~\mathrm{L}_\mathrm{pores}^{-1}) = 1\) and \(P_\text{HPMe-LE-peat-l0}(W_{opt}>0.45~\mathrm{L}_\mathrm{water}~\mathrm{L}_\mathrm{pores}^{-1}) = 0.98\)) have larger values than the standard parameter values, indicating a discrepancy between the HPM and available litterbag data (Fig. \ref{fig:out-mm30-hpm-test-p2} and supporting Fig. \ref{fig:sup-hpmd-plot-4-4}).

Both models also estimate a large posterior probability (\(>95\)\%) that \emph{S. russowii} and \emph{S. rubellum} have a larger, and that \emph{S. cuspidatum} has a smaller maximum possible decomposition rate (\(k_{0,i}\)) than the standard values for the respective PFT (Fig. \ref{fig:out-mm30-hpm-test-p2} and supporting Fig. \ref{fig:sup-hpmd-plot-4-4}). However, because of the larger variability of \(k_{0,i}\) in the cross-validation (compare with the previous subsection), this discrepancy is probably more uncertain when new data would become available.



\begin{figure}[H]

{\centering \includegraphics[width=1\linewidth]{figures/hpmd_plot_4_3} 

}

\caption{Marginal posterior distributions of HPM decomposition model parameters as estimated by HPMe-LE-peat. (a) \(k_{0,i}\) (maximum possible decomposition rate for species \(i\)) estimated for each species. Species were assigned to HPM microhabitats as described in section \ref{sdm-003-methods-4}. (b) other HPM parameters. Vertical black lines are the standard parameter values from Frolking et al. (\protect\hyperlink{ref-Frolking.2010}{2010}). \emph{Sphagnum} spec. are samples that have been identified only to the genus level.}\label{fig:out-mm30-hpm-test-p2}
\end{figure}

\hypertarget{discussion}{%
\section{Discussion}\label{discussion}}

Our aims were to test whether the HPM can predict litterbag decomposition rates for different \emph{Sphagnum} species along the gradient from oxic to anoxic conditions, and to test whether HPM parameters estimated from litterbag data are compatible with the HPM standard values.

Our analysis suggests that the HPM decomposition module with standard parameter values can fit available litterbag data, but only because the uncertainties in litterbag data are large enough to support a range of parameter values. The price to be paid for this is to assume larger initial leaching losses and smaller decomposition rates than estimated with the litterbag decomposition model alone (\protect\hyperlink{ref-Teickner.2024}{Teickner et al., 2024}) (Fig. \ref{fig:out-sdm-all-models-p2}). Comparable or better fits could be achieved by estimating HPM parameters from litterbag data (HPMe-LE-peat and HPMe-LE-peat-l0) and similar decomposition rate and initial leaching losses as estimated from litterbag data alone were predicted by a model that assumes smaller initial leaching losses (HPMe-LE-peat-l0). Decomposition rates can be estimated more accurately from litterbag experiments when initial leaching losses are estimated more accurately (\protect\hyperlink{ref-Teickner.2024}{Teickner et al., 2024}). Therefore, an important result of our study is that stronger tests of the HPM decomposition module and other peatland models require litterbag experiments that allow to estimate initial leaching losses more accurately than is possible with available experiments.

Despite these uncertainties, our analysis of the HPM suggests that better fits to available litterbag data are possible only if several HPM parameter values are adjusted, namely the maximum possible decomposition rates for HPM PFT or \emph{Sphagnum} species (\(k_{0,i}\)), the optimum degree of saturation for decomposition (\(W_{opt}\)), and the anoxia scale length (\(c_2\)).

In the following sections, we discuss these discrepancies. In particular, we show that they imply a less steep gradient of decomposition rates from oxic to anoxic conditions than assumed by the standard HPM. We discuss how reliable this pattern is, considering that the data are from heterogeneous studies, what processes may cause the less steep gradient, and how important the suggested differences in parameter values are for the predicted C accumulation.

In supporting information \ref{sup-7}, we present R code to predict decomposition rates and remaining masses in litterbag experiments with different species and water table levels with HPMe-LE-peat-l0 (\protect\hyperlink{ref-Teickner.2024b}{Teickner and Knorr, 2024a}) which can be useful to improve the HPM, test HPMe-LE-peat-l0 against novel litterbag data, or plan litterbag studies.

\hypertarget{out-discussion-1}{%
\subsection{HPM parameters for which estimates differ from their standard values}\label{out-discussion-1}}

Three HPM parameters had estimates contrasting to their standard values:

\begin{enumerate}
\def\labelenumi{\arabic{enumi}.}
\item
  There is a large posterior probability that \(c_2\) is larger than the standard value of 0.3 m. \(c_2\) is the anoxia scale length of decomposition and is assumed to represent limitation of anaerobic decomposition below the water table depth as consequences of the accumulation of decomposition end products and depletion of electron acceptors (\protect\hyperlink{ref-Frolking.2010}{Frolking et al., 2010}). A larger value implies larger anaerobic decomposition rates at the same depth below the water table.
\item
  There is a large posterior probability that \(W_{opt}\) is larger than the standard value of 0.45 L\(_\text{water}\) L\(_\text{pores}^{-1}\). \(W_{opt}\) is the degree of saturation at which the decomposition rate is largest. Larger values mean that the largest decomposition rates are reached at larger degrees of saturation.
\item
  For some species, there is a large posterior probability that \(k_{0,i}\) is smaller (\emph{S. cuspidatum}) or larger (\emph{S. russowii} and \emph{S. rubellum}) than the standard value for the HPM microhabitat class we assigned them to. In addition, \(k_{0,i}\) was not consistent between HPMe-LE-peat and HPMe-LE-peat-l0 and also differed between models estimated when removing portions of the data during the cross-validation (supporting Fig. \ref{fig:sup-hpmd-plot-5}). \(k_{0,i}\) defines how decomposition rates differ between \emph{Sphagnum} species and is therefore an important control of C accumulation if there are vegetation changes.
\end{enumerate}

Of these parameters, \(c_2\), and \(k_{0,i}\) are of particular relevance for C accumulation in the HPM, as indicated by previous sensitivity analyses (\protect\hyperlink{ref-Quillet.2013}{Quillet et al., 2013a}, b). Explaining the discrepancies and finding ways to test them more accurately than possible with available litterbag data should therefore improve our understanding of peat C accumulation.

\hypertarget{out-discussion-2}{%
\subsection{\texorpdfstring{Estimated \(c_2\) and \(W_{opt}\) suggest larger anaerobic decomposition rates relative to aerobic decomposition rates than the standard HPM}{Estimated c\_2 and W\_\{opt\} suggest larger anaerobic decomposition rates relative to aerobic decomposition rates than the standard HPM}}\label{out-discussion-2}}

The discrepancies in \(c_2\) and \(W_{opt}\) together imply smaller aerobic and larger anaerobic decomposition rates and therefore a less steep decrease of decomposition rates from oxic to anoxic conditions (Fig. \ref{fig:out-p-hpm-mm27-2-mm29-1-mm30-1-p2}). These relative rates are scaled by \(k_{0,i}\) to absolute decomposition rates. With \(k_{0,i}\) estimated from litterbag experiments, the discrepancies in \(c_2\) and \(W_{opt}\) also indicate larger anaerobic decomposition rates than assumed by the HPM for several species (Fig. \ref{fig:out-p-hpmd-plot-9}). Therefore, the discrepancies to the HPM indicate a less steep decrease of decomposition rates from oxic to anoxic conditions and, at least for some species, larger absolute anaerobic decomposition rates.

To illustrate that the estimated \(c_2\) and \(W_{opt}\) imply smaller aerobic and larger anaerobic decomposition rates, we simulated decomposition of \emph{S. fuscum} incubated at different depths in a peatland with water table depth of 40 cm below the surface, a porosity of 0.7 L\(_\text{pores}\) L\(_\text{sample}^{-1}\), and a minimum water content at the surface of 0.05 g\(_\text{water}\) g\(_\text{sample}^{-1}\). We predicted average \(k_0\) of \emph{S. fuscum} with HPMe-LE-peat-l0 (\(k_{0,\text{modified}}(\text{HPMe-LE-peat-l0})\)) and with HPMe-LE-peat-l0 setting either \(c_1\), \(W_{opt}\), \(f_{min}\), or \(c_2\) to the standard value (\(k_{0,\text{standard}}(\text{HPMe-LE-peat-l0})\)) and computed their differences. This gives the difference in decomposition rates of HPMe-LE-peat-l0 if we would set individual HPM parameters to their standard values. We plotted this difference versus the depth of the water table below the litter (litter at the surface has a value of +40 cm, litter at the water table level of 0 cm, and litter below the water table level has negative values), as shown in Fig. \ref{fig:out-sdm-parameters-standard-vs-estimated-p1}.

With the standard \(W_{opt}\) value, HPMe-LE-peat-l0 predicts larger decomposition rates above and smaller decomposition rates below the water table than when using the parameter values estimated from litterbag data. Similarly, setting \(c_2\) to its standard value also results in smaller decomposition rates below the water table level. The other parameters do not have a qualitative influence (Fig. \ref{fig:out-sdm-parameters-standard-vs-estimated-p1}). Thus, the discrepancies in \(W_{opt}\) and \(c_2\) are the main drivers of the less steep decrease of decomposition rates from oxic to anoxic conditions in HPMe-LE-peat-l0 compared to the standard HPM.



\begin{figure}[H]

{\centering \includegraphics[width=0.9\linewidth]{figures/hpmd_plot_9} 

}

\caption{\(k_0\) predicted by HPM modifications (either HPMf-LE-peat, HPMe-LE-peat, or HPMe-LE-peat-l0) minus \(k_0\) predicted by the HPM with standard parameter values (HPMf) versus estimated average water table depths below the litterbags. Points represent average estimates and error bars 95\% posterior intervals. \emph{Sphagnum} spec. are samples which that been identified only to the genus level. Only data for species with at least three replicates are shown.}\label{fig:out-p-hpmd-plot-9}
\end{figure}



\begin{figure}[H]

{\centering \includegraphics[width=1\linewidth]{figures/hpmd_simulation_1_plot_1_1} 

}

\caption{Difference between decomposition rates for \emph{S. fuscum} predicted with parameter values estmated by HPMe-LE-peat-l0 (\(k_{0,\text{modified}}(\text{HPMe-LE-peat-l0})\)), and when setting the HPM parameter in the panel title to its standard value (\(k_{0,\text{standard}}(\text{HPMe-LE-peat-l0})\)). Panels show results when different parameters are set to their standard values. Positive \(k_{0,\text{modified}}(\text{HPMe-LE-peat-l0}) - k_{0,\text{standard}}(\text{HPMe-LE-peat-l0})\) means that decomposition rates are larger when using the estimated parameter value compared to using the standard parameter value.}\label{fig:out-sdm-parameters-standard-vs-estimated-p1}
\end{figure}

\hypertarget{out-discussion-3}{%
\subsection{Reliability of the identified discrepancies}\label{out-discussion-3}}

Before analyzing potential causes of the discrepancies found for \(c_2\) and \(W_{opt}\) we first ask if combining different litterbag experiments is reliable evidence for the less steep gradient in decomposition rates from oxic to anoxic conditions.

If we take a look at the misfits of the standard HPM (HPMf) shown in Fig. \ref{fig:out-p-hpm-mm27-2-mm29-1-mm30-1-p2}, many, but not all underestimations of aerobic decomposition rates could have been caused by other factors: For example for \emph{S. balticum} the difference may have been caused by differences in the two litterbag experiments from which we collected the data because the replicate with positive water table depth is from Straková et al. (\protect\hyperlink{ref-Strakova.2010}{2010}), whereas the two others are from Mäkilä et al. (\protect\hyperlink{ref-Makila.2018}{2018}).

The less pronounced gradient in measured decomposition rates above the water table depth is, however, also visible for \emph{S. fuscum} replicates within the same study (\protect\hyperlink{ref-Johnson.1991}{Johnson and Damman, 1991}; \protect\hyperlink{ref-Golovatskaya.2017}{Golovatskaya and Nikonova, 2017}; \protect\hyperlink{ref-Makila.2018}{Mäkilä et al., 2018}) and in addition similar across these (independent) studies (supporting information \ref{sup-8}), indicating that this pattern cannot be explained in all cases by differences between studies. In addition, during the cross-validation, we removed data from individual studies from the model and the remaining subsets still resulted in similar estimates for \(c_2\) and \(W_{opt}\) (supporting Fig. \ref{fig:sup-hpmd-plot-5}). Finally, numerous previous studies suggest that water table depth is an important control of decomposition rates (e.g., Blodau et al. (\protect\hyperlink{ref-Blodau.2004}{2004})) and one may therefore expect that also between different studies decomposition rate differences should be controlled to a large degree by differences in water table depths. Thus, even with the heterogeneous litterbag data which is currently available, a less steep gradient of decomposition rates from oxic to anoxic conditions appears to be replicable between studies and species. Controlled litterbag experiments should test this.

The \(W_{opt}\) suggested by HPMe-LE-peat-l0 is also near the average optimum of heterotrophic respiration estimated across a range of mineral soils (\protect\hyperlink{ref-Moyano.2013}{Moyano et al., 2013}). The estimate is also in line with a study where the largest decomposition rates of the same litter type were observed at or just above the average water table level in hummocks (\protect\hyperlink{ref-Belyea.1996}{Belyea, 1996}), and with maximum CO\(_2\) production rates around 13 cm above the water table level in a mesocosm study (\protect\hyperlink{ref-Blodau.2004}{Blodau et al., 2004}). According to the ModGberg model the degree of saturation at this depth is near the \(W_{opt}\) suggested by HPMe-LE-peat and HPMe-LE-peat-l0. For example, for our simulation analysis used to produce Fig. \ref{fig:out-sdm-parameters-standard-vs-estimated-p1}, the average \(W_{opt}\) estimated by model HPMe-LE-peat-l0 (0.57 L\(_\text{water}\) L\(_\text{pores}^{-1}\)) is reached around 16 cm above the water table level, as shown in Fig. \ref{fig:out-sdm-parameters-standard-vs-estimated-p2}. At shallower depths, the degree of saturation decreases below \(W_{opt}\) which would decrease decomposition rates as observed in Belyea (\protect\hyperlink{ref-Belyea.1996}{1996}). In contrast, according to the the ModGberg model, a degree of saturation corresponding to the standard \(W_{opt}\) value (0.45 L\(_\text{water}\) L\(_\text{pores}^{-1}\)) is reached at shallower depths and in the same simulation with this standard \(W_{opt}\) value, no pronounced sub-surface peak in decomposition rates is observed (supporting Fig. \ref{fig:sup-hpmd-simulation-1-plot-5}). In hollows, the optimum degree of saturation suggested by HPMe-LE-peat-l0 is reached near the surface for either \(W_{opt}\) value (supporting Fig. \ref{fig:sup-hpmd-simulation-1-plot-5}). Thus, a larger \(W_{opt}\) would be compatible with results from several previous studies.

Larger and smaller \(c_2\) than the standard value have been estimated for several permafrost peatland cores with a modified version of the HPM with monthly time step (\protect\hyperlink{ref-Treat.2021}{Treat et al., 2021}, \protect\hyperlink{ref-Treat.2022}{2022}). Smaller values have been estimated for tropical peatlands (\protect\hyperlink{ref-Kurnianto.2015}{Kurnianto et al., 2015}). To our knowledge, no litterbag experiment directly estimated \(c_2\). A difficulty is that available litterbag experiments cover only a comparatively small depth below the water table level (at most around 30 cm, Fig. \ref{fig:out-p-hpm-mm27-2-mm29-1-mm30-1-p2}) and therefore gradients in anaerobic decomposition rates across larger depths below the water table currently cannot be estimated.



\begin{figure}[H]

{\centering \includegraphics[width=0.6\linewidth]{figures/hpmd_simulation_1_plot_1_2} 

}

\caption{Decomposition rates predicted with HPMe-LE-peat-l0 (\(k_{0,\text{modified}}(\text{HPMe-LE-peat-l0})\)) for \emph{S. fuscum} (hummocks), using either the standard value for \(W_{opt}\) or the \(W_{opt}\) value estimated by HPMe-LE-peat-l0 versus depth of the litter below the peat surface. The horizontal line is the average water table depth.}\label{fig:out-sdm-parameters-standard-vs-estimated-p2}
\end{figure}

\hypertarget{out-discussion-4}{%
\subsection{\texorpdfstring{Water table fluctuations may explain the discrepancies in \(c_2\) and \(W_{opt}\) and larger anaerobic and smaller aerobic decomposition rates.}{Water table fluctuations may explain the discrepancies in c\_2 and W\_\{opt\} and larger anaerobic and smaller aerobic decomposition rates.}}\label{out-discussion-4}}

The HPM predicts decomposition rates based on average annual water table depths (\protect\hyperlink{ref-Frolking.2010}{Frolking et al., 2010}). Our evaluation of the HPM also assumed an average water table depth during the litterbag experiments and the HPM translated this into a clear pronounced transition between anaerobic and aerobic decomposition rates (Fig. \ref{fig:out-p-hpm-mm27-2-mm29-1-mm30-1-p2}). In reality, water table depths fluctuate and this causes transient and nonlinear changes in decomposition rates due to variations in the availability of oxygen and other electron acceptors, flushing of end products of anaerobic decomposition, and possibly other factors (\protect\hyperlink{ref-Siegel.1995}{Siegel et al., 1995}; \protect\hyperlink{ref-Blodau.2003}{Blodau and Moore, 2003}; \protect\hyperlink{ref-Blodau.2004}{Blodau et al., 2004}; \protect\hyperlink{ref-Beer.2007}{Beer and Blodau, 2007}; \protect\hyperlink{ref-Knorr.2009}{Knorr and Blodau, 2009}; \protect\hyperlink{ref-Walpen.2018}{Walpen et al., 2018}; \protect\hyperlink{ref-Campeau.2021}{Campeau et al., 2021}; \protect\hyperlink{ref-Kim.2021}{Kim et al., 2021}; \protect\hyperlink{ref-Treat.2022}{Treat et al., 2022}; \protect\hyperlink{ref-Obradovic.2023}{Obradović et al., 2023}). A possible explanation why the gradient in decomposition rate from oxic to anoxic decomposition is less steep, on average across litterbag experiments, than suggested by the standard HPM could therefore be that an averaging effect of fluctuating water table levels on both aerobic and anaerobic decomposition rates is neglected by the HPM.

An additional factor may be that litterbags are buried over a depth range, but we assumed a single fixed depth. If the buried litterbags cover some depth range, this would spatially average decomposition rates, with similar effects as the temporal average caused by water table fluctuations.

According to our results, \(c_2\) would have to be re-interpreted as transition parameter that accounts both for limitation of anaerobic decomposition under anoxic conditions and for the effects of periodically oxic conditions. Similarly, \(W_{opt}\) would have to be re-interpreted as the optimum average degree of saturation for decomposition under water table level variations and its value would be necessarily different from the optimum degree of saturation for depolymerization under static degree of saturation.

Adjusting the HPM parameters as implied by our modified models may be an easy way to account for the effect of sub-annual variation in water table levels on decomposition rates, if the discrepancies are caused by fluctuating water tables and if the model is representative for different effects variations in water table level may have on decomposition rates (e.g.~short-term fluctuations compared to seasonal water table variations compared to prolonged droughts). What we have not considered due to limited data is that \(c_2\) can be expected to depend on long-term changes in groundwater flow (e.g., Siegel et al. (\protect\hyperlink{ref-Siegel.1995}{1995})) or site-specific differences in hydrology and other factors (e.g., Treat et al. (\protect\hyperlink{ref-Treat.2021}{2021}), Treat et al. (\protect\hyperlink{ref-Treat.2022}{2022})). Therefore, \(c_2\) may differ between litterbag studies and our data only indicate that \(c_2\) is larger on average, whereas more research is necessary to estimate and understand site-specific controls of \(c_2\) and how a change in hydrology controls \(c_2\). Similarly, \(W_{opt}\) may differ between sites and over time. It would be interesting to know whether litterbag experiments can quantify these controls and whether \(c_2\) estimated from litterbag experiments is generally larger in peatlands with larger water table fluctuations.

\hypertarget{out-discussion-5}{%
\subsection{\texorpdfstring{Implications of the discrepancies in \(W_{opt}\) and \(c_2\) for long-term C accumulation}{Implications of the discrepancies in W\_\{opt\} and c\_2 for long-term C accumulation}}\label{out-discussion-5}}

A larger \(c_2\) implies larger anaerobic decomposition and may thus indicate that the HPM underestimates anaerobic decomposition rates. Previous sensitivity analyses identified \(c_2\) as influential for C accumulation in the HPM (\protect\hyperlink{ref-Quillet.2013}{Quillet et al., 2013a}, b).

If \(c_2\) is varied within the range from the standard value (0.3 m) to the average posterior estimate from HPMe-LE-peat-l0 (0.64 m), this would cause differences in predicted C accumulation of a maximum of ca. 20\% in the sensitivity experiment of Quillet et al. (\protect\hyperlink{ref-Quillet.2013}{2013a}) (depending on precipitation, Fig. 1 c in Quillet et al. (\protect\hyperlink{ref-Quillet.2013}{2013a})). If values are changed across the complete posterior range compatible with litterbag data and if other HPM parameters would also be varied, the effect would be even larger (Fig. 2 c in Quillet et al. (\protect\hyperlink{ref-Quillet.2013}{2013a})).

Due to parameter interactions and feedbacks, an increase in anaerobic decomposition rates can result in smaller or larger C accumulation of the HPM, depending on environmental conditions (\protect\hyperlink{ref-Quillet.2013}{Quillet et al., 2013a}). Small anaerobic decomposition may cause too rapid C accumulation resulting in a low water table level, a thick aerobic zone, and thus smaller overall C accumulation after a longer time. Larger anaerobic decomposition may result in higher water table levels and this can increase C accumulation in the long-term. Too large anaerobic decomposition decreases C accumulation (\protect\hyperlink{ref-Quillet.2013}{Quillet et al., 2013a}).

A larger \(W_{opt}\) implies that the largest aerobic decomposition rates are reached under more saturated conditions. \(W_{opt}\) has not been identified as influential in a sensitivity analysis of the HPM (\protect\hyperlink{ref-Quillet.2013}{Quillet et al., 2013a}), but as shown above, it contributes to the less steep decrease of decomposition rates from oxic to anoxic conditions. Importantly, since the HPM does not have a seasonally resolved water table depth, the two sensitivity analyses did not consider how seasonal variations of the water table depth may control long-term C accumulation, and consequently the re-interpreted \(W_{opt}\) may be more important to long-term C accumulation than previously assumed. In addition, HPMe-LE-peat-l0 suggests an average \(W_{opt}\) value of 0.57 L\(_\text{water}\) \(_\text{pores}^{-1}\), which is outside the range of values tested in Quillet et al. (\protect\hyperlink{ref-Quillet.2013}{2013a}) (0.3 to 0.5 L\(_\text{water}\) \(_\text{pores}^{-1}\)). This implies that the sensitivity of long-term C accumulation to \(W_{opt}\) has been evaluated over a too small range.

A further aspect that needs to be considered is that HPMe-LE-peat and HPMe-LE-peat-l0 estimate parameter distributions based on available data, whereas existing studies defined fixed parameter values or ranges of parameter values based on expert knowledge. Based on Quillet et al. (\protect\hyperlink{ref-Quillet.2013}{2013a}), the uncertainties would cause non-negligible differences in predicted long-term C accumulation. For example, values within the uncertainty range of \(c_2\) estimated by HPMe-LE-peat-l0 ((0.4, 0.97), 95\% confidence interval), would imply differences up to 100 kg m\(^{-2}\) of accumulated C over 5000 years in some simulations (Fig. 1 (c) in Quillet et al. (\protect\hyperlink{ref-Quillet.2013}{2013a}), with a maximum total accumulation of ca. 430 kg\(_\text{C}\) m\(^{-2}\)). Simulations of remaining masses for different \emph{Sphagnum} species under different conditions also indicate large uncertainties in predicted remaining masses (supporting info \ref{sup-11}). This implies that more work is required to estimate parameters accurately enough to detect even relative large differences among peatland models and between model predictions and peat cores.

Summarized, based on existing sensitivity analyses of the HPM the parameter discrepancies suggested by HPMe-LE-peat and HPMe-LE-peat-l0 can translate into non-negligible differences in long-term C accumulation rates. They also imply gaps in previous sensitivity analyses of the HPM, namely that \(W_{opt}\) has been analyzed over a too restricted value range and may play a more important role if water table fluctuations are taken into account.

\hypertarget{out-discussion-7}{%
\subsection{\texorpdfstring{Large errors in \(k_{0,i}\) estimates for individual species cause large errors in decomposition rates predicted by the HPM}{Large errors in k\_\{0,i\} estimates for individual species cause large errors in decomposition rates predicted by the HPM}}\label{out-discussion-7}}

We found some discrepancies between the maximum potential decomposition rates (\(k_{0,i}\)) HPMe-LE-peat-l0 estimated for some species and the standard HPM values after assigning species to the three HPM microhabitat PFT (hollow, lawn, hummock \emph{Sphagnum} mosses), however as noted above, these discrepancies were neither consistent between the two modifications of the HPM (HPMe-LE-peat and HPMe-LE-peat-l0) (supporting information \ref{sup-5}), nor when HPMe-LE-peat-l0 was fitted to different subsets of the data during cross-validation (supporting Fig. \ref{fig:sup-hpmd-plot-5}).

Altogether, this indicates that the \(k_{0,i}\) for many of the \emph{Sphagnum} species are difficult to estimate from available litterbag data and more research should address this task. For example, HPMe-LE-peat-l0 could be extended, with suitable data, by modelling how \(k_{0,i}\) is controlled by factors such as temperature or within-species differences in litter chemistry.

We expect that better estimating \(k_{0,i}\) is an important step to improve the predictive accuracy of the HPM because the cross-validation of HPMe-LE-peat-l0 indicated a larger RMSE\(_\text{test}\) than RMSE\(_\text{train}\), with only small variability in estimates of \(c_1\), \(W_{opt}\), \(f_{min}\), and \(c_2\), but much more variability in estimates of \(k_{0,i}\). This indicates that a large part of the difference between RMSE\(_\text{test}\) and RMSE\(_\text{train}\) of HPMe-LE-peat-l0 may be explained by missing information about \(k_{0,i}\).

Moreover, as noted above, \(k_{0,i}\) scales the relative differences in anaerobic versus aerobic decomposition rates to absolute decomposition rates. For example, as shown in Fig. \ref{fig:out-p-hpmd-plot-9}, HPMe-LE-peat-l0 indicates that the standard HPM underestimates aerobic and anaerobic decomposition rates for \emph{S. angustifolium} and \emph{S. magellanicum aggr.} litterbag data, whereas for \emph{S. fuscum} only anaerobic decomposition rates are underestimated.

Values of \(k_{0,i}\) can be estimated more accurately if decomposition rates in the litterbag experiments can be estimated more accurately and there is again a direct link to initial leaching losses. Our analysis of differences in behavior of HPMf, HPMf-LE-peat, HPMe-LE-peat, and HPMe-LE-peat-l0 suggests that HPMf-LE-peat and HPMe-LE-peat produced smaller decomposition rate estimates and larger initial leaching loss estimates to make the litterbag data compatible with the (smaller predictions of the) HPM, whereas HPMe-LE-peat-l0 did not (Fig. \ref{fig:out-sdm-all-models-p2}) and consequently had larger estimates for \(k_{0,i}\) than the other two models (Fig. \ref{fig:out-p-hpm-mm27-2-mm29-1-mm30-1-p2}). Thus, more accurate estimation of initial leaching losses --- which vary a lot for the same species between different studies (\protect\hyperlink{ref-Teickner.2024}{Teickner et al., 2024}) --- should make decomposition rate estimates more accurate, and this should improve accuracy of \(k_{0,i}\) in the HPM, according to our analyses.

\hypertarget{conclusions}{%
\section{Conclusions}\label{conclusions}}

Estimating HPM parameters from \emph{Sphagnum} litterbag experiments suggests larger anaerobic decomposition rates and a less steep gradient of decomposition rates from oxic to anoxic conditions than implied by the HPM with standard parameter values. With these modifications, the HPM fits available litterbag data within the range of uncertainties. However, due to large uncertainties in available litterbag data, particularly about how much of the mass is lost due to initial leaching and how much due to decomposition, the HPM with standard values can achieve comparable fits if mass loss in litterbag experiments is explained by larger initial leaching and slower subsequent decomposition. Therefore, stronger tests of the HPM require more accurate estimates for initial leaching losses and decomposition rates.

The larger anaerobic decomposition rates and less steep gradient of decomposition rates from oxic to anoxic conditions compared to the HPM with standard parameter values are a consequence of larger estimates for the anoxia scale length (\(c_2\)) and the optimium degree of saturation for decomposition (\(W_{opt}\)). This discrepancy may be caused by neglecting an increase of decomposition rates below the annual average water table depth due to water table fluctuations, differences in groundwater flow, or spatial averaging in litterbag experiments. Our estimates may be an easy way to account for such effects in the HPM if effects of these fluctuations on decomposition rates can be averaged over time as implied by the suggested parameter estimates.

Less limitation of anaerobic decomposition rates than suggested by the HPM would imply differences in predicted C accumulation rates of up to 100 kg\(_\text{C}\) m\(^{-2}\) over 5000 years (with a maximum total C accumulation of ca. 430 kg\(_\text{C}\) m\(^{-2}\)), according to previous sensitivity analyses. Future litterbag experiments should improve the accuracy of initial leaching loss and decomposition rate estimates and then test whether the identified parameter discrepancies are reproducible and whether they can be described by known, but not yet fully quantified, controls of decomposition rates in dependency of water table fluctuations.

\hypertarget{acknowledgements}{%
\section*{Acknowledgements}\label{acknowledgements}}
\addcontentsline{toc}{section}{Acknowledgements}

This study was funded by the Deutsche Forschungsgemeinschaft (DFG, German Research Foundation) grant no. KN 929/23-1 to Klaus-Holger Knorr and grant no. PE 1632/18-1 to Edzer Pebesma.

\hypertarget{author-contributions}{%
\section*{Author contributions}\label{author-contributions}}
\addcontentsline{toc}{section}{Author contributions}

HT: Conceptualization, methodology, software, validation, formal analysis, investigation, data curation, writing, visualization, project administration. EP: supervision, funding acquisition, writing - review \& editing. KHK: supervision, funding acquisition, writing - review \& editing.

\hypertarget{data-and-code-availability}{%
\section*{Data and code availability}\label{data-and-code-availability}}
\addcontentsline{toc}{section}{Data and code availability}

Data and code to reproduce this manuscript are available from \url{https://github.com/henningte/eb1125}.

\hypertarget{references}{%
\section*{References}\label{references}}
\addcontentsline{toc}{section}{References}

\hypertarget{refs}{}
\begin{CSLReferences}{0}{0}
\leavevmode\vadjust pre{\hypertarget{ref-Asada.2005b}{}}%
Asada, T. and Warner, B. G.: Surface {Peat Mass} and {Carbon Balance} in a {Hypermaritime Peatland}, Soil Science Society of America Journal, 69, 549--562, \url{https://doi.org/10.2136/sssaj2005.0549}, 2005.

\leavevmode\vadjust pre{\hypertarget{ref-Bartsch.1985}{}}%
Bartsch, I. and Moore, T. R.: A preliminary investigation of primary production and decomposition in four peatlands near {Schefferville}, {Qu{é}bec}, Canadian Journal of Botany, 63, 1241--1248, \url{https://doi.org/10.1139/b85-171}, 1985.

\leavevmode\vadjust pre{\hypertarget{ref-Bauer.2004}{}}%
Bauer, I. E.: Modelling effects of litter quality and environment on peat accumulation over different time-scales: {Peat} accumulation over different time-scales, Journal of Ecology, 92, 661--674, \url{https://doi.org/10.1111/j.0022-0477.2004.00905.x}, 2004.

\leavevmode\vadjust pre{\hypertarget{ref-Beer.2007}{}}%
Beer, J. and Blodau, C.: Transport and thermodynamics constrain belowground carbon turnover in a northern peatland, Geochimica et Cosmochimica Acta, 71, 2989--3002, \url{https://doi.org/10.1016/j.gca.2007.03.010}, 2007.

\leavevmode\vadjust pre{\hypertarget{ref-Belyea.1996}{}}%
Belyea, L. R.: Separating the effects of litter quality and microenvironment on decomposition rates in a patterned peatland, Oikos, 77, 529--539, \url{https://doi.org/10.2307/3545942}, 1996.

\leavevmode\vadjust pre{\hypertarget{ref-Bengtsson.2018}{}}%
Bengtsson, F., Rydin, H., and Hájek, T.: Biochemical determinants of litter quality in 15 species of {\emph{Sphagnum}}, Plant and Soil, 425, 161--176, \url{https://doi.org/10.1007/s11104-018-3579-8}, 2018.

\leavevmode\vadjust pre{\hypertarget{ref-Blodau.2003}{}}%
Blodau, C. and Moore, T. R.: Experimental response of peatland carbon dynamics to a water table fluctuation, Aquatic Sciences - Research Across Boundaries, 65, 47--62, \url{https://doi.org/10.1007/s000270300004}, 2003.

\leavevmode\vadjust pre{\hypertarget{ref-Blodau.2004}{}}%
Blodau, C., Basiliko, N., and Moore, T. R.: Carbon turnover in peatland mesocosms exposed to different water table levels, Biogeochemistry, 67, 331--351, \url{https://doi.org/10.1023/B:BIOG.0000015788.30164.e2}, 2004.

\leavevmode\vadjust pre{\hypertarget{ref-Bona.2018}{}}%
Bona, K. A., Hilger, A., Burgess, M., Wozney, N., and Shaw, C.: A peatland productivity and decomposition parameter database, Ecology, 99, 2406--2406, \url{https://doi.org/10.1002/ecy.2462}, 2018.

\leavevmode\vadjust pre{\hypertarget{ref-Bona.2020}{}}%
Bona, K. A., Shaw, C., Thompson, D. K., Hararuk, O., Webster, K., Zhang, G., Voicu, M., and Kurz, W. A.: The {Canadian} model for peatlands ({CaMP}): {A} peatland carbon model for national greenhouse gas reporting, Ecological Modelling, 431, 109164, \url{https://doi.org/10.1016/j.ecolmodel.2020.109164}, 2020.

\leavevmode\vadjust pre{\hypertarget{ref-Breeuwer.2008}{}}%
Breeuwer, A., Heijmans, M., Robroek, B. J. M., Limpens, J., and Berendse, F.: The effect of increased temperature and nitrogen deposition on decomposition in bogs, Oikos, 117, 1258--1268, \url{https://doi.org/10.1111/j.0030-1299.2008.16518.x}, 2008.

\leavevmode\vadjust pre{\hypertarget{ref-Campeau.2021}{}}%
Campeau, A., Vachon, D., Bishop, K., Nilsson, M. B., and Wallin, M. B.: Autumn destabilization of deep porewater {CO}{\textsubscript{2}} store in a northern peatland driven by turbulent diffusion, Nature Communications, 12, 6857, \url{https://doi.org/10.1038/s41467-021-27059-0}, 2021.

\leavevmode\vadjust pre{\hypertarget{ref-Chaudhary.2018}{}}%
Chaudhary, N., Miller, P. A., and Smith, B.: Biotic and abiotic drivers of peatland growth and microtopography: {A} model demonstration, Ecosystems, 21, 1196--1214, \url{https://doi.org/10.1007/s10021-017-0213-1}, 2018.

\leavevmode\vadjust pre{\hypertarget{ref-Frolking.2001}{}}%
Frolking, S., Roulet, N. T., Moore, T. R., Richard, P. J. H., Lavoie, M., and Muller, S. D.: Modeling northern peatland decomposition and peat accumulation, Ecosystems, 4, 479--498, \url{https://doi.org/10.1007/s10021-001-0105-1}, 2001.

\leavevmode\vadjust pre{\hypertarget{ref-Frolking.2010}{}}%
Frolking, S., Roulet, N. T., Tuittila, E., Bubier, J. L., Quillet, A., Talbot, J., and Richard, P. J. H.: A new model of {Holocene} peatland net primary production, decomposition, water balance, and peat accumulation, Earth System Dynamics, 1, 1--21, \url{https://doi.org/10.5194/esd-1-1-2010}, 2010.

\leavevmode\vadjust pre{\hypertarget{ref-Frolking.2011}{}}%
Frolking, S., Talbot, J., Jones, M. C., Treat, C. C., Kauffman, J. B., Tuittila, E.-S., and Roulet, N.: Peatlands in the {Earth}'s 21st century climate system, Environmental Reviews, 19, 371--396, \url{https://doi.org/10.1139/a11-014}, 2011.

\leavevmode\vadjust pre{\hypertarget{ref-Gelman.2014}{}}%
Gelman, A., Carlin, J., Stern, H., Dunson, D., Vehtari, A., and Rubin, D. B.: Bayesian data analysis, Third edition., CRC Press, Boca Raton, 2014.

\leavevmode\vadjust pre{\hypertarget{ref-Golovatskaya.2017}{}}%
Golovatskaya, E. A. and Nikonova, L. G.: The influence of the bog water level on the transformation of sphagnum mosses in peat soils of oligotrophic bogs, Eurasian Soil Science, 50, 580--588, \url{https://doi.org/10.1134/S1064229317030036}, 2017.

\leavevmode\vadjust pre{\hypertarget{ref-Granberg.1999}{}}%
Granberg, G., Grip, H., Löfvenius, M. O., Sundh, I., Svensson, B. H., and Nilsson, M.: A simple model for simulation of water content, soil frost, and soil temperatures in boreal mixed mires, Water Resources Research, 35, 3771--3782, \url{https://doi.org/10.1029/1999WR900216}, 1999.

\leavevmode\vadjust pre{\hypertarget{ref-Hagemann.2015}{}}%
Hagemann, U. and Moroni, M. T.: Moss and lichen decomposition in old-growth and harvested high-boreal forests estimated using the litterbag and minicontainer methods, Soil Biology and Biochemistry, 87, 10--24, \url{https://doi.org/10.1016/j.soilbio.2015.04.002}, 2015.

\leavevmode\vadjust pre{\hypertarget{ref-Hassel.2018}{}}%
Hassel, K., Kyrkjeeide, M. O., Yousefi, N., Prestø, T., Stenøien, H. K., Shaw, J. A., and Flatberg, K. I.: {\emph{Sphagnum Divinum}} ({\emph{Sp. Nov.}}) And {\emph{S}}{\emph{. Medium}} {Limpr}. And their relationship to {\emph{S}}{\emph{. Magellanicum}} {Brid}., Journal of Bryology, 40, 197--222, \url{https://doi.org/10.1080/03736687.2018.1474424}, 2018.

\leavevmode\vadjust pre{\hypertarget{ref-Heijmans.2008}{}}%
Heijmans, M. M. P. D., Mauquoy, D., Van Geel, B., and Berendse, F.: Long-term effects of climate change on vegetation and carbon dynamics in peat bogs, Journal of Vegetation Science, 19, 307--320, \url{https://doi.org/10.3170/2008-8-18368}, 2008.

\leavevmode\vadjust pre{\hypertarget{ref-Heinemeyer.2010}{}}%
Heinemeyer, A., Croft, S., Garnett, M. H., Gloor, E., Holden, J., Lomas, M. R., and Ineson, P.: The {MILLENNIA} peat cohort model: {Predicting} past, present and future soil carbon budgets and fluxes under changing climates in peatlands, Climate Research, 45, 207--226, \url{https://doi.org/10.3354/cr00928}, 2010.

\leavevmode\vadjust pre{\hypertarget{ref-Hoffman.2014}{}}%
Hoffman, M. D. and Gelman, A.: The no-{U-turn} sampler: {Adaptively} setting path lengths in hamiltonian monte carlo, Journal of Machine Learning Research, 15, 1593--1623, 2014.

\leavevmode\vadjust pre{\hypertarget{ref-Johnson.1991}{}}%
Johnson, L. C. and Damman, A. W. H.: Species-controlled {\emph{Sphagnum}} decay on a south {Swedish} raised bog, Oikos, 61, 234, \url{https://doi.org/10.2307/3545341}, 1991.

\leavevmode\vadjust pre{\hypertarget{ref-Johnson.2015}{}}%
Johnson, M. G., Granath, G., Tahvanainen, T., Pouliot, R., Stenøien, H. K., Rochefort, L., Rydin, H., and Shaw, A. J.: Evolution of niche preference in {\emph{Sphagnum}} peat mosses, Evolution, 69, 90--103, \url{https://doi.org/10.1111/evo.12547}, 2015.

\leavevmode\vadjust pre{\hypertarget{ref-Kallioinen.2024}{}}%
Kallioinen, N., Paananen, T., Bürkner, P.-C., and Vehtari, A.: Detecting and diagnosing prior and likelihood sensitivity with power-scaling, Statistics and Computing, 34, 57, \url{https://doi.org/10.1007/s11222-023-10366-5}, 2024.

\leavevmode\vadjust pre{\hypertarget{ref-Kettridge.2007}{}}%
Kettridge, N. and Baird, A.: In situ measurements of the thermal properties of a northern peatland: {Implications} for peatland temperature models, Journal of Geophysical Research: Earth Surface, 112, \url{https://doi.org/10.1029/2006JF000655}, 2007.

\leavevmode\vadjust pre{\hypertarget{ref-Kim.2021}{}}%
Kim, J., Rochefort, L., Hogue-Hugron, S., Alqulaiti, Z., Dunn, C., Pouliot, R., Jones, T. G., Freeman, C., and Kang, H.: Water table fluctuation in peatlands facilitates fungal proliferation, impedes {\emph{Sphagnum}} growth and accelerates decomposition, Frontiers in Earth Science, 8, 9, 2021.

\leavevmode\vadjust pre{\hypertarget{ref-Knorr.2009}{}}%
Knorr, K.-H. and Blodau, C.: Impact of experimental drought and rewetting on redox transformations and methanogenesis in mesocosms of a northern fen soil, Soil Biology and Biochemistry, 41, 1187--1198, \url{https://doi.org/10.1016/j.soilbio.2009.02.030}, 2009.

\leavevmode\vadjust pre{\hypertarget{ref-Kurnianto.2015}{}}%
Kurnianto, S., Warren, M., Talbot, J., Kauffman, B., Murdiyarso, D., and Frolking, S.: Carbon accumulation of tropical peatlands over millennia: A modeling approach, Global Change Biology, 21, 431--444, \url{https://doi.org/10.1111/gcb.12672}, 2015.

\leavevmode\vadjust pre{\hypertarget{ref-Lind.2022}{}}%
Lind, L., Harbicht, A., Bergman, E., Edwartz, J., and Eckstein, R. L.: Effects of initial leaching for estimates of mass loss and microbial decomposition---{Call} for an increased nuance, Ecology and Evolution, 12, \url{https://doi.org/10.1002/ece3.9118}, 2022.

\leavevmode\vadjust pre{\hypertarget{ref-Liu.2019}{}}%
Liu, H. and Lennartz, B.: Hydraulic properties of peat soils along a bulk density gradient-{A} meta study, Hydrological Processes, 33, 101--114, \url{https://doi.org/10.1002/hyp.13314}, 2019.

\leavevmode\vadjust pre{\hypertarget{ref-Loisel.2017}{}}%
Loisel, J., van Bellen, S., Pelletier, L., Talbot, J., Hugelius, G., Karran, D., Yu, Z., Nichols, J., and Holmquist, J.: Insights and issues with estimating northern peatland carbon stocks and fluxes since the {Last Glacial Maximum}, Earth-Science Reviews, 165, 59--80, \url{https://doi.org/10.1016/j.earscirev.2016.12.001}, 2017.

\leavevmode\vadjust pre{\hypertarget{ref-Makila.2018}{}}%
Mäkilä, M., Säävuori, H., Grundström, A., and Suomi, T.: {\emph{Sphagnum}} decay patterns and bog microtopography in south-eastern {Finland}, Mires and Peat, 1--12, \url{https://doi.org/10.19189/MaP.2017.OMB.283}, 2018.

\leavevmode\vadjust pre{\hypertarget{ref-Morris.2012}{}}%
Morris, P. J., Baird, A. J., and Belyea, L. R.: The {DigiBog} peatland development model 2: Ecohydrological simulations in {2D}, Ecohydrology, 5, 256--268, \url{https://doi.org/10.1002/eco.229}, 2012.

\leavevmode\vadjust pre{\hypertarget{ref-Moyano.2013}{}}%
Moyano, F. E., Manzoni, S., and Chenu, C.: Responses of soil heterotrophic respiration to moisture availability: {An} exploration of processes and models, Soil Biology and Biochemistry, 59, 72--85, \url{https://doi.org/10.1016/j.soilbio.2013.01.002}, 2013.

\leavevmode\vadjust pre{\hypertarget{ref-Nichols.2019}{}}%
Nichols, J. E. and Peteet, D. M.: Rapid expansion of northern peatlands and doubled estimate of carbon storage, Nature Geoscience, 12, 917--921, \url{https://doi.org/10.1038/s41561-019-0454-z}, 2019.

\leavevmode\vadjust pre{\hypertarget{ref-Obradovic.2023}{}}%
Obradović, N., Joshi, P., Arn, S., Aeppli, M., Schroth, M. H., and Sander, M.: Reoxidation of reduced peat organic matter by dissolved oxygen: {Combined} laboratory column-breakthrough experiments and in-field push-pull tests, Journal of Geophysical Research: Biogeosciences, 128, e2023JG007640, \url{https://doi.org/10.1029/2023JG007640}, 2023.

\leavevmode\vadjust pre{\hypertarget{ref-Prevost.1997}{}}%
Prevost, M., Belleau, P., and Plamondon, A. P.: Substrate conditions in a treed peatland: {Responses} to drainage, {É}coscience, 4, 543--554, \url{https://doi.org/10.1080/11956860.1997.11682434}, 1997.

\leavevmode\vadjust pre{\hypertarget{ref-Quillet.2013}{}}%
Quillet, A., Frolking, S., Garneau, M., Talbot, J., and Peng, C.: Assessing the role of parameter interactions in the sensitivity analysis of a model of peatland dynamics, Ecological Modelling, 248, 30--40, \url{https://doi.org/10.1016/j.ecolmodel.2012.08.023}, 2013a.

\leavevmode\vadjust pre{\hypertarget{ref-Quillet.2013a}{}}%
Quillet, A., Garneau, M., and Frolking, S.: Sobol' sensitivity analysis of the {Holocene Peat Model}: {What} drives carbon accumulation in peatlands?, Journal of Geophysical Research: Biogeosciences, 118, 203--214, \url{https://doi.org/10.1029/2012JG002092}, 2013b.

\leavevmode\vadjust pre{\hypertarget{ref-Rydin.2013}{}}%
Rydin, H., Jeglum, J. K., and Bennett, K. D.: The biology of peatlands, 2nd ed., Oxford University Press, Oxford, 2013.

\leavevmode\vadjust pre{\hypertarget{ref-Scheffer.2001}{}}%
Scheffer, R. A., Van Logtestijn, R. S. P., and Verhoeven, J. T. A.: Decomposition of {\emph{Carex}} and {\emph{Sphagnum}} litter in two mesotrophic fens differing in dominant plant species, Oikos, 92, 44--54, \url{https://doi.org/10.1034/j.1600-0706.2001.920106.x}, 2001.

\leavevmode\vadjust pre{\hypertarget{ref-Siegel.1995}{}}%
Siegel, D. I., Reeve, A. S., Glaser, P. H., and Romanowicz, E. A.: Climate-driven flushing of pore water in peatlands, Nature, 374, 531--533, \url{https://doi.org/10.1038/374531a0}, 1995.

\leavevmode\vadjust pre{\hypertarget{ref-StanDevelopmentTeam.2021b}{}}%
Stan Development Team: {RStan}: The {R} interface to {Stan}, 2021a.

\leavevmode\vadjust pre{\hypertarget{ref-StanDevelopmentTeam.2021a}{}}%
Stan Development Team: Stan {Modeling Language Users Guide} and {Reference Manual}, 2021b.

\leavevmode\vadjust pre{\hypertarget{ref-Strakova.2010}{}}%
Straková, P., Anttila, J., Spetz, P., Kitunen, V., Tapanila, T., and Laiho, R.: Litter quality and its response to water level drawdown in boreal peatlands at plant species and community level, Plant and Soil, 335, 501--520, \url{https://doi.org/10.1007/s11104-010-0447-6}, 2010.

\leavevmode\vadjust pre{\hypertarget{ref-Szumigalski.1996}{}}%
Szumigalski, A. R. and Bayley, S. E.: Decomposition along a bog to rich fen gradient in central {Alberta}, {Canada}, Canadian Journal of Botany, 74, 573--581, \url{https://doi.org/10.1139/b96-073}, 1996.

\leavevmode\vadjust pre{\hypertarget{ref-Teickner.2024b}{}}%
Teickner, H. and Knorr, K.-H.: {hpmdpredict}: {Predictions} with model {HPMe-LE-peat-l0} from {Teickner} et al. (2024), 2024a.

\leavevmode\vadjust pre{\hypertarget{ref-Teickner.2024c}{}}%
Teickner, H. and Knorr, K.-H.: Peatland {Decomposition Database} (1.0.0), \url{https://doi.org/10.5281/ZENODO.11276065}, 2024b.

\leavevmode\vadjust pre{\hypertarget{ref-Teickner.2024}{}}%
Teickner, H., Pebesma, E., and Knorr, K.-H.: A synthesis of {\emph{Sphagnum}} litterbag experiments: {Initial} leaching losses bias decomposition rate estimates, 2024.

\leavevmode\vadjust pre{\hypertarget{ref-Thormann.2001}{}}%
Thormann, M. N., Bayley, S. E., and Currah, R. S.: Comparison of decomposition of belowground and aboveground plant litters in peatlands of boreal {Alberta}, {Canada}, Canadian Journal of Botany, 79, 9--22, \url{https://doi.org/10.1139/b00-138}, 2001.

\leavevmode\vadjust pre{\hypertarget{ref-Treat.2021}{}}%
Treat, C. C., Jones, M. C., Alder, J., Sannel, A. B. K., Camill, P., and Frolking, S.: Predicted vulnerability of carbon in permafrost peatlands with future climate change and permafrost thaw in western {Canada}, Journal of Geophysical Research: Biogeosciences, 126, e2020JG005872, \url{https://doi.org/10.1029/2020JG005872}, 2021.

\leavevmode\vadjust pre{\hypertarget{ref-Treat.2022}{}}%
Treat, C. C., Jones, M. C., Alder, J., and Frolking, S.: Hydrologic controls on peat permafrost and carbon processes: {New} insights from past and future modeling, Frontiers in Environmental Science, 10, 892925, \url{https://doi.org/10.3389/fenvs.2022.892925}, 2022.

\leavevmode\vadjust pre{\hypertarget{ref-Trinder.2008}{}}%
Trinder, C. J., Johnson, D., and Artz, R. R. E.: Interactions among fungal community structure, litter decomposition and depth of water table in a cutover peatland, FEMS Microbiology Ecology, 64, 433--448, \url{https://doi.org/10.1111/j.1574-6941.2008.00487.x}, 2008.

\leavevmode\vadjust pre{\hypertarget{ref-Tuittila.2013}{}}%
Tuittila, E.-S., Juutinen, S., Frolking, S., Väliranta, M., Laine, A. M., Miettinen, A., Seväkivi, M.-L., Quillet, A., and Merilä, P.: Wetland chronosequence as a model of peatland development: {Vegetation} succession, peat and carbon accumulation, The Holocene, 23, 25--35, \url{https://doi.org/10.1177/0959683612450197}, 2013.

\leavevmode\vadjust pre{\hypertarget{ref-vanBreemen.1995}{}}%
van Breemen, N.: How {\emph{Sphagnum}} bogs down other plants, 6, 1995.

\leavevmode\vadjust pre{\hypertarget{ref-Vehtari.2021}{}}%
Vehtari, A., Gelman, A., Simpson, D., Carpenter, B., and Bürkner, P.-C.: Rank-{Normalization}, {Folding}, and {Localization}: {An Improved R{\^{}}} for {Assessing Convergence} of {MCMC} (with {Discussion}), Bayesian Analysis, 16, \url{https://doi.org/10.1214/20-BA1221}, 2021.

\leavevmode\vadjust pre{\hypertarget{ref-Vitt.1990}{}}%
Vitt, D. H.: Growth and production dynamics of boreal mosses over climatic, chemical and topographic gradients, Botanical Journal of the Linnean Society, 104, 35--59, \url{https://doi.org/10.1111/j.1095-8339.1990.tb02210.x}, 1990.

\leavevmode\vadjust pre{\hypertarget{ref-Walpen.2018}{}}%
Walpen, N., Lau, M. P., Fiskal, A., Getzinger, G. J., Meyer, S. A., Nelson, T. F., Lever, M. A., Schroth, M. H., and Sander, M.: Oxidation of reduced peat particulate organic matter by dissolved oxygen: {Quantification} of apparent rate constants in the field, Environmental Science \& Technology, 52, 11151--11160, \url{https://doi.org/10.1021/acs.est.8b03419}, 2018.

\leavevmode\vadjust pre{\hypertarget{ref-Yu.2001}{}}%
Yu, Z., Turetsky, M. R., Campbell, I. D., and Vitt, D. H.: Modelling long-term peatland dynamics. {II}. {Processes} and rates as inferred from litter and peat-core data, Ecological Modelling, 145, 159--173, \url{https://doi.org/10.1016/S0304-3800(01)00387-8}, 2001.

\leavevmode\vadjust pre{\hypertarget{ref-Yu.2012}{}}%
Yu, Z. C.: Northern peatland carbon stocks and dynamics: A review, Biogeosciences, 9, 4071--4085, \url{https://doi.org/10.5194/bg-9-4071-2012}, 2012.

\leavevmode\vadjust pre{\hypertarget{ref-Zhao.2022}{}}%
Zhao, B., Zhuang, Q., Treat, C., and Frolking, S.: A model intercomparison analysis for controls on {C} accumulation in {North American} peatlands, Journal of Geophysical Research: Biogeosciences, \url{https://doi.org/10.1029/2021JG006762}, 2022.

\end{CSLReferences}

\end{document}
