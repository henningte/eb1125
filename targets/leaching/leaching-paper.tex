% Options for packages loaded elsewhere
\PassOptionsToPackage{unicode}{hyperref}
\PassOptionsToPackage{hyphens}{url}
%
\documentclass[
  12pt,
]{article}
\usepackage{amsmath,amssymb}
\usepackage{lmodern}
\usepackage{iftex}
\ifPDFTeX
  \usepackage[T1]{fontenc}
  \usepackage[utf8]{inputenc}
  \usepackage{textcomp} % provide euro and other symbols
\else % if luatex or xetex
  \usepackage{unicode-math}
  \defaultfontfeatures{Scale=MatchLowercase}
  \defaultfontfeatures[\rmfamily]{Ligatures=TeX,Scale=1}
\fi
% Use upquote if available, for straight quotes in verbatim environments
\IfFileExists{upquote.sty}{\usepackage{upquote}}{}
\IfFileExists{microtype.sty}{% use microtype if available
  \usepackage[]{microtype}
  \UseMicrotypeSet[protrusion]{basicmath} % disable protrusion for tt fonts
}{}
\makeatletter
\@ifundefined{KOMAClassName}{% if non-KOMA class
  \IfFileExists{parskip.sty}{%
    \usepackage{parskip}
  }{% else
    \setlength{\parindent}{0pt}
    \setlength{\parskip}{6pt plus 2pt minus 1pt}}
}{% if KOMA class
  \KOMAoptions{parskip=half}}
\makeatother
\usepackage{xcolor}
\IfFileExists{xurl.sty}{\usepackage{xurl}}{} % add URL line breaks if available
\IfFileExists{bookmark.sty}{\usepackage{bookmark}}{\usepackage{hyperref}}
\hypersetup{
  pdftitle={A Synthesis of Sphagnum Litterbag Experiments: Initial Leaching Losses Bias Decomposition Rate Estimates},
  pdfauthor={Henning Teickner1,2,; Edzer Pebesma2; Klaus-Holger Knorr1},
  pdfkeywords={Sphagnum, litterbag, decomposition, leaching, error analysis, Bayesian hierarchical model},
  hidelinks,
  pdfcreator={LaTeX via pandoc}}
\urlstyle{same} % disable monospaced font for URLs
\usepackage[margin=1in]{geometry}
\usepackage{longtable,booktabs,array}
\usepackage{calc} % for calculating minipage widths
% Correct order of tables after \paragraph or \subparagraph
\usepackage{etoolbox}
\makeatletter
\patchcmd\longtable{\par}{\if@noskipsec\mbox{}\fi\par}{}{}
\makeatother
% Allow footnotes in longtable head/foot
\IfFileExists{footnotehyper.sty}{\usepackage{footnotehyper}}{\usepackage{footnote}}
\makesavenoteenv{longtable}
\usepackage{graphicx}
\makeatletter
\def\maxwidth{\ifdim\Gin@nat@width>\linewidth\linewidth\else\Gin@nat@width\fi}
\def\maxheight{\ifdim\Gin@nat@height>\textheight\textheight\else\Gin@nat@height\fi}
\makeatother
% Scale images if necessary, so that they will not overflow the page
% margins by default, and it is still possible to overwrite the defaults
% using explicit options in \includegraphics[width, height, ...]{}
\setkeys{Gin}{width=\maxwidth,height=\maxheight,keepaspectratio}
% Set default figure placement to htbp
\makeatletter
\def\fps@figure{htbp}
\makeatother
\setlength{\emergencystretch}{3em} % prevent overfull lines
\providecommand{\tightlist}{%
  \setlength{\itemsep}{0pt}\setlength{\parskip}{0pt}}
\setcounter{secnumdepth}{5}
\newlength{\cslhangindent}
\setlength{\cslhangindent}{1.5em}
\newlength{\csllabelwidth}
\setlength{\csllabelwidth}{3em}
\newlength{\cslentryspacingunit} % times entry-spacing
\setlength{\cslentryspacingunit}{\parskip}
\newenvironment{CSLReferences}[2] % #1 hanging-ident, #2 entry spacing
 {% don't indent paragraphs
  \setlength{\parindent}{0pt}
  % turn on hanging indent if param 1 is 1
  \ifodd #1
  \let\oldpar\par
  \def\par{\hangindent=\cslhangindent\oldpar}
  \fi
  % set entry spacing
  \setlength{\parskip}{#2\cslentryspacingunit}
 }%
 {}
\usepackage{calc}
\newcommand{\CSLBlock}[1]{#1\hfill\break}
\newcommand{\CSLLeftMargin}[1]{\parbox[t]{\csllabelwidth}{#1}}
\newcommand{\CSLRightInline}[1]{\parbox[t]{\linewidth - \csllabelwidth}{#1}\break}
\newcommand{\CSLIndent}[1]{\hspace{\cslhangindent}#1}
\usepackage{float}
\usepackage[version=4]{mhchem}
\usepackage{booktabs}
\usepackage{bm}
\usepackage{xr} \externaldocument{leaching-supporting-info}
\usepackage{lineno}
\linenumbers
\usepackage{booktabs}
\usepackage{longtable}
\usepackage{array}
\usepackage{multirow}
\usepackage{wrapfig}
\usepackage{float}
\usepackage{colortbl}
\usepackage{pdflscape}
\usepackage{tabu}
\usepackage{threeparttable}
\usepackage{threeparttablex}
\usepackage[normalem]{ulem}
\usepackage{makecell}
\usepackage{xcolor}
\ifLuaTeX
  \usepackage{selnolig}  % disable illegal ligatures
\fi

\title{A Synthesis of \emph{Sphagnum} Litterbag Experiments: Initial Leaching Losses Bias Decomposition Rate Estimates}
\author{Henning Teickner\textsuperscript{1,2,*} \and Edzer Pebesma\textsuperscript{2} \and Klaus-Holger Knorr\textsuperscript{1}}
\date{24 May, 2024}

\begin{document}
\maketitle
\begin{abstract}
Decomposition is one of the major controls of long-term sequestration of carbon in northern peatlands. Our knowledge of the magnitude and controls of decomposition rates is derived to a large extent from litterbag experiments. Similar to other litter types, initial leaching losses may bias decomposition rates of \emph{Sphagnum}, but their magnitudes and variability are not well known.\\
We present a meta-analysis of 15 \emph{Sphagnum} litterbag studies to estimate initial leaching losses (\(l_0\)), to test whether initial leaching losses can bias \emph{Sphagnum} decomposition rate estimates (\(k_0\)), and to analyze how initial leaching losses increase errors in \(k_0\) estimates.\\
Average \(l_0\) estimates range between 3 to 18 mass-\%, average \(k_0\) estimates between 0.01 to 1.16 yr\(^{-1}\). Simulations and models fitted to empirical data indicate that ignoring initial leaching losses leads to an overestimation of \(k_0\). An error analysis suggests that \(k_0\) and \(l_0\) can be estimated only with relatively large errors because of limitations in the design of most available litterbag experiments. Sampling the first litterbags shortly after the start of the experiments allows more accurate estimation of \(l_0\) and \(k_0\). We estimated large \(l_0\) (\(>5\) mass-\%) also for only air-dried samples which could imply that \emph{Sphagnum} litterbag experiments with dried litter are unrepresentative for natural decomposition processes in which \(l_0\) may be smaller according to leaching experiments with fresh litter.\\
We conclude that comparing results of litterbag experiments between experimental treatments and between studies and accurately estimating decomposition rates may be possible only if initial leaching losses are explicitly considered.
\end{abstract}

\textsuperscript{1} ILÖK, Ecohydrology \& Biogeochemistry Group, Institute of Landscape Ecology, University of Münster, 48149, Germany\\
\textsuperscript{2} IfGI, Spatiotemporal Modelling Lab, Institute for Geoinformatics, University of Münster, 48149, Germany

\textsuperscript{*} Correspondence: \href{mailto:henning.teickner@uni-muenster.de}{Henning Teickner \textless{}\href{mailto:henning.teickner@uni-muenster.de}{\nolinkurl{henning.teickner@uni-muenster.de}}\textgreater{}}

Keywords: Sphagnum, litterbag, decomposition, leaching, error analysis, Bayesian hierarchical model

\hypertarget{introduction}{%
\section{Introduction}\label{introduction}}

Decomposition is one of the major controls of long-term sequestration of carbon in northern peatlands, which are a large global store of carbon sequestered from the atmosphere (\protect\hyperlink{ref-Yu.2012}{Yu, 2012}). Our knowledge of the magnitude and controls of decomposition rates is derived to a large extent from litterbag experiments (\protect\hyperlink{ref-Rydin.2013}{Rydin et al., 2013}) and these estimates inform parameters in long-term peatland models (e.g. Frolking et al. (\protect\hyperlink{ref-Frolking.2010}{2010})). To make correct inferences about decomposition processes and past and future controls of peat accumulation, it has therefore to be validated that the decomposition rate estimates from litterbag experiments are unbiased.

In litterbag experiments, a defined mass of litter or peat is filled into bags which are buried into a peatland or laboratory container and after weeks to years of decomposition excavated, dried, and reweighed. From the resulting mass trajectories over time, decomposition rates can be estimated with suitable decomposition models (e.g. Frolking et al. (\protect\hyperlink{ref-Frolking.2001}{2001}), Rovira and Rovira (\protect\hyperlink{ref-Rovira.2010}{2010})) and how they depend on environmental conditions. Finally, decomposition rate estimates are used to define parameter values in peatland models which are a major tool to analyze peat accumulation and process interactions during time ranges exceeding the duration of litterbag experiments.

A potential problem with current estimates of \emph{Sphagnum} decomposition rates is that many of them ignore initial leaching losses which has the potential to bias decomposition rate estimates and therefore peat models. Initial leaching losses are here defined as the export and possible mineralization of water extractable organic matter from litter within the first period of decomposition, typically observed within the first two days to three weeks for \emph{Sphagnum} and peat (\protect\hyperlink{ref-Coulson.1978}{Coulson and Butterfield, 1978}; \protect\hyperlink{ref-Thormann.2001}{Thormann et al., 2001}; \protect\hyperlink{ref-Moore.2001}{Moore and Dalva, 2001}; \protect\hyperlink{ref-Kim.2014}{Kim et al., 2014}; \protect\hyperlink{ref-Muller.2023}{Müller et al., 2023}), after which mass loss rates decrease markedly. Ignoring initial leaching losses means to estimate a one-pool decomposition rate from litterbag data and taking this to represent depolymerization. In reality, depolymerization typically is slower than initial leaching and ignoring initial leaching losses can therefore lead to larger decomposition rate estimates which would overestimate depolymerization on longer time scales relevant to peatland models.

Yu et al. (\protect\hyperlink{ref-Yu.2001}{2001}) illustrated this by re-analyzing data from a \emph{Rubus chamaemorus} litterbag experiment with a one-pool exponential model (ignoring initial leaching losses) and a two-pool exponential model, where the first pool represents initial leaching losses, and came to the conclusion that ignoring initial leaching losses causes non-negligible overestimation of decomposition rates. Similar analyses with comparable outcomes have been performed for non-peatland vegetation (e.g., Bärlocher (\protect\hyperlink{ref-Barlocher.1997}{1997})) and tea bags (\protect\hyperlink{ref-Lind.2022}{Lind et al., 2022}). A systematic analysis for \emph{Sphagnum} litter, which often has smaller decomposition rates, may have smaller initial leaching losses, and often represents the bulk of peat material, has not been performed yet to our knowledge.

Available estimates from direct measurement and few two-pool litterbag experiments suggest that initial leaching losses from \emph{Sphagnum} range from \(<1\) to 18 percent of the initial mass (mass-\%) (\protect\hyperlink{ref-Coulson.1978}{Coulson and Butterfield, 1978}; \protect\hyperlink{ref-Scheffer.2001}{Scheffer et al., 2001}; \protect\hyperlink{ref-Moore.2001}{Moore and Dalva, 2001}; \protect\hyperlink{ref-Thormann.2002}{Thormann et al., 2002}; \protect\hyperlink{ref-Limpens.2003}{Limpens and Berendse, 2003}; \protect\hyperlink{ref-Castells.2005}{Castells et al., 2005}; \protect\hyperlink{ref-Moore.2007}{Moore et al., 2007}; \protect\hyperlink{ref-DelGiudice.2017}{Del Giudice and Lindo, 2017}; \protect\hyperlink{ref-Mastny.2018}{Mastný et al., 2018}; \protect\hyperlink{ref-Muller.2023}{Müller et al., 2023}). Some studies argued that larger leaching losses of 8 or more percent estimated by Scheffer et al. (\protect\hyperlink{ref-Scheffer.2001}{2001}) are artifacts from freeze-drying \emph{Sphagnum} material which disrupts cell walls (\protect\hyperlink{ref-Limpens.2003}{Limpens and Berendse, 2003}), and that leaching from \emph{Sphagnum} generally accounts for only few percent (\protect\hyperlink{ref-Johnson.1991}{Johnson and Damman, 1991}). This is in line with small leaching losses reported in most of the studies which explicitly quantified initial leaching losses (\protect\hyperlink{ref-Coulson.1978}{Coulson and Butterfield, 1978}; \protect\hyperlink{ref-Moore.2001}{Moore and Dalva, 2001}; \protect\hyperlink{ref-Thormann.2002}{Thormann et al., 2002}; \protect\hyperlink{ref-Limpens.2003}{Limpens and Berendse, 2003}; \protect\hyperlink{ref-Castells.2005}{Castells et al., 2005}; \protect\hyperlink{ref-Mastny.2018}{Mastný et al., 2018}). However, larger potential leaching has also been reported or can be estimated for only air- or oven-dried samples, e.g. Moore et al. (\protect\hyperlink{ref-Moore.2007}{2007}) (supporting information \ref{sup-1}), Thormann et al. (\protect\hyperlink{ref-Thormann.2001}{2001}), Müller et al. (\protect\hyperlink{ref-Muller.2023}{2023}). In addition, experiments have shown that air drying of non-\emph{Sphagnum} litter can increase initial leaching losses relative to undried litter (\protect\hyperlink{ref-Gessner.1989}{Gessner and Schwoerbel, 1989}; \protect\hyperlink{ref-Barlocher.1997}{Bärlocher, 1997}) and that effects of drying are variable between species (\protect\hyperlink{ref-Taylor.1996}{Taylor and Bärlocher, 1996}). This indicates that initial leaching losses from \emph{Sphagnum} can be larger than the few percents assumed by some previous studies, even if the litter was only air-dried. With decomposition rates ranging from \(<0.01\) to around 0.15 yr\(^{-1}\) (e.g., Moore et al. (\protect\hyperlink{ref-Moore.2007}{2007}), Turetsky et al. (\protect\hyperlink{ref-Turetsky.2008}{2008})), initial leaching losses in the range from \(<1\) to 18 mass-\% could bias decomposition rate estimates.

Our aims are to quantify the magnitude and variability of initial leaching losses for \emph{Sphagnum} litterbag experiments, to analyze how much decomposition rate estimates are biased when initial leaching losses are ignored, and to analyze how one could improve the design of litterbag experiments to avoid such biases and more accurately estimate decomposition rates. Specifically, we address the following questions:

\begin{enumerate}
\def\labelenumi{\arabic{enumi}.}
\item
  What is the magnitude of initial leach losses and their variability between species and studies?
\item
  How does ignoring initial leaching losses bias decomposition rate estimates in \emph{Sphagnum} litterbag experiments?
\item
  What conditions may cause small initial leaching losses from \emph{Sphagnum} litter?
\item
  How to design litterbag experiments to improve estimates of decomposition rates?
\end{enumerate}

To address these questions, we first simulate litterbag experiments with initial leaching losses of different magnitude, fit a one pool exponential decomposition model that ignores initial leaching losses, and analyze how much \(k_0\) estimates are biased. Next, we re-analyze litterbag experiments collected from the literature with a one pool decomposition model that ignores initial leaching losses and a two pool model that estimates initial leaching losses from the data and compare their results. Finally, we use error and sensitivity analyses, to test which litterbag experiment designs allow to most accurately estimate \(l_0\) and \(k_0\).

Since our arguments about the importance of initial leaching losses are general and in line with findings for non-\emph{Sphagnum} litter (e.g., Bärlocher (\protect\hyperlink{ref-Barlocher.1997}{1997}), Lind et al. (\protect\hyperlink{ref-Lind.2022}{2022})), we expect that our study is also relevant for evaluating litterbag experiments of vascular plant and lichen litter in peatlands. Given that decomposition rates in long-term dynamic peatland models are mainly parameterized based on data from litterbag studies (e.g. Frolking et al. (\protect\hyperlink{ref-Frolking.2001}{2001}), Bauer (\protect\hyperlink{ref-Bauer.2004}{2004}), Heijmans et al. (\protect\hyperlink{ref-Heijmans.2008}{2008}), Heinemeyer et al. (\protect\hyperlink{ref-Heinemeyer.2010}{2010}), Morris et al. (\protect\hyperlink{ref-Morris.2012}{2012}), Chaudhary et al. (\protect\hyperlink{ref-Chaudhary.2018}{2018}), Bona et al. (\protect\hyperlink{ref-Bona.2020}{2020})), our analysis indicates that they should use decomposition rates obtained from litterbag experiments that consider intitial leaching losses.

\hypertarget{methods}{%
\section{Methods}\label{methods}}

\hypertarget{out-methods-1}{%
\subsection{Modeling leaching losses in litterbag experiments}\label{out-methods-1}}

A general formula for the change in remaining mass with incubation duration \(t\) of a litterbag experiment is (\protect\hyperlink{ref-Frolking.2001}{Frolking et al., 2001}) (equation \eqref{eq:decomposition-differential-1}):

\begin{equation}
\frac{dm(t)}{dt} = -k_0 m_0 \left(\frac{m(t)}{m_0}\right)^\alpha,
\label{eq:decomposition-differential-1}
\end{equation}

where \(m(t)\) is the remaining mass at time \(t\) after the start of the incubation, \(k_0\) is the decomposition rate constant, \(m_0\) is the initial mass (\(m(t = 0)\)), and \(\left(\frac{m(t)}{m_0}\right)^\alpha\), with \(\alpha \ge 0\) describes how the decomposition rate changes as mass is lost over time (if \(\alpha<1\), the decomposition rate increases as mass is lost, if \(\alpha=1\), the decomposition rate is constant, if \(\alpha>1\), the decomposition rate decreases as mass is lost).

If \(\alpha = 1\), the solution of equation \eqref{eq:decomposition-differential-1} is the simple one-pool exponential decomposition model (\protect\hyperlink{ref-Frolking.2001}{Frolking et al., 2001}):

\begin{equation}
m(t) = m_0 \exp(-k_0 t)
\label{eq:decomposition-solution-1-no-leaching-1}
\end{equation}

In this study, we define initial leaching losses as export of water-extractable organic matter from the litter due to diffusive or advective transport or respiration of soluble compounds within the first period (up to three weeks (\protect\hyperlink{ref-Coulson.1978}{Coulson and Butterfield, 1978}; \protect\hyperlink{ref-Thormann.2001}{Thormann et al., 2001}; \protect\hyperlink{ref-Moore.2001}{Moore and Dalva, 2001}; \protect\hyperlink{ref-Kim.2014}{Kim et al., 2014}; \protect\hyperlink{ref-Muller.2023}{Müller et al., 2023})) of a litterbag experiment to differentiate it from the subsequent decomposition of polymeric organic matter which is the dominant process by which mass is lost in the long-term. Initial leaching losses can be included in equation \eqref{eq:decomposition-solution-1-no-leaching-1} as constant parameter \(l_0\) which gets subtracted from \(m_0\) if \(t>0\):

\begin{equation}
m(t) = \begin{cases}
m_0 & \mathrm{if}~t=0\\
(m_0 - l_0) \exp(-k_0 t) & \mathrm{if}~t>0\\
\end{cases}
\label{eq:decomposition-solution-1-with-leaching-1}
\end{equation}

An alternative would be to define a two-pool exponential decomposition model where one of the pools represents initial leaching losses (e.g. Yu et al. (\protect\hyperlink{ref-Yu.2001}{2001}), Rovira and Rovira (\protect\hyperlink{ref-Rovira.2010}{2010}), Hagemann and Moroni (\protect\hyperlink{ref-Hagemann.2015}{2015})). However if the data have no daily resolution, this is equivalent to the previous simpler approach.

If \(\alpha>1\), the decomposition rate decreases as mass has been lost which is in line with the assumption that litter quality decreases during decomposition. With \(\alpha>1\), equation \eqref{eq:decomposition-differential-1} has the following solution (\protect\hyperlink{ref-Frolking.2001}{Frolking et al., 2001}):

\begin{equation}
m(t) = \frac{m_0}{(1 + (\alpha - 1) k_0 t)^{\frac{1}{\alpha - 1}}}
\label{eq:decomposition-solution-2-no-leaching-1}
\end{equation}

Or, if initial leaching losses are considered as in equation \eqref{eq:decomposition-solution-1-with-leaching-1}:

\begin{equation}
m(t) = \begin{cases}
m_0 & \mathrm{if}~t=0\\
\frac{m_0 - l_0}{(1 + (\alpha - 1) k_0 t)^{\frac{1}{\alpha - 1}}} & \mathrm{if}~t>0\\
\end{cases}
\label{eq:decomposition-solution-2-with-leaching-1}
\end{equation}

Over longer time periods, \(\alpha\) is an important control of remaining masses (\protect\hyperlink{ref-Frolking.2001}{Frolking et al., 2001}) and is therefore included in the Holocene Peatland Model (\protect\hyperlink{ref-Frolking.2010}{Frolking et al., 2010}), one of the peatland models studied in many studies. Even though \(\alpha\) has little influence on remaining masses during time ranges as covered by litterbag experiments (\protect\hyperlink{ref-Frolking.2001}{Frolking et al., 2001}; \protect\hyperlink{ref-Frolking.2010}{Frolking et al., 2010}), it needs to be considered to accurately estimate \(l_0\).

Fig. \ref{fig:out-leaching-sim-interpretation-alpha} illustrates how different values for \(\alpha\), \(l_0\), and \(k_0\) can produce comparable fits to litterbag data while representing contrasting interpretations of the decomposition process: In the first case, \(\alpha=2\) and a larger \(l_0\) and smaller \(k_0\) fit the litterbag data and this corresponds to the decomposition process assumed in the Holocene Peatland Model (\protect\hyperlink{ref-Frolking.2010}{Frolking et al., 2010}). In the second case, a comparable fit is achieved by setting \(l_0=0\) mass-\% and instead increasing \(\alpha\) and \(k_0\). In the latter case, the change in mass caused by initial leaching is captured by assuming a very large initial decomposition rate that decreases rapidly. This also implies strongly reduced decomposition rates when extrapolating to longer time ranges and therefore describes a completely different decomposition process than intended in the Holocene Peatland Model. Therefore \(\alpha\) needs to be considered to obtain estimates for \(l_0\) which are consistent with a particular interpretation of the decomposition process.

In our simulation analysis, we assume \(\alpha = 1\) to make the results comparable to previous evaluations of litterbag experiments. For the same reason, we also assume \(\alpha = 1\) when analyzing how ignoring initial leaching losses biases \(k_0\) estimates for available litterbag data. To provide estimates for \(l_0\) and \(k_0\) in available litterbag experiments that consider some of the uncertainty about \(\alpha\) and constrain it to values near 2 to make sure that the model does not confound the slowdown of depolymerization as described by \(\alpha\) in the Holocene Peatland Model with the slowdown of leaching losses after the initial period.

In supporting information \ref{sup-2} we show that estimating \(\alpha\) from the litterbag data while ignoring initial leaching losses causes even larger bias of \(k_0\) estimates than when \(\alpha\) is set to \(1\). In supporting information \ref{sup-11}, we show that \(\alpha\) cannot be accurately estimated even when combining data from available litterbag experiments, and that uncertainty about \(\alpha\) has little effect on the accuracy with which we could estimate \(k_0\) and \(l_0\) as long as \(\alpha\) is forced to a value near 2.



\begin{figure}[H]

{\centering \includegraphics[width=1\linewidth]{figures/leaching_simulation_interpretation_alpha_1} 

}

\caption{Remaining masses during two hypothetical litterbag experiments where decomposition is controlled by different sets of parameter values for \(l_0\), \(k_0\), and \(\alpha\). As can be seen, very similar remaining masses can be produced for a typical litterbag experiment (incubation duration \(\le5\) years) either with an initial leaching loss \(>0\) mass-\%, a small \(k_0\) and a small \(\alpha\), or without initial leaching loss, a large \(k_0\), and a large \(\alpha\). Extrapolation to longer incubation durations shows that both models represent different interpretations of the decomposition process (dashed lines).}\label{fig:out-leaching-sim-interpretation-alpha}
\end{figure}

\hypertarget{database-of-sphagnum-litterbag-decomposition-data}{%
\subsection{Database of Sphagnum litterbag decomposition data}\label{database-of-sphagnum-litterbag-decomposition-data}}

Through a Scopus search with search string \texttt{(\ TITLE-ABS-KEY\ (\ peat*\ AND\ (\ "litter\ bag"\ OR\ "decomposition\ rate"\ OR\ "decay\ rate"\ OR\ "mass\ loss"\ )\ )\ AND\ NOT\ (\ "tropic*"\ )\ )} (2022-12-17), we identified studies which analyzed litterbag data in northern peatlands. These studies were further screened to exclude those which do not contain litterbag data or which recycle data from other studies which have already been obtained or which do not use \emph{Sphagnum} litter (identified down to the species level) or which did not include any estimate for water table depths. Authors of the selected studies not older than 10 years were contacted to obtain raw data. In case this was not successful or studies were older than 10 years, relevant data (remaining masses, species identified, mesh sizes, incubation durations, depths where litter were buried, senescence status of collected litter, water table depths) were extracted from the papers where possible. The data are accessible from the Peatland Decomposition Database (\protect\hyperlink{ref-Teickner.2024c}{Teickner and Knorr, 2024}).

In this study, we use data from 15 studies which sampled litterbags at least at two time points after the start of the incubation because otherwise \(k_0\) and \(l_0\) become unidentifiable. The selected studies are: Bartsch and Moore (\protect\hyperlink{ref-Bartsch.1985}{1985}), Vitt (\protect\hyperlink{ref-Vitt.1990}{1990}), Johnson and Damman (\protect\hyperlink{ref-Johnson.1991}{1991}), Szumigalski and Bayley (\protect\hyperlink{ref-Szumigalski.1996}{1996}), Prevost et al. (\protect\hyperlink{ref-Prevost.1997}{1997}), Scheffer et al. (\protect\hyperlink{ref-Scheffer.2001}{2001}), Thormann et al. (\protect\hyperlink{ref-Thormann.2001}{2001}), Asada and Warner (\protect\hyperlink{ref-Asada.2005b}{2005}), Trinder et al. (\protect\hyperlink{ref-Trinder.2008}{2008}), Breeuwer et al. (\protect\hyperlink{ref-Breeuwer.2008}{2008}), Straková et al. (\protect\hyperlink{ref-Strakova.2010}{2010}), Hagemann and Moroni (\protect\hyperlink{ref-Hagemann.2015}{2015}), Bengtsson et al. (\protect\hyperlink{ref-Bengtsson.2017}{2017}), Golovatskaya and Nikonova (\protect\hyperlink{ref-Golovatskaya.2017}{2017}), and Mäkilä et al. (\protect\hyperlink{ref-Makila.2018}{2018}). Samples originally classified as \emph{Sphagnum magellanicum} are here classified as \emph{Sphagnum magellanicum aggr.} (\protect\hyperlink{ref-Hassel.2018}{Hassel et al., 2018}).

\hypertarget{simulation-to-check-how-initial-leaching-losses-can-potentially-confound-sphagnum-decomposition-rate-estimates}{%
\subsection{\texorpdfstring{Simulation to check how initial leaching losses can potentially confound \emph{Sphagnum} decomposition rate estimates}{Simulation to check how initial leaching losses can potentially confound Sphagnum decomposition rate estimates}}\label{simulation-to-check-how-initial-leaching-losses-can-potentially-confound-sphagnum-decomposition-rate-estimates}}

As a first step, we simulated \emph{Sphagnum} litterbag data with initial leaching losses of different magnitude and then analyzed how \(k_0\) estimates are biased if the data are fitted with a one-pool exponential decomposition model that ignores initial leaching.

We used equation \eqref{eq:decomposition-solution-1-with-leaching-1} to simulate litterbag mass-time trajectories over five years, assuming \(m(t_0)=1\), \(l_0\) ranging between 0 and 15 mass-\%, and \(k_0\) of either 0.01, 0.05, or 0.15 (the range roughly covered by \emph{Sphagnum} in litterbag experiments (e.g., Moore et al. (\protect\hyperlink{ref-Moore.2007}{2007}), Turetsky et al. (\protect\hyperlink{ref-Turetsky.2008}{2008}))). To avoid a perfect fit of the models, we added a small amount of noise to the trajectories. Fig. \ref{fig:out-simulation-p1} shows the result.

We then simulated litterbag retrievals after half a year, one year, two years, three years, and five years to simulate a litterbag study with relatively high temporal resolution and long duration. This results in a subset of the litterbag mass-time trajectory which mimics real litterbag data compatible with equation \eqref{eq:decomposition-solution-1-with-leaching-1}. This subset of the simulated masses is shown as points in Fig. \ref{fig:out-simulation-p1}.

These simulated masses were then used to fit the model ignoring initial leach loss (equation \eqref{eq:decomposition-solution-1-no-leaching-1}) using nonlinear least squares regression regression to obtain estimated average and standard deviation for \(k_0\), as is often done in litterbag experiments. We compared these values to the decomposition rate values that were used to simulate the data.

This allowed us to analyze how decomposition rate estimates get biased in dependency of initial leaching losses and how their errors are influenced by initial leaching losses if they are ignored during data analysis. We also analyzed how the estimated models fit the remaining masses and how predicted remaining masses are biased when the decomposition rate estimates are used for extrapolations to 20 or 100 years, as would be the case when the estimates would be directly used in a long-term peatland model and all conditions except the remaining mass were kept constant.



\begin{figure}[H]

{\centering \includegraphics[width=1\linewidth]{figures/leaching_plot_simulation_p1} 

}

\caption{Mass trajectories of simulated litterbag experiments over five years with three different decomposition rates (yr\(^{-1}\)) indicated by panel titles and five initial leaching loss levels (indicated by the color gradient). Lines represent the remaining mass of a litter replicate over time and points represent the simulated sampling dates for litterbag replicates after half a year, one year, two years, three years, and five years.}\label{fig:out-simulation-p1}
\end{figure}

\hypertarget{methods-bias-real-1}{%
\subsection{\texorpdfstring{Estimating the bias in \(k_0\) in available litterbag experiments when initial leaching is ignored}{Estimating the bias in k\_0 in available litterbag experiments when initial leaching is ignored}}\label{methods-bias-real-1}}

To analyze how \(k_0\) estimates for available litterbag data change when we consider or ignore initial leaching losses, we fitted a model that considers initial leaching losses (equation \eqref{eq:decomposition-solution-1-with-leaching-1}) and a model that ignores initial leaching losses (equation \eqref{eq:decomposition-solution-1-no-leaching-1}) to the synthesized litterbag data. As described in section \ref{out-methods-1}, we assume \(\alpha=1\) for this analysis. In supporting table \ref{tab:m-litterbag-synthesis-models} we provide a list of all models computed in this study.

We excluded data from Bengtsson et al. (\protect\hyperlink{ref-Bengtsson.2017}{2017}), a large laboratory study where litterbags were incubated in water-filled containers and for which the model estimated larger \(l_0\) than for any other study, to make sure that our estimates are representative for conditions similar to field conditions (160 out of 289 litterbag experiments were from Bengtsson et al. (\protect\hyperlink{ref-Bengtsson.2017}{2017})). Results of the same models including data from Bengtsson et al. (\protect\hyperlink{ref-Bengtsson.2017}{2017}) are shown in supporting information \ref{sup-3} and \ref{sup-7} and the average estimates for other studies were not much changed when data from Bengtsson et al. (\protect\hyperlink{ref-Bengtsson.2017}{2017}) were excluded or included.

The models assumed a Beta distribution for the fraction of initial mass remaining and a Gamma distribution for the precision parameter (\(\phi\)) of the Beta distribution which was computed from reported standard deviations (see supporting information \ref{sup-13}). Where no standard deviation was reported, \(\phi\) was estimated from the data. Remaining masses larger than 100 mass-\% for some experiments are due to net import of matter during the experiment and were corrected to 100 mass-\%, to make the data compatible with a Beta distribution.

We used mixed effects models (Bayesian hierarchical models) to pool information across relevant groups. Group-level intercepts for \(k_0\), \(l_0\), and \(\phi\) were estimated for species, study-species combinations, and individual experiments within studies, but not for different experimental conditions. For example, \(l_0\) for sample (litterbag experiment) \(i\) is computed as follows:

\begin{equation}
l\_2[i]  =  \text{logit}^{-1}(l\_2\_p1 + l\_2\_p2[\text{species}[i]] + l\_2\_p3[\text{species x study}[i]] + l\_2\_p4[i]),
\label{eq:leaching-hierarchical-model-l0}
\end{equation}

where \(l\_2\_p1\) is the global intercept, and \(l\_2\_p2[\text{species}[i]]\), \(l\_2\_p3[\text{species x study}[i]]\), and \(l\_2\_p4[i]\) are the group-level intercepts for the species, species x study combination, and litterbag experiment (one value per group), respectively. Each of these is assumed to follow a normal distribution with standard deviation following a half-normal distribution:

\begin{equation}
\begin{aligned}
l\_2\_p1 & \sim & \text{Normal}(l\_2\_p1\_p1, l\_2\_p1\_p2)\\
l\_2\_p2_{\text{species}} & \sim & \text{Normal}(l\_2\_p2\_p1, l\_2\_p2\_p2)\\
l\_2\_p3_{\text{species x study}} & \sim & \text{Normal}(l\_2\_p3\_p1, l\_2\_p3\_p2)\\
l\_2\_p4_{\text{samples}} & \sim & \text{Normal}(l\_2\_p4\_p1, l\_2\_p4\_p2)\\
l\_2\_p1\_p2 & \sim & \text{Normal}^+(0, l\_2\_p1\_p2\_p1)\\
l\_2\_p2\_p2_{\text{species}} & \sim & \text{Normal}^+(0, l\_2\_p2\_p2\_p1)\\
l\_2\_p3\_p2_{\text{species x study}} & \sim & \text{Normal}^+(0, l\_2\_p3\_p2\_p1)\\
l\_2\_p4\_p2_{\text{samples}} & \sim & \text{Normal}^+(0, l\_2\_p4\_p2\_p1),\\
\label{eq:leaching-hierarchical-model-l0-2}
\end{aligned}
\end{equation}

where the unknowns are parameters for the prior distributions (see supporting information \ref{sup-13}). There are reasonable objections against this choice of hierarchical levels, most importantly that different experimental designs clearly cause systematic differences in decomposition rates and these differences should be explicitly considered, and that there is consensus that \(k_0\) is smaller for some species (e.g.~\emph{S. fuscum}) than others (\emph{S. fallax}) and one may wish to incorporate this prior information into the analysis instead of assuming all \emph{Sphagnum} species to be exchangeable (\protect\hyperlink{ref-Gelman.2014}{Gelman et al., 2014}).

However, the litterbag experiments are heterogeneous and report heterogeneous information on experimental conditions. Explicitly considering all relevant additional information would therefore require a much more complex model. In addition, where sufficient data are available for individual species, species-specific differences in parameters could be estimated, and where this is not the case, it seems a reasonable choice to assume exchangeability. Future models may consider additional factors. For example, in a future study, we plan to add to the model another model which describes how decomposition rates change along the gradient from oxic to anoxic conditions.

For \(k_0\) and \(\phi\) we assume the same model structure with appropriate link functions. All intercepts are assumed to have a normal distribution. Further details are described in supporting information \ref{sup-13}.

\hypertarget{methods-estimate-real-1}{%
\subsection{\texorpdfstring{Estimating \(l_0\) and \(k_0\) from available litterbag experiments}{Estimating l\_0 and k\_0 from available litterbag experiments}}\label{methods-estimate-real-1}}

To estimate \(k_0\) and \(l_0\) while considering some of the uncertainty about \(\alpha\), we additionally fitted the data with a model that estimates also \(\alpha\) from the data (equation \eqref{eq:decomposition-solution-1-with-leaching-1}), where we assume the same hierarchical structure as for \(l_0\), \(k_0\), and \(\phi\) in the previous model. Here, we did not estimate group-level standard deviations for \(\alpha\) because it is known that litterbag experiments provide little information about \(\alpha\), as mentioned in section \ref{out-methods-1}, and fixing group-level standard deviations avoided potential computational problems.

In supporting information \ref{sup-11}, we analyzed how sensitive parameter estimates are to our prior choices. The sensitivity analysis allowed us to explore what biases can be expected for specific true values of \(k_0\), \(l_0\), \(\alpha\) and this is a rough estimate of the accuracy and errors of the parameters estimated from available litterbag data under different experimental designs. The results indicate that parameter values (except \(\alpha\)) can be estimated accurately with our method when the models are a good approximation to the data generating process. In particular, our estimates for \(l_0\) are conservative.

\hypertarget{error-analysis}{%
\subsection{Error analysis}\label{error-analysis}}

Error analysis allows to estimate the influence that the error of one parameter has on the error of another parameter. Here, we analyze how estimation errors in \(k_0\) are related to errors in \(l_0\) and how this relation depends on aspects of the litterbag experiment. If \(k_0\) estimates have larger errors due to errors in \(l_0\), this indicates that we can reduce errors in \(k_0\) estimates by measuring \(l_0\) more accurately.

We computed the error analysis as suggested in Eriksson et al. (\protect\hyperlink{ref-Eriksson.2019}{2019}). Briefly, this method computes a sensitivity index for some model parameter \(\alpha\) in dependency of another model parameter \(\Theta_i\) (\(S_i(\alpha)\)) using the Markov Chain Monte Carlo (MCMC) draws representing the posterior distribution of a model as:

\begin{equation}
\begin{aligned}
S_i(\alpha) = \frac{V_{\Theta_i}(E_{\bm{{\Theta}}-i}(\alpha|\Theta_i))}{V(\alpha)},
\label{eq:eriksson2019-1}
\end{aligned}
\end{equation}

where \(\bm{{\Theta}}-i\) are all model parameters except \(\Theta_i\) and \(\alpha\), \(E_{\bm{{\Theta}}-i}(\alpha|\Theta_i)\) is the expected value of \(\alpha\) over all parameters except \(\Theta_i\), when \(\Theta_i\) is fixed to a specific value, \(V_{\Theta_i}(E_{\bm{{\Theta}}-i}(\alpha|\Theta_i))\) is the variance over the expected values \(E_{\bm{{\Theta}}-i}(\alpha|\Theta_i)\) for different values of \(\Theta_i\), and \(V(\alpha)\) is the unconditional variance of \(\alpha\). Thus, each sensitivity index \(S_i(\alpha)\) is the variance of expectations of \(\alpha\) if \(\Theta_i\) is fixed to different specific values (while other parameters \(\bm{{\Theta}}-i\) are allowed to vary conditional on the fixed value of \(\Theta_i\)) divided by the variance of \(\alpha\). Larger values of \(S_i(\alpha)\) indicate that \(\alpha\) is more sensitive to \(\Theta_i\).

We are interested in the sensitivity of the decomposition rates for each replicate litterbag (\(k\_2\)) conditional on initial leaching losses for each replicate (\(l\_2\)), \(S_{j, l\_2}(k\_2)\). If differences in \(S_{j, l\_2}(k\_2)\) between litterbag experiments are related to an aspect of the experimental design, this may provide information on how to design litterbag experiments to get more accurate estimates for both \(l_0\) and \(k_0\). We computed \(S_{j, l\_2}(k\_2)\) for each litterbag experiment with MCMC draws from our model.

Intuitively, it would make sense that initial leaching losses can be estimated more accurately if the first litterbag retrieval in a litterbag experiments occurs shortly after the start of the incubation and in these cases we would also expect a small \(S_{j, l\_2}(k\_2)\) because the model has already enough information to separate initial leaching losses and decomposition rates. To test this hypothesis, we computed linear regression models between \(S_{j, l\_2}(k\_2)\) and the duration until the first time a litterbag was retrieved in a litterbag experiment conditional on \(l_0\) and \(k_0\).

\hypertarget{bayesian-data-analysis}{%
\subsection{Bayesian data analysis}\label{bayesian-data-analysis}}

Bayesian data analysis was used to compute all models to account for relevant error sources and include relevant prior knowledge (for example that \emph{Sphagnum} decomposition rates are unlikely to be larger than 0.5 yr\(^{-1}\)). Bayesian computations were performed using Markov Chain Monte Carlo (MCMC) sampling with Stan (2.32.2) (\protect\hyperlink{ref-StanDevelopmentTeam.2021a}{Stan Development Team, 2021b}) and rstan (2.32.5) (\protect\hyperlink{ref-StanDevelopmentTeam.2021b}{Stan Development Team, 2021a}) using the NUTS sampler (\protect\hyperlink{ref-Hoffman.2014}{Hoffman and Gelman, 2014}), with four chains, 4000 total iterations per chain, and 2000 warmup iterations per chain. All models used the same priors for the same parameters and prior choices are listed and justified in supporting Tab. \ref{tab:sup-out-d-sdm-all-models-priors-1}. Further information on the Bayesian data analysis are described in supporting information \ref{sup-14}.

\hypertarget{results}{%
\section{Results}\label{results}}

\hypertarget{fit-of-the-models-to-available-litterbag-data-and-errors-in-parameter-estimates}{%
\subsection{Fit of the models to available litterbag data and errors in parameter estimates}\label{fit-of-the-models-to-available-litterbag-data-and-errors-in-parameter-estimates}}

Average predicted remaining masses of all models, considering or ignoring initial leaching losses, fitted the data well, but errors were often large and the models ignoring initial leaching losses did not fit the data as well, unless \(\alpha\) was also estimated from the data (supporting Fig. \ref{fig:sup-out-sdm-all-models-p1} and supporting Fig. \ref{fig:sup-out-p-sdm-all-models-check-2-1}).
Some litterbag experiments fitted badly under either model. These experiments had average reported remaining masses which increased over time, sampling dates with much larger mass losses compared to previous dates than explainable by the models, or the incubation began in autumn and the replicates experienced cold winters that probably delayed mass losses from both leaching and depolymerization (data from Golovatskaya and Nikonova (\protect\hyperlink{ref-Golovatskaya.2017}{2017})) (supporting Fig. \ref{fig:sup-out-sdm-mm36-1-outlier-decomposition-trajectory-p1}).

Estimated errors for all parameters were comparatively large for initial leaching losses, decomposition rates, and \(\alpha\), with median coefficients of variation of 28, 44, and 38\% respectively, indicating that none of the parameters can be estimated very accurately from available litterbag data.

\hypertarget{out-res-1}{%
\subsection{Magnitude and variation of initial leaching losses and decomposition rates estimated from available litterbag data}\label{out-res-1}}

Estimates for \(l_0\) ranged between 3 to 18 mass-\% (3 and 33 mass-\% when data from Bengtsson et al. (\protect\hyperlink{ref-Bengtsson.2017}{2017}) are also included). There was a large posterior probability (\(>99\)\%) that \(l_0>5\) mass-\% for 42 out of 289 litterbag experiments, that \(l_0>10\) mass-\% for 6 experiments, that \(l_0<5\) mass-\% for none of the experiments, and that \(l_0<10\) mass-\% for 16 experiments. The posterior probability was larger than 70\% that \(l_0<5\) mass-\% for 13 experiments from Bartsch and Moore (\protect\hyperlink{ref-Bartsch.1985}{1985}), Prevost et al. (\protect\hyperlink{ref-Prevost.1997}{1997}), and Golovatskaya and Nikonova (\protect\hyperlink{ref-Golovatskaya.2017}{2017}). Overall, the estimates agree well with the range given in the introduction when data from Bengtsson et al. (\protect\hyperlink{ref-Bengtsson.2017}{2017}) are excluded.

Average \(l_0\) varied between species and studies (Fig. \ref{fig:out-mm36-1-p5}, Tab. \ref{tab:tab-out-fit-4-species-averages-caption}). The median within-species variance was 0.3 times as large as the between-species variance (logit scale). Replicates from Bengtsson et al. (\protect\hyperlink{ref-Bengtsson.2017}{2017}) had the largest leaching losses across species which appear to be a result of the laboratory setup (Fig. \ref{fig:out-mm36-1-p5}).

For species where data from several studies were available, the variation of \(l_0\) was relatively large. For example, for \emph{Sphagnum fuscum} average \(l_0\) estimates ranged from 3 to 18 mass-\% (3 to 19 mass-\% with data from Bengtsson et al. (\protect\hyperlink{ref-Bengtsson.2017}{2017})), with largest values for data from Thormann et al. (\protect\hyperlink{ref-Thormann.2001}{2001}), the study already mentioned in the introduction as support for the existence of large initial leaching losses, and from Asada and Warner (\protect\hyperlink{ref-Asada.2005b}{2005}). This is similar to the range of initial leaching losses estimated across all species.

Small average initial leaching losses (\textless{} 5\%) were estimated for \emph{Sphagnum} spec., either peat from 10 to 30 cm depth (\protect\hyperlink{ref-Prevost.1997}{Prevost et al., 1997}), or hollow and hummock \emph{Sphagna} from the surface (\protect\hyperlink{ref-Bartsch.1985}{Bartsch and Moore, 1985}), for \emph{S. lindbergii} (also from Bartsch and Moore (\protect\hyperlink{ref-Bartsch.1985}{1985})), for \emph{S. fuscum} replicates (incubated in central Sweden (\protect\hyperlink{ref-Breeuwer.2008}{Breeuwer et al., 2008}) or in Western Siberia (\protect\hyperlink{ref-Golovatskaya.2017}{Golovatskaya and Nikonova, 2017})), and for \emph{S. auriculatum} (\protect\hyperlink{ref-Trinder.2008}{Trinder et al., 2008}). Large average \(l_0\) (often \textgreater10 mass-\%) were estimated for \emph{S. angustifolium}, \emph{S. balticum}, \emph{S. fallax}, and \emph{S. russowii}.

Average decomposition rates are in the range 0.01 to 2.09 yr\(^{-1}\) (0.01 to 1.16 yr\(^{-1}\) without data from Bengtsson et al. (\protect\hyperlink{ref-Bengtsson.2017}{2017})). As for initial leaching losses, Fig. \ref{fig:out-mm36-1-p5} (b) indicates some differences between species and studies and the median within-species variance was 0.9 times as large as the between-species variance (log scale). Decomposition rates were particularly small and consistent for \emph{Sphagnum fuscum} (range: 0.01 to 0.06 yr\(^{-1}\)), and small also for peat samples from 10 or 20 cm depth (\protect\hyperlink{ref-Prevost.1997}{Prevost et al., 1997}), and unidentified lawn and hummock mosses (\protect\hyperlink{ref-Bartsch.1985}{Bartsch and Moore, 1985}) (Fig. \ref{fig:out-mm36-1-p5} (b)). Replicates for which the model estimated larger initial leaching losses also had on average larger estimated decomposition rates (supporting information \ref{sup-7}).

Estimates for \(\alpha\), the parameter controlling how fast the decomposition rate decreases over time, were variable, but average values did not differ much between species or studies and were similar to the prior average of \textasciitilde2 (the posterior average \(\alpha\) is 2.56 (2.04, 3.1), 95\% confidence interval), indicating that available litterbag data do not provide much information against or in favor of a decrease in decomposition rate with progressing decomposition if initial leaching losses are considered. An exception are \emph{S. russowii} and \emph{S. capillifolium} litters from Hagemann and Moroni (\protect\hyperlink{ref-Hagemann.2015}{2015}) for which we estimated a larger \(\alpha\), though with large errors (9.34 (5.03, 12.15)).



\begin{figure}[H]

{\centering \includegraphics[width=1\linewidth,angle=0]{figures/leaching_plot_1_4} 

}

\caption{Estimated initial leaching losses (a), the parameter controlling a decrease of decomposition rates over time (\(\alpha\)) (b), and decomposition rates (c) grouped by species and study. Points represent averages and error bars 95\% confidence intervals. The study is indicated by numbers on the x axis: (1) Asada and Warner (\protect\hyperlink{ref-Asada.2005b}{2005}), (2) Bartsch and Moore (\protect\hyperlink{ref-Bartsch.1985}{1985}), (3) Breeuwer et al. (\protect\hyperlink{ref-Breeuwer.2008}{2008}), (4) Golovatskaya and Nikonova (\protect\hyperlink{ref-Golovatskaya.2017}{2017}), (5) Hagemann and Moroni (\protect\hyperlink{ref-Hagemann.2015}{2015}), (6) Johnson and Damman (\protect\hyperlink{ref-Johnson.1991}{1991}), (7) Mäkilä et al. (\protect\hyperlink{ref-Makila.2018}{2018}), (8) Prevost et al. (\protect\hyperlink{ref-Prevost.1997}{1997}), (9) Scheffer et al. (\protect\hyperlink{ref-Scheffer.2001}{2001}), (10) Straková et al. (\protect\hyperlink{ref-Strakova.2010}{2010}), (11) Szumigalski and Bayley (\protect\hyperlink{ref-Szumigalski.1996}{1996}), (12) Thormann et al. (\protect\hyperlink{ref-Thormann.2001}{2001}), (13) Trinder et al. (\protect\hyperlink{ref-Trinder.2008}{2008}), (14) Vitt (\protect\hyperlink{ref-Vitt.1990}{1990}). \emph{Sphagnum} spec. are samples that have been identified only to the genus level.}\label{fig:out-mm36-1-p5}
\end{figure}



\begin{table}[H]

\caption{\label{tab:tab-out-fit-4-species-averages-caption}Averages and 95\% confidence intervals for initial leaching losses (\(l_0\)), decomposition rates (\(k_0\)), and rates at which decomposition rates decrease with increasing mass loss (\(\alpha\)) of \emph{Sphagnum} species for available litterbag studies (without data from Bengtsson et al. (\protect\hyperlink{ref-Bengtsson.2017}{2017})).}
\centering
\resizebox{\linewidth}{!}{
\begin{tabular}[t]{llll}
\toprule
Species & $l_0$ (mass-\%) & $k_0$ (yr$^{-1}$) & $\alpha$ (-)\\
\midrule
$Sphagnum$ spec. & 4.9 (2.2, 9.8) & 0.04 (0.02, 0.07) & 2.9 (1.8, 4.8)\\
$S.~lindbergii$ & 7.9 (2.8, 13.3) & 0.05 (0.02, 0.08) & 2.9 (1.8, 4.7)\\
$S.~fuscum$ & 9.8 (7.8, 12.4) & 0.04 (0.02, 0.06) & 2.8 (1.8, 4.6)\\
$S.~magellanicum~aggr.$ & 10.1 (6.5, 14.4) & 0.05 (0.03, 0.1) & 2.8 (1.8, 4.5)\\
$S.~angustifolium$ & 10.8 (6, 17.7) & 0.13 (0.05, 0.25) & 2.6 (1.8, 4.1)\\
\addlinespace
$S.~teres$ & 10.5 (5.8, 16.7) & 0.05 (0.02, 0.09) & 2.9 (1.8, 4.8)\\
$S.~papillosum$ & 9.2 (5, 13.5) & 0.05 (0.02, 0.08) & 2.8 (1.8, 4.5)\\
$S.~squarrosum$ & 9.5 (4.8, 15) & 0.05 (0.02, 0.08) & 2.9 (1.8, 4.7)\\
$S.~auriculatum$ & 7.5 (1.4, 13.9) & 0.05 (0.01, 0.08) & 2.9 (1.8, 4.7)\\
$S.~balticum$ & 13 (9, 17.7) & 0.05 (0.03, 0.07) & 2.9 (1.8, 4.8)\\
\addlinespace
$S.~fallax$ & 10.7 (5.7, 18) & 0.07 (0.03, 0.14) & 2.8 (1.8, 4.5)\\
$S.~russowii$ & 11.3 (6.1, 19.7) & 0.07 (0.03, 0.16) & 2.8 (1.8, 4.5)\\
$S.~cuspidatum$ & 11.8 (7.1, 17.8) & 0.04 (0.02, 0.07) & 2.9 (1.8, 4.8)\\
$S.~majus$ & 10.1 (5.8, 15.8) & 0.05 (0.02, 0.09) & 2.9 (1.8, 4.7)\\
$S.~rubellum$ & 12.1 (6.8, 20.5) & 0.05 (0.02, 0.09) & 2.9 (1.8, 4.8)\\
\addlinespace
$S.~russowii$ and $capillifolium$ & 10.3 (6.1, 15.3) & 0.37 (0.04, 1.48) & 5.1 (3.3, 7.6)\\
\bottomrule
\end{tabular}}
\end{table}

\hypertarget{out-res-2}{%
\subsection{Ignoring initial leaching losses results in larger estimated decomposition rates}\label{out-res-2}}

Decomposition rates were overestimated in our simulation when initial leaching losses are ignored. The larger the simulated initial leaching losses were, the larger became the bias (Fig. \ref{fig:out-p-simulation-p2} (a)). This indicates that if there actually are initial leaching losses as described by equation \eqref{eq:decomposition-solution-1-with-leaching-1}, but these are not considered, \emph{Sphagnum} \(k_0\) will be overestimated in proportion to the actual initial leaching losses.

This overestimation did result in misfits to the data within the five year period which are similar to misfits of models fitted to real data (Fig. \ref{fig:out-p-simulation-p2} (c)). The minimum difference of simulated and estimated remaining mass is -13\%, the maximum difference is 7\%, which is compatible with the median error in remaining masses of replicates in our synthesized litterbag data, 3.2 mass-\%.

The overestimation of \(k_0\) when ignoring initial leaching losses becomes however important when extrapolating from the typical duration of litterbag studies to longer time ranges. For example, extrapolating the models to 20 or 100 years generally increases the difference between simulated and estimated remaining masses, as shown in Fig. \ref{fig:out-p-simulation-p2} (b). After 100 years with \(k_0 = 0.01\) yr\(^{-1}\) the models that account for initial leaching losses will yield about 30\% more peat stock than those that do not consider initial leaching losses because of the overestimated decomposition rate. For litter with \(k_0 = 0.15\) yr\(^{-1}\), even large \(l_0\) cause only a small bias because overall mass loss is dominated by decomposition. However, for example for litter with \(k_0<0.05\) yr\(^{-1}\), predicted average masses would be 5 to 30\% smaller than if initial leaching losses had been considered. The differences are therefore not negligible any more for predictions of peatland models, particularly for peat decomposing at smaller rates and --- if \(k_0<0.05\) yr\(^{-1}\) --- even if \(l_0<5\)\%.



\begin{figure}[H]

{\centering \includegraphics[width=0.7\linewidth]{figures/leaching_plot_simulation_p2} 

}

\caption{Results of the simulation experiment. (a) Estimated divided by simulated decomposition rates (\(k_0\)) versus simulated initial leaching losses (\(l_0\)) for the three simulated decomposition rates. Error bars are standard errors. (b) Remaining masses predicted by the model ignoring initial leaching losses minus remaining masses with the simulation model, either after 5, 20, or 100 years of decomposition. Positive values mean that with \(k_0\) estimated while ignoring initial leaching losses remaining masses are underestimated. (c) Simulated remaining masses versus remaining masses predicted by the model ignoring initial leaching losses for the three simulated decomposition rates and the simulated litterbag retrieval times.}\label{fig:out-p-simulation-p2}
\end{figure}

The analysis of the synthesized litterbag data reproduces both patterns we have observed in the simulation: First, average \(k_0\) as estimated by the model ignoring initial leaching losses increased with increasing \(l_0\) (as estimated by the model considering initial leaching losses) (Fig. \ref{fig:out-mm27-1-mm28-1-p2} (a)). On average, \(k_0\) estimates were 1.4 to 9.5-fold larger when initial leaching losses are ignored compared to when initial leaching losses are considered (1.2 to 9.5-fold larger with data from Bengtsson et al. (\protect\hyperlink{ref-Bengtsson.2017}{2017})). Second, the standard deviation of \(k_0\) increased with increasing \(l_0\), even though this is the case for some species also for the model that considered initial leaching (Fig. \ref{fig:out-mm27-1-mm28-1-p2} (b)).

Overall, both our simulation and our analysis of available litterbag data suggest that \(k_0\) will be overestimated and have larger errors when initial leaching losses are ignored.



\begin{figure}[H]

{\centering \includegraphics[width=0.9\linewidth]{figures/leaching_plot_2_fit_6_and_fit_2} 

}

\caption{(a) Decomposition rate estimates, either considering leaching (black) or ignoring leaching (grey) versus average initial leaching losses estimated by the model considering initial leaching losses. Points are average estimates and error bars are 95\% prediction intervals. (b) Standard deviation of decomposition rate estimates, either considering leaching (black) or ignoring leaching (grey) versus average initial leaching losses estimated by the model considering initial leaching losses. Both plots show values for species with at least 5 estimates.}\label{fig:out-mm27-1-mm28-1-p2}
\end{figure}

\hypertarget{out-res-3}{%
\subsection{\texorpdfstring{Sensitivity of \(k_0\) and \(l_0\) to the design of litterbag experiments}{Sensitivity of k\_0 and l\_0 to the design of litterbag experiments}}\label{out-res-3}}

For litterbag experiments with small estimated \(l_0\), \(k_0\) was less sensitive to \(l_0\) if the first litterbags were collected shortly after the start of the litterbag experiment, as expected. In contrast, for litterbag experiments with larger estimated \(l_0\), this relation was less pronounced or apparently absent (Fig. \ref{fig:out-sdm-ua1-p1-p2} (b)). Because average initial leaching losses and decomposition rates are positively related (Pearson correlation coefficients and 95\% confidence interval: 0.16 (0.06, 0.26)), a similar relation can be observed if the data are grouped by the estimated \(k_0\) (Fig. \ref{fig:out-sdm-ua1-p1-p2} (c)), i.e.~for litterbag experiments with small estimated decomposition rates, the sensitivity indices were smaller if the first litterbags were collected shortly after the start of the litterbag experiment and the pattern is less pronounced for larger decomposition rates.

A rough approximation based on Fig. \ref{fig:out-sdm-ua1-p1-p2} (b) indicates that the average sensitivity of decomposition rates to initial leaching losses can be halved if the first litterbags are collected 20 days after the start of the incubation instead of after a year, if initial leaching losses are smaller than approximately 9\% and decomposition rates smaller than approximately 0.07 yr\(^{-1}\).



\begin{figure}[H]

{\centering \includegraphics[width=0.7\linewidth]{figures/leaching_plot_3} 

}

\caption{Sensitivity indices for decomposition rates conditional on initial leaching losses with all data except from Bengtsson et al. (\protect\hyperlink{ref-Bengtsson.2017}{2017}). (a) Histogram of the sensitivity indices. (b) Sensitivity indices versus the duration after the start of the litterbag experiment after which the first litterbags were retrieved for three groups of initial leaching losses. Panel titles are initial leaching losses in mass-\% (c) Same as (b), but for three groups of decomposition rates. Panel titles are decomposition rates in yr\(^{-1}\). In (b) and (c) the line is a regression line fitted to the data and the shaded area is the 95\% confidence interval. ``slope'' is the slope of the regression line given as average with the lower and upper limit of the 95\% confidence interval (yr\(^{-1}\)).}\label{fig:out-sdm-ua1-p1-p2}
\end{figure}

\hypertarget{discussion}{%
\section{Discussion}\label{discussion}}

We have estimated initial leaching losses and decomposition rates of \emph{Sphagnum} from available litterbag data and results indicate that initial leaching losses are not small in general and large enough to bias predictions of peat accumulation rates over longer time periods. Our sensitivity analysis indicates that our estimates for \(l_0\) are conservative for available litterbag data and the risk that we have overestimated them is low (supporting information \ref{sup-11}). We can thus build on our estimates to discuss the following three points.\\
First, we discuss which factors may have caused small initial leaching losses in previous studies and in litterbag experiments where we estimated small initial leaching losses. If we can identify factors which cause small initial leaching losses, we may in turn explain under what conditions there are larger initial leaching losses. It would also allow us to assess whether initial leaching losses estimated from litterbag experiments are representative for those under natural conditions. Next, we discuss the consequences of ignoring initial leaching losses for decomposition rate estimates, but also for studies which do not estimate decomposition rates and instead simply interpret mass loss differences between experimental groups as decomposition. Finally, we make suggestions how to design litterbag experiments to improve estimates of \(k_0\) and \(l_0\).

\hypertarget{out-discussion-2}{%
\subsection{Possible causes of variations in initial leaching losses between studies}\label{out-discussion-2}}

We suggest that small initial leaching losses (\(<5\) mass-\%) in many of the studies that found small initial leaching losses can be explained by four factors, of which the first three (litter has already been pre-leached, mild drying, and little water volume or water movement) indeed cause small initial leaching losses, and the fourth (underestimated mass losses due to influx of external matter during the incubation) is a measurement artifact.

The following studies quantified or reported small initial leaching losses (\textless5 mass-\%): Coulson and Butterfield (\protect\hyperlink{ref-Coulson.1978}{1978}); Thormann et al. (\protect\hyperlink{ref-Thormann.2002}{2002}), and Castells et al. (\protect\hyperlink{ref-Castells.2005}{2005}) have directly quantified or reported small initial leaching losses without litterbag experiments. In Moore et al. (\protect\hyperlink{ref-Moore.2007}{2007}), some litterbag samples, but not all, have small estimated \(l_0\) (supporting information \ref{sup-1}). Our synthesis adds to this small \(l_0\) estimates for replicates from Prevost et al. (\protect\hyperlink{ref-Prevost.1997}{1997}) (peat), from Bartsch and Moore (\protect\hyperlink{ref-Bartsch.1985}{1985}) (hollow and hummock mosses and \emph{S. lindbergii}), from Breeuwer et al. (\protect\hyperlink{ref-Breeuwer.2008}{2008}) (\emph{S. fuscum} from northern Sweden incubated in central Sweden), from Golovatskaya and Nikonova (\protect\hyperlink{ref-Golovatskaya.2017}{2017}) (\emph{S. fuscum} incubated in Western Siberia), and from Trinder et al. (\protect\hyperlink{ref-Trinder.2008}{2008}) (\emph{S. auriculatum}) (Fig. \ref{fig:out-mm36-1-p5}). In the following paragraphs we suggest what caused small initial leaching losses in these studies.

\hypertarget{litter-has-been-pre-leached-or-already-decomposed}{%
\paragraph*{Litter has been pre-leached or already decomposed}\label{litter-has-been-pre-leached-or-already-decomposed}}
\addcontentsline{toc}{paragraph}{Litter has been pre-leached or already decomposed}

Prevost et al. (\protect\hyperlink{ref-Prevost.1997}{1997}) used peat samples from depths of 10 to 30 cm as litterbag material. This material probably has already experienced decomposition and lost the cytoplasm contents and therefore no large initial leaching losses are observed. Moore and Dalva (\protect\hyperlink{ref-Moore.2001}{2001}) have quantified larger net DOC losses from fresh oven-dried \emph{Sphagnum} litter and \emph{Sphagnum} peat than from more decomposed peat.

\hypertarget{litter-has-been-dried-only-mildly-so-that-sphagnum-plants-do-not-die-completely}{%
\paragraph*{\texorpdfstring{Litter has been dried only mildly so that \emph{Sphagnum} plants do not die (completely)}{Litter has been dried only mildly so that Sphagnum plants do not die (completely)}}\label{litter-has-been-dried-only-mildly-so-that-sphagnum-plants-do-not-die-completely}}
\addcontentsline{toc}{paragraph}{Litter has been dried only mildly so that \emph{Sphagnum} plants do not die (completely)}

Castells et al. (\protect\hyperlink{ref-Castells.2005}{2005}) used fresh \emph{Sphagnum} plants in their study where they quantified only small initial leaching losses. Bartsch and Moore (\protect\hyperlink{ref-Bartsch.1985}{1985}) air-dried their samples for only 24 to 48 h, Schipperges and Rydin (\protect\hyperlink{ref-Schipperges.1998}{1998}) (Fig. 2 and 3) have shown that \emph{Sphagna} can survive drying for several hours if the water content does not decrease too much. Therefore, the \emph{Sphagnum} plants may have not been completely dead which reduces initial leaching losses.

\hypertarget{the-incubation-environment-is-closed-with-small-volume-and-little-water-movmement}{%
\paragraph*{The incubation environment is closed, with small volume and little water movmement}\label{the-incubation-environment-is-closed-with-small-volume-and-little-water-movmement}}
\addcontentsline{toc}{paragraph}{The incubation environment is closed, with small volume and little water movmement}

Thormann et al. (\protect\hyperlink{ref-Thormann.2002}{2002}) incubated \emph{S. fuscum} in petri dishes in the laboratory. We suggest that initial leaching losses were small because leachates could not be exported, there was little water movement, and the volume of the petri dishes was small. Similarly, Golovatskaya and Nikonova (\protect\hyperlink{ref-Golovatskaya.2017}{2017}) started their experiment in September and we assume that the peat was either already partly frozen at this time or that cold temperatures limited leaching (\protect\hyperlink{ref-Lind.2022}{Lind et al., 2022}). This is supported by small initial leaching losses and large and rapid mass losses during spring from \emph{S. angustifolium} samples incubated in the same study during the same period (see Fig. 2 in Golovatskaya and Nikonova (\protect\hyperlink{ref-Golovatskaya.2017}{2017})).

\hypertarget{measurement-artifact-not-properly-subtracting-mass-influx-from-remaining-masses}{%
\paragraph*{Measurement artifact: Not properly subtracting mass influx from remaining masses}\label{measurement-artifact-not-properly-subtracting-mass-influx-from-remaining-masses}}
\addcontentsline{toc}{paragraph}{Measurement artifact: Not properly subtracting mass influx from remaining masses}

Trinder et al. (\protect\hyperlink{ref-Trinder.2008}{2008}) used oven-dried \emph{Sphagnum} samples where we would expect larger initial leaching losses than indicated by our model. However, Trinder et al. (\protect\hyperlink{ref-Trinder.2008}{2008}) report that there was mass influx from the peat matrix (as supported by recorded remaining masses larger than 100 mass-\%) and that they tried to correct this by estimating the amount of peat matrix influx from replicates at the end of the decomposition experiment and assuming a linear influx over time. This procedure does not seem to be robust because many of the corrected remaining masses still are larger than 100 mass-\%. Consequently, not properly subtracted mass influxes are a plausible explanation for apparently small initial leaching losses (and probably also decomposition rates) in this case.

\hypertarget{possible-counterexamples}{%
\paragraph*{Possible counterexamples}\label{possible-counterexamples}}
\addcontentsline{toc}{paragraph}{Possible counterexamples}

The four factors can explain small initial leaching losses in many litterbag experiments and studies directly measuring leaching we are aware of, except for one \emph{Sphagnum} replicate from Breeuwer et al. (\protect\hyperlink{ref-Breeuwer.2008}{2008}), some replicates in Moore et al. (\protect\hyperlink{ref-Moore.2007}{2007}), and direct leaching loss measurements in Coulson and Butterfield (\protect\hyperlink{ref-Coulson.1978}{1978}).

Both a lack of knowledge about the controls of the initial leaching and a lack of information in the studies makes it difficult to explain small initial leaching losses in these studies.\\
Samples from Breeuwer et al. (\protect\hyperlink{ref-Breeuwer.2008}{2008}) were not yet decomposed \emph{S. fuscum} stems which were oven-dried at 30°C for 48h and incubated in Sweden in the field starting in spring, making it unlikely that one of the first three factors is responsible for the small initial leaching losses. Breeuwer et al. (\protect\hyperlink{ref-Breeuwer.2008}{2008}) mention no external mass influx into litterbags (except ingrown roots which were removed), but for some replicates, the remaining masses increased over time (Fig. 3 in Breeuwer et al. (\protect\hyperlink{ref-Breeuwer.2008}{2008})), indicating that measurement artifacts may have played a role here, too.\\
In Moore et al. (\protect\hyperlink{ref-Moore.2007}{2007}), senesced \emph{Sphagnum} samples were air dried, but it is not described what properties of the samples indicated senescence or how long they were dried for. Estimated initial leaching losses were larger than 5 mass-\% for some samples but particularly small if incubated in a pond, suggesting that the incubation environment may have caused small initial leaching losses in some cases if there were no measurement artifacts.\\
Coulson and Butterfield (\protect\hyperlink{ref-Coulson.1978}{1978}) used air- or oven-dried complete shoots of \emph{S. recurvum} and measured initial leaching losses in the laboratory by placing litter in water-filled containers over 7 days. This study reported leaching losses of 0.0 mass-\%, which deviates extremely compared to other studies where initial leaching losses were directly measured, even over much shorter durations (\protect\hyperlink{ref-Moore.2001}{Moore and Dalva, 2001}; \protect\hyperlink{ref-Castells.2005}{Castells et al., 2005}; \protect\hyperlink{ref-Mastny.2018}{Mastný et al., 2018}). The samples were collected in spring and if contents of water-extractable compounds are smaller in spring (\protect\hyperlink{ref-Sytiuk.2023}{Sytiuk et al., 2023}), this may explain small leaching losses, but still not zero leaching.

Our suggestions here are incomplete and there are many potential confounding factors which appear to have received little attention in litterbag experiments. Available litterbag data do not allow to analyze whether there is a seasonal pattern of initial leaching losses as can be expected based on studies analyzing contents of water extractable organic matter (\protect\hyperlink{ref-Edwards.2018}{Edwards et al., 2018}; \protect\hyperlink{ref-Sytiuk.2023}{Sytiuk et al., 2023}) or whether initial leaching losses differ between studies which discard capitula, which use whole plants, or which use stem parts of different length, as can be expected from previous studies and the observation that already senesced or decomposed \emph{Sphagnum} litter has smaller initial leaching losses (\protect\hyperlink{ref-Moore.2001}{Moore and Dalva, 2001}). Ssystematic experiments are necessary to test the suggested causes for small initial leaching losses and potential confounding factors.

To summarize, small initial leaching losses estimated in many existing studies appear to be linked to at least four factors (pre-leaching, only mild drying such that the \emph{Sphagnum} plants do not die, closed incubation environments with small volume and little water movement, measurement artifacts). Conversely, even only air drying can cause large (\(>5\) mass-\%) initial leaching losses, as has been observed for non-\emph{Sphagnum} litter (e.g., Bärlocher (\protect\hyperlink{ref-Barlocher.1997}{1997})). Since many \emph{Sphagnum} litterbag studies oven-dry or air-dry their samples and such procedures are poorly standardized, this could explain some part of the large inter-study variation in initial leaching losses we observed in available litterbag data.











\hypertarget{out-discussion-3}{%
\subsection{Relevance of considering leaching losses in litterbag experiments}\label{out-discussion-3}}

If initial leaching losses are small only under very specific conditions as suggested in the previous section, but not in general, our results suggest that ignoring initial leaching losses can bias decomposition rate estimates. We discuss four reasons why \emph{Sphagnum} litterbag studies should consider initial leaching losses.

\hypertarget{ignoring-initial-leaching-losses-leads-to-overestimated-decomposition-rates}{%
\paragraph*{Ignoring initial leaching losses leads to overestimated decomposition rates}\label{ignoring-initial-leaching-losses-leads-to-overestimated-decomposition-rates}}
\addcontentsline{toc}{paragraph}{Ignoring initial leaching losses leads to overestimated decomposition rates}

First, our simulation suggests that ignoring initial leaching losses leads to overestimation of decomposition rates and that this is not negligible even for leaching losses \(<5\) mass-\% if the decomposition rates are small and if decomposition is extrapolated to longer durations (e.g.~20 years), as is the case in peatland models. That this risk is real can be inferred from the overview of published leaching losses given in the introduction and from our analysis of available litterbag data which indicates that average initial leaching losses range from 3 to 18 mass-\% in past litterbag studies under natural conditions and that leaching losses \(>5\) mass-\% may not be uncommon (in laboratory studies (\protect\hyperlink{ref-Bengtsson.2017}{Bengtsson et al., 2017}) initial leaching losses can be larger, up to 33 mass-\%). Thus, ignoring initial leaching losses can bias decomposition rate estimates.

\hypertarget{ignoring-initial-leaching-losses-can-bias-differences-between-experimental-groups}{%
\paragraph*{Ignoring initial leaching losses can bias differences between experimental groups}\label{ignoring-initial-leaching-losses-can-bias-differences-between-experimental-groups}}
\addcontentsline{toc}{paragraph}{Ignoring initial leaching losses can bias differences between experimental groups}

Second, available litterbag data indicate that initial leaching losses differ between studies (Fig. \ref{fig:out-mm36-1-p5} (a)). Some of the differences between different studies can be explained by differences in litter pre-treatment or experimental setup as described in the previous section. This indicates that results from different litterbag studies cannot be compared directly if the aim is to understand decomposition of the polymer fractions of litter. Moreover, if initial leaching losses differ between two experimental groups within the same experiment --- for example because decomposition of samples under different moisture conditions is compared (\protect\hyperlink{ref-Lind.2022}{Lind et al., 2022}) --- but all mass loss is interpreted as decomposition, this can bias results within the same study.\\
Thus, relative differences in decomposition rates or mass losses due to decomposition may not in general be preserved between different experimental groups in the same study if initial leaching losses are not the same and this can lead to erroneous conclusions on decomposition in peatlands, even if only remaining masses are compared.

\hypertarget{better-knowledge-of-initial-leaching-losses-allows-to-more-accurately-estimate-decomposition-rates}{%
\paragraph*{Better knowledge of initial leaching losses allows to more accurately estimate decomposition rates}\label{better-knowledge-of-initial-leaching-losses-allows-to-more-accurately-estimate-decomposition-rates}}
\addcontentsline{toc}{paragraph}{Better knowledge of initial leaching losses allows to more accurately estimate decomposition rates}

Third, our analysis suggests that errors in all model parameters --- \(l_0\), \(k_0\), and \(\alpha\) --- are large and that the errors in \(k_0\) are sensitive to initial leaching losses (and vice versa). This indicates that a more accurate estimation of initial leaching losses allows to more accurately estimate decomposition rates. Since small differences in decomposition rates can cause larger differences in accumulated C over time (Fig. \ref{fig:out-p-simulation-p2}), this increased accuracy is necessary for more accurate long-term predictions of peatland models.

\hypertarget{does-sphagnum-litter-pre-treatment-change-decomposition-qualitatively-and-alter-microbial-colonization-patterns}{%
\paragraph*{\texorpdfstring{Does \emph{Sphagnum} litter pre-treatment change decomposition qualitatively and alter microbial colonization patterns?}{Does Sphagnum litter pre-treatment change decomposition qualitatively and alter microbial colonization patterns?}}\label{does-sphagnum-litter-pre-treatment-change-decomposition-qualitatively-and-alter-microbial-colonization-patterns}}
\addcontentsline{toc}{paragraph}{Does \emph{Sphagnum} litter pre-treatment change decomposition qualitatively and alter microbial colonization patterns?}

If \emph{Sphagnum} mosses leach under natural conditions much less of their initial mass and over a longer time range, or if they leach compounds inhibiting or facilitating decomposition at different proportions (e.g.~phenolics, sphagnan (\protect\hyperlink{ref-Fenner.2011}{Fenner and Freeman, 2011}; \protect\hyperlink{ref-Hajek.2011}{Hájek et al., 2011}; \protect\hyperlink{ref-Hajek.2024}{Hájek and Urbanová, 2024}) or nutrients), this may change how microbials colonize and decompose litter, possibly making \emph{Sphagnum} litterbag experiments unrepresentative for decomposition under natural conditions; this has already been discussed for non-\emph{Sphagnum} litter (\protect\hyperlink{ref-Barlocher.1997}{Bärlocher, 1997}). Future studies should test whether not drying \emph{Sphagnum} litter decreases initial leaching losses and what consequences this may have on microbial colonization patterns and decomposition rates.

\hypertarget{out-discussion-5}{%
\subsection{How can we improve litterbag experiments?}\label{out-discussion-5}}

The design of litterbag experiments, and specifically when the first litterbags after the start of the experiment is sampled, is an important contributor to the relative large errors in \(k_0\) and \(l_0\) estimated from available litterbag experiments.

The error analysis indicates that when the first litterbag is collected one year after the start of the experiment, errors and biases in average \(k_0\) estimates are larger if the decomposition rate is larger than approximately 0.05 yr\(^{-1}\) and if there are large initial leaching losses. Similarly, our sensitivity analysis suggests that it is difficult to accurately estimate both \(l_0\) and \(k_0\) if the first litterbag was collected a longer time after the start of the experiment (supporting information \ref{sup-11}). In these cases, the mass loss until collection of the first litterbags may be explained either by a large initial leaching loss or a larger initial decomposition rate which slows down over time, as mentioned in section \ref{out-methods-1} (Fig. \ref{fig:out-leaching-sim-interpretation-alpha}).

For available experiments, the first litterbag was collected one year after the start of the incubation points only in 52 out of 129 cases. 22 litterbag experiments collected the first litterbags within 60 days after the start of the experiment and only one within 20 days after the start of the experiment. This indicates that the design of available litterbag experiments is an important contributor to the errors in \(k_0\) and \(l_0\) estimates and that experiments where the first litterbags were collected within approximately 20 days after the start of the experiment or where the true decomposition rate is small, can be expected to provide the most accurate estimates for \(l_0\) and \(k_0\).

Based on this, we make the following suggestions for the design of litterbag experiments:

\begin{enumerate}
\def\labelenumi{\arabic{enumi}.}
\item
  One batch of litterbags should be collected shortly after the begin of the experiment (for example after two days or a week). The mass loss measured for this batch should be a good estimate of initial leaching losses, whereas subsequent mass losses are mass losses attributable to decomposition (including all subsequent leaching losses).
\item
  Environmental conditions which are expected to postpone initial leaching losses (e.g.~due to freezing) should be avoided when possible. If this is not possible, extra batches should ideally be samples directly before and after the initial leaching process took place.
\item
  Even though we have not explicitly tested the effect of increasing the temperal resolution by more frequently retrieving litterbags during an experiment, we expect that this is another step to estimate \(k_0\) and \(l_0\) more accurately and which does not require the development of novel methods.
\end{enumerate}

Additional information that should be provided to correctly interpret litterbag experiments are the date when litter to use in an experiment was collected in the field (to allow future studies to evaluate possible influences of seasonal variations in concentrations of soluble compounds), whether the litter collected in the field was already dry (e.g.~as water content measurements), and how intensely litter was dried (e.g.~drying temperature and residual water content).

\hypertarget{conclusions}{%
\section{Conclusions}\label{conclusions}}

Simulations, estimated initial leaching losses from 15 litterbag studies, and error analysis suggest that decomposition rates are overestimated if initial leaching losses are ignored. With average initial leaching loss magnitudes as reported in previous studies and as estimated here (3 to 18 mass-\%), this implies an overestimation of remaining masses up to several tens of percent during decades of decomposition. A correct estimation of decomposition rates thus requires to explicitly estimate initial leaching losses. This increases the accuracy of predicted peat accumulation from long-term peatland models and allows to better test them.

If initial leaching losses are considered when evaluating available litterbag data, parameter errors are large also due to the experimental design. Based on our error analysis, we suggest that future \emph{Sphagnum} litterbag experiments should sample a batch of litterbags few days to weeks after the start of the experiment because this allows a more accurate estimation of both initial leaching losses and decomposition rates. This applies particularly to experiments in which decomposition rates are small.

Our estimates indicate that initial leaching losses \(>5\) mass-\% are not uncommon and vary as much within species as overall, somewhat contradictory to most results of many previous studies measuring small initial leaching losses from \emph{Sphagnum}. This may be explained by pre-treatment of litter --- even only air-drying --- which may increase initial leaching losses compared to fresh \emph{Sphagnum} and may cause large inter-study variation in initial leaching losses for the same species, similar to what has been observed for leaves from trees. This would imply that a sensible comparison of mass losses from \emph{Sphagnum} litterbag experiments is possible only when initial leaching losses are estimated. Even more, if pre-treatment controls differential leaching of inhibiting or facilitating compounds, this may make litterbag experiments with large initial leaching losses causes by pre-treatment unrepresentative for decomposition under natural conditions.

\hypertarget{acknowledgements}{%
\section*{Acknowledgements}\label{acknowledgements}}
\addcontentsline{toc}{section}{Acknowledgements}

This study was funded by the Deutsche Forschungsgemeinschaft (DFG, German Research Foundation) grant no. KN 929/23-1 to Klaus-Holger Knorr and grant no. PE 1632/18-1 to Edzer Pebesma.

\hypertarget{author-contributions}{%
\section*{Author contributions}\label{author-contributions}}
\addcontentsline{toc}{section}{Author contributions}

HT: Conceptualization, methodology, software, validation, formal analysis, investigation, data curation, writing - original draft, visualization, project administration. EP: supervision, funding acquisition, writing - review \& editing. KHK: supervision, funding acquisition, writing - review \& editing.

\hypertarget{data-and-code-availability}{%
\section*{Data and code availability}\label{data-and-code-availability}}
\addcontentsline{toc}{section}{Data and code availability}

Data and code to reproduce this manuscript are available from \url{https://github.com/henningte/eb1125}.

\hypertarget{references}{%
\section*{References}\label{references}}
\addcontentsline{toc}{section}{References}

\hypertarget{refs}{}
\begin{CSLReferences}{0}{0}
\leavevmode\vadjust pre{\hypertarget{ref-Asada.2005b}{}}%
Asada, T. and Warner, B. G.: Surface {Peat Mass} and {Carbon Balance} in a {Hypermaritime Peatland}, Soil Science Society of America Journal, 69, 549--562, \url{https://doi.org/10.2136/sssaj2005.0549}, 2005.

\leavevmode\vadjust pre{\hypertarget{ref-Barlocher.1997}{}}%
Bärlocher, F.: Pitfalls of traditional techniques when studying decomposition of vascular plant remains in aquatic habitats, Limnetica, 13, 1--11, 1997.

\leavevmode\vadjust pre{\hypertarget{ref-Bartsch.1985}{}}%
Bartsch, I. and Moore, T. R.: A preliminary investigation of primary production and decomposition in four peatlands near {Schefferville}, {Qu{é}bec}, Canadian Journal of Botany, 63, 1241--1248, \url{https://doi.org/10.1139/b85-171}, 1985.

\leavevmode\vadjust pre{\hypertarget{ref-Bauer.2004}{}}%
Bauer, I. E.: Modelling effects of litter quality and environment on peat accumulation over different time-scales: {Peat} accumulation over different time-scales, Journal of Ecology, 92, 661--674, \url{https://doi.org/10.1111/j.0022-0477.2004.00905.x}, 2004.

\leavevmode\vadjust pre{\hypertarget{ref-Bengtsson.2017}{}}%
Bengtsson, F., Granath, G., and Rydin, H.: Data from: {Photosynthesis}, growth, and decay traits in {\emph{Sphagnum}} -- a multispecies comparison, 83493 bytes, \url{https://doi.org/10.5061/DRYAD.62054}, 2017.

\leavevmode\vadjust pre{\hypertarget{ref-Bona.2020}{}}%
Bona, K. A., Shaw, C., Thompson, D. K., Hararuk, O., Webster, K., Zhang, G., Voicu, M., and Kurz, W. A.: The {Canadian} model for peatlands ({CaMP}): {A} peatland carbon model for national greenhouse gas reporting, Ecological Modelling, 431, 109164, \url{https://doi.org/10.1016/j.ecolmodel.2020.109164}, 2020.

\leavevmode\vadjust pre{\hypertarget{ref-Breeuwer.2008}{}}%
Breeuwer, A., Heijmans, M., Robroek, B. J. M., Limpens, J., and Berendse, F.: The effect of increased temperature and nitrogen deposition on decomposition in bogs, Oikos, 117, 1258--1268, \url{https://doi.org/10.1111/j.0030-1299.2008.16518.x}, 2008.

\leavevmode\vadjust pre{\hypertarget{ref-Castells.2005}{}}%
Castells, E., Peñuelas, J., and Valentine, D. W.: Effects of {Plant Leachates} from {Four Boreal Understorey Species} on {Soil N Mineralization}, and {White Spruce} ({Picea} glauca) {Germination} and {Seedling Growth}, Annals of Botany, 95, 1247--1252, \url{https://doi.org/10.1093/aob/mci139}, 2005.

\leavevmode\vadjust pre{\hypertarget{ref-Chaudhary.2018}{}}%
Chaudhary, N., Miller, P. A., and Smith, B.: Biotic and abiotic drivers of peatland growth and microtopography: {A} model demonstration, Ecosystems, 21, 1196--1214, \url{https://doi.org/10.1007/s10021-017-0213-1}, 2018.

\leavevmode\vadjust pre{\hypertarget{ref-Coulson.1978}{}}%
Coulson, J. C. and Butterfield, J.: An investigation of the biotic factors determining the rates of plant decomposition on blanket bog, The Journal of Ecology, 66, 631, \url{https://doi.org/10.2307/2259155}, 1978.

\leavevmode\vadjust pre{\hypertarget{ref-DelGiudice.2017}{}}%
Del Giudice, R. and Lindo, Z.: Short-term leaching dynamics of three peatland plant species reveals how shifts in plant communities may affect decomposition processes, Geoderma, 285, 110--116, \url{https://doi.org/10.1016/j.geoderma.2016.09.028}, 2017.

\leavevmode\vadjust pre{\hypertarget{ref-Edwards.2018}{}}%
Edwards, K. R., Kaštovská, E., Borovec, J., Šantrůčková, H., and Picek, T.: Species effects and seasonal trends on plant efflux quantity and quality in a spruce swamp forest, Plant and Soil, 426, 179--196, \url{https://doi.org/10.1007/s11104-018-3610-0}, 2018.

\leavevmode\vadjust pre{\hypertarget{ref-Eriksson.2019}{}}%
Eriksson, O., Jauhiainen, A., Maad Sasane, S., Kramer, A., Nair, A. G., Sartorius, C., and Hellgren Kotaleski, J.: Uncertainty quantification, propagation and characterization by {Bayesian} analysis combined with global sensitivity analysis applied to dynamical intracellular pathway models, Bioinformatics, 35, 284--292, \url{https://doi.org/10.1093/bioinformatics/bty607}, 2019.

\leavevmode\vadjust pre{\hypertarget{ref-Fenner.2011}{}}%
Fenner, N. and Freeman, C.: Drought-induced carbon loss in peatlands, Nature Geoscience, 4, 895--900, \url{https://doi.org/10.1038/ngeo1323}, 2011.

\leavevmode\vadjust pre{\hypertarget{ref-Frolking.2001}{}}%
Frolking, S., Roulet, N. T., Moore, T. R., Richard, P. J. H., Lavoie, M., and Muller, S. D.: Modeling northern peatland decomposition and peat accumulation, Ecosystems, 4, 479--498, \url{https://doi.org/10.1007/s10021-001-0105-1}, 2001.

\leavevmode\vadjust pre{\hypertarget{ref-Frolking.2010}{}}%
Frolking, S., Roulet, N. T., Tuittila, E., Bubier, J. L., Quillet, A., Talbot, J., and Richard, P. J. H.: A new model of {Holocene} peatland net primary production, decomposition, water balance, and peat accumulation, Earth System Dynamics, 1, 1--21, \url{https://doi.org/10.5194/esd-1-1-2010}, 2010.

\leavevmode\vadjust pre{\hypertarget{ref-Gelman.2014}{}}%
Gelman, A., Carlin, J., Stern, H., Dunson, D., Vehtari, A., and Rubin, D. B.: Bayesian data analysis, Third edition., CRC Press, Boca Raton, 2014.

\leavevmode\vadjust pre{\hypertarget{ref-Gessner.1989}{}}%
Gessner, M. O. and Schwoerbel, J.: Leaching kinetics of fresh leaf-litter with implications for the current concept of leaf-processing in streams, Archiv f{ü}r Hydrobiologie, 115, 81--90, 1989.

\leavevmode\vadjust pre{\hypertarget{ref-Golovatskaya.2017}{}}%
Golovatskaya, E. A. and Nikonova, L. G.: The influence of the bog water level on the transformation of sphagnum mosses in peat soils of oligotrophic bogs, Eurasian Soil Science, 50, 580--588, \url{https://doi.org/10.1134/S1064229317030036}, 2017.

\leavevmode\vadjust pre{\hypertarget{ref-Hagemann.2015}{}}%
Hagemann, U. and Moroni, M. T.: Moss and lichen decomposition in old-growth and harvested high-boreal forests estimated using the litterbag and minicontainer methods, Soil Biology and Biochemistry, 87, 10--24, \url{https://doi.org/10.1016/j.soilbio.2015.04.002}, 2015.

\leavevmode\vadjust pre{\hypertarget{ref-Hajek.2024}{}}%
Hájek, T. and Urbanová, Z.: Enzyme adaptation in {\emph{Sphagnum}} peatlands questions the significance of dissolved organic matter in enzyme inhibition, Science of The Total Environment, 911, 168685, \url{https://doi.org/10.1016/j.scitotenv.2023.168685}, 2024.

\leavevmode\vadjust pre{\hypertarget{ref-Hajek.2011}{}}%
Hájek, T., Ballance, S., Limpens, J., Zijlstra, M., and Verhoeven, J. T. A.: Cell-wall polysaccharides play an important role in decay resistance of {\emph{Sphagnum}} and actively depressed decomposition in vitro, Biogeochemistry, 103, 45--57, \url{https://doi.org/10.1007/s10533-010-9444-3}, 2011.

\leavevmode\vadjust pre{\hypertarget{ref-Hassel.2018}{}}%
Hassel, K., Kyrkjeeide, M. O., Yousefi, N., Prestø, T., Stenøien, H. K., Shaw, J. A., and Flatberg, K. I.: {\emph{Sphagnum Divinum}} ({\emph{Sp. Nov.}}) And {\emph{S}}{\emph{. Medium}} {Limpr}. And their relationship to {\emph{S}}{\emph{. Magellanicum}} {Brid}., Journal of Bryology, 40, 197--222, \url{https://doi.org/10.1080/03736687.2018.1474424}, 2018.

\leavevmode\vadjust pre{\hypertarget{ref-Heijmans.2008}{}}%
Heijmans, M. M. P. D., Mauquoy, D., Van Geel, B., and Berendse, F.: Long-term effects of climate change on vegetation and carbon dynamics in peat bogs, Journal of Vegetation Science, 19, 307--320, \url{https://doi.org/10.3170/2008-8-18368}, 2008.

\leavevmode\vadjust pre{\hypertarget{ref-Heinemeyer.2010}{}}%
Heinemeyer, A., Croft, S., Garnett, M. H., Gloor, E., Holden, J., Lomas, M. R., and Ineson, P.: The {MILLENNIA} peat cohort model: {Predicting} past, present and future soil carbon budgets and fluxes under changing climates in peatlands, Climate Research, 45, 207--226, \url{https://doi.org/10.3354/cr00928}, 2010.

\leavevmode\vadjust pre{\hypertarget{ref-Hoffman.2014}{}}%
Hoffman, M. D. and Gelman, A.: The no-{U-turn} sampler: {Adaptively} setting path lengths in hamiltonian monte carlo, Journal of Machine Learning Research, 15, 1593--1623, 2014.

\leavevmode\vadjust pre{\hypertarget{ref-Johnson.1991}{}}%
Johnson, L. C. and Damman, A. W. H.: Species-controlled {\emph{Sphagnum}} decay on a south {Swedish} raised bog, Oikos, 61, 234, \url{https://doi.org/10.2307/3545341}, 1991.

\leavevmode\vadjust pre{\hypertarget{ref-Kim.2014}{}}%
Kim, Y., Ullah, S., Moore, T. R., and Roulet, N. T.: Dissolved organic carbon and total dissolved nitrogen production by boreal soils and litter: The role of flooding, oxygen concentration, and temperature, Biogeochemistry, 118, 35--48, \url{https://doi.org/10.1007/s10533-013-9903-8}, 2014.

\leavevmode\vadjust pre{\hypertarget{ref-Limpens.2003}{}}%
Limpens, J. and Berendse, F.: How litter quality affects mass loss and {N} loss from decomposing {\emph{Sphagnum}}, Oikos, 103, 537--547, \url{https://doi.org/10.1034/j.1600-0706.2003.12707.x}, 2003.

\leavevmode\vadjust pre{\hypertarget{ref-Lind.2022}{}}%
Lind, L., Harbicht, A., Bergman, E., Edwartz, J., and Eckstein, R. L.: Effects of initial leaching for estimates of mass loss and microbial decomposition---{Call} for an increased nuance, Ecology and Evolution, 12, \url{https://doi.org/10.1002/ece3.9118}, 2022.

\leavevmode\vadjust pre{\hypertarget{ref-Makila.2018}{}}%
Mäkilä, M., Säävuori, H., Grundström, A., and Suomi, T.: {\emph{Sphagnum}} decay patterns and bog microtopography in south-eastern {Finland}, Mires and Peat, 1--12, \url{https://doi.org/10.19189/MaP.2017.OMB.283}, 2018.

\leavevmode\vadjust pre{\hypertarget{ref-Mastny.2018}{}}%
Mastný, J., Kaštovská, E., Bárta, J., Chroňáková, A., Borovec, J., Šantrůčková, H., Urbanová, Z., Edwards, K. R., and Picek, T.: Quality of {DOC} produced during litter decomposition of peatland plant dominants, Soil Biology and Biochemistry, 121, 221--230, \url{https://doi.org/10.1016/j.soilbio.2018.03.018}, 2018.

\leavevmode\vadjust pre{\hypertarget{ref-Moore.2001}{}}%
Moore, T. R. and Dalva, M.: Some controls on the release of dissolved organic carbon by plant rissues and soils, Soil Science, 166, 38--47, \url{https://doi.org/10.1097/00010694-200101000-00007}, 2001.

\leavevmode\vadjust pre{\hypertarget{ref-Moore.2007}{}}%
Moore, T. R., Bubier, J. L., and Bledzki, L.: Litter decomposition in temperate peatland ecosystems: {The} effect of substrate and site, Ecosystems, 10, 949--963, \url{https://doi.org/10.1007/s10021-007-9064-5}, 2007.

\leavevmode\vadjust pre{\hypertarget{ref-Morris.2012}{}}%
Morris, P. J., Baird, A. J., and Belyea, L. R.: The {DigiBog} peatland development model 2: Ecohydrological simulations in {2D}, Ecohydrology, 5, 256--268, \url{https://doi.org/10.1002/eco.229}, 2012.

\leavevmode\vadjust pre{\hypertarget{ref-Muller.2023}{}}%
Müller, R., Zahorka, A., Holawe, F., Inselsbacher, E., and Glatzel, S.: Incubation of ombrotrophic bog litter and mixtures of {\emph{Sphagnum}}{\emph{,} }{\emph{Betula}} and {\emph{Calluna}} results in the formation of single litter-specific decomposition patterns, Geoderma, 440, 116702, \url{https://doi.org/10.1016/j.geoderma.2023.116702}, 2023.

\leavevmode\vadjust pre{\hypertarget{ref-Prevost.1997}{}}%
Prevost, M., Belleau, P., and Plamondon, A. P.: Substrate conditions in a treed peatland: {Responses} to drainage, {É}coscience, 4, 543--554, \url{https://doi.org/10.1080/11956860.1997.11682434}, 1997.

\leavevmode\vadjust pre{\hypertarget{ref-Rovira.2010}{}}%
Rovira, P. and Rovira, R.: Fitting litter decomposition datasets to mathematical curves: {Towards} a generalised exponential approach, Geoderma, 155, 329--343, \url{https://doi.org/10.1016/j.geoderma.2009.11.033}, 2010.

\leavevmode\vadjust pre{\hypertarget{ref-Rydin.2013}{}}%
Rydin, H., Jeglum, J. K., and Bennett, K. D.: The biology of peatlands, 2nd ed., Oxford University Press, Oxford, 2013.

\leavevmode\vadjust pre{\hypertarget{ref-Scheffer.2001}{}}%
Scheffer, R. A., Van Logtestijn, R. S. P., and Verhoeven, J. T. A.: Decomposition of {\emph{Carex}} and {\emph{Sphagnum}} litter in two mesotrophic fens differing in dominant plant species, Oikos, 92, 44--54, \url{https://doi.org/10.1034/j.1600-0706.2001.920106.x}, 2001.

\leavevmode\vadjust pre{\hypertarget{ref-Schipperges.1998}{}}%
Schipperges, B. and Rydin, H.: Response of photosynthesis of {\emph{Sphagnum}} species from contrasting microhabitats to tissue water content and repeated desiccation, New Phytologist, 140, 677--684, \url{https://doi.org/10.1046/j.1469-8137.1998.00311.x}, 1998.

\leavevmode\vadjust pre{\hypertarget{ref-StanDevelopmentTeam.2021b}{}}%
Stan Development Team: {RStan}: The {R} interface to {Stan}, 2021a.

\leavevmode\vadjust pre{\hypertarget{ref-StanDevelopmentTeam.2021a}{}}%
Stan Development Team: Stan {Modeling Language Users Guide} and {Reference Manual}, 2021b.

\leavevmode\vadjust pre{\hypertarget{ref-Strakova.2010}{}}%
Straková, P., Anttila, J., Spetz, P., Kitunen, V., Tapanila, T., and Laiho, R.: Litter quality and its response to water level drawdown in boreal peatlands at plant species and community level, Plant and Soil, 335, 501--520, \url{https://doi.org/10.1007/s11104-010-0447-6}, 2010.

\leavevmode\vadjust pre{\hypertarget{ref-Sytiuk.2023}{}}%
Sytiuk, A., Hamard, S., Céréghino, R., Dorrepaal, E., Geissel, H., Küttim, M., Lamentowicz, M., Tuittila, E. S., and Jassey, V. E. J.: Linkages between {\emph{Sphagnum}} metabolites and peatland { \textsc{} }{\textsc{CO}}{\textsc{\textsubscript{2}}}{ \textsc{} } uptake are sensitive to seasonality in warming trends, New Phytologist, 237, 1164--1178, \url{https://doi.org/10.1111/nph.18601}, 2023.

\leavevmode\vadjust pre{\hypertarget{ref-Szumigalski.1996}{}}%
Szumigalski, A. R. and Bayley, S. E.: Decomposition along a bog to rich fen gradient in central {Alberta}, {Canada}, Canadian Journal of Botany, 74, 573--581, \url{https://doi.org/10.1139/b96-073}, 1996.

\leavevmode\vadjust pre{\hypertarget{ref-Taylor.1996}{}}%
Taylor, B. R. and Bärlocher, F.: Variable effects of air-drying on leaching losses from tree leaf litter, Hydrobiologia, 325, 173--182, \url{https://doi.org/10.1007/BF00014982}, 1996.

\leavevmode\vadjust pre{\hypertarget{ref-Teickner.2024c}{}}%
Teickner, H. and Knorr, K.-H.: Peatland {Decomposition Database} (1.0.0), \url{https://doi.org/10.5281/ZENODO.11276065}, 2024.

\leavevmode\vadjust pre{\hypertarget{ref-Thormann.2001}{}}%
Thormann, M. N., Bayley, S. E., and Currah, R. S.: Comparison of decomposition of belowground and aboveground plant litters in peatlands of boreal {Alberta}, {Canada}, Canadian Journal of Botany, 79, 9--22, \url{https://doi.org/10.1139/b00-138}, 2001.

\leavevmode\vadjust pre{\hypertarget{ref-Thormann.2002}{}}%
Thormann, M. N., Currah, R. S., and Bayley, S. E.: The relative ability of fungi from {\emph{Sphagnum}}{ \emph{Fuscum}} to decompose selected carbon substrates, Canadian Journal of Microbiology, 48, 204--211, \url{https://doi.org/10.1139/w02-010}, 2002.

\leavevmode\vadjust pre{\hypertarget{ref-Trinder.2008}{}}%
Trinder, C. J., Johnson, D., and Artz, R. R. E.: Interactions among fungal community structure, litter decomposition and depth of water table in a cutover peatland, FEMS Microbiology Ecology, 64, 433--448, \url{https://doi.org/10.1111/j.1574-6941.2008.00487.x}, 2008.

\leavevmode\vadjust pre{\hypertarget{ref-Turetsky.2008}{}}%
Turetsky, M. R., Crow, S. E., Evans, R. J., Vitt, D. H., and Wieder, R. K.: Trade-offs in resource allocation among moss species control decomposition in boreal peatlands, Journal of Ecology, 96, 1297--1305, \url{https://doi.org/10.1111/j.1365-2745.2008.01438.x}, 2008.

\leavevmode\vadjust pre{\hypertarget{ref-Vitt.1990}{}}%
Vitt, D. H.: Growth and production dynamics of boreal mosses over climatic, chemical and topographic gradients, Botanical Journal of the Linnean Society, 104, 35--59, \url{https://doi.org/10.1111/j.1095-8339.1990.tb02210.x}, 1990.

\leavevmode\vadjust pre{\hypertarget{ref-Yu.2001}{}}%
Yu, Z., Turetsky, M. R., Campbell, I. D., and Vitt, D. H.: Modelling long-term peatland dynamics. {II}. {Processes} and rates as inferred from litter and peat-core data, Ecological Modelling, 145, 159--173, \url{https://doi.org/10.1016/S0304-3800(01)00387-8}, 2001.

\leavevmode\vadjust pre{\hypertarget{ref-Yu.2012}{}}%
Yu, Z. C.: Northern peatland carbon stocks and dynamics: A review, Biogeosciences, 9, 4071--4085, \url{https://doi.org/10.5194/bg-9-4071-2012}, 2012.

\end{CSLReferences}

\end{document}
