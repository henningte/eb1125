% Options for packages loaded elsewhere
\PassOptionsToPackage{unicode}{hyperref}
\PassOptionsToPackage{hyphens}{url}
%
\documentclass[
  12pt,
]{article}
\usepackage{amsmath,amssymb}
\usepackage{lmodern}
\usepackage{iftex}
\ifPDFTeX
  \usepackage[T1]{fontenc}
  \usepackage[utf8]{inputenc}
  \usepackage{textcomp} % provide euro and other symbols
\else % if luatex or xetex
  \usepackage{unicode-math}
  \defaultfontfeatures{Scale=MatchLowercase}
  \defaultfontfeatures[\rmfamily]{Ligatures=TeX,Scale=1}
\fi
% Use upquote if available, for straight quotes in verbatim environments
\IfFileExists{upquote.sty}{\usepackage{upquote}}{}
\IfFileExists{microtype.sty}{% use microtype if available
  \usepackage[]{microtype}
  \UseMicrotypeSet[protrusion]{basicmath} % disable protrusion for tt fonts
}{}
\makeatletter
\@ifundefined{KOMAClassName}{% if non-KOMA class
  \IfFileExists{parskip.sty}{%
    \usepackage{parskip}
  }{% else
    \setlength{\parindent}{0pt}
    \setlength{\parskip}{6pt plus 2pt minus 1pt}}
}{% if KOMA class
  \KOMAoptions{parskip=half}}
\makeatother
\usepackage{xcolor}
\IfFileExists{xurl.sty}{\usepackage{xurl}}{} % add URL line breaks if available
\IfFileExists{bookmark.sty}{\usepackage{bookmark}}{\usepackage{hyperref}}
\hypersetup{
  pdftitle={Supporting information to: A Synthesis of Sphagnum Litterbag Experiments: Initial Leaching Losses Bias Decomposition Rate Estimates},
  pdfauthor={Henning Teickner1,2,; Edzer Pebesma2; Klaus-Holger Knorr1},
  hidelinks,
  pdfcreator={LaTeX via pandoc}}
\urlstyle{same} % disable monospaced font for URLs
\usepackage[margin=1in]{geometry}
\usepackage{longtable,booktabs,array}
\usepackage{calc} % for calculating minipage widths
% Correct order of tables after \paragraph or \subparagraph
\usepackage{etoolbox}
\makeatletter
\patchcmd\longtable{\par}{\if@noskipsec\mbox{}\fi\par}{}{}
\makeatother
% Allow footnotes in longtable head/foot
\IfFileExists{footnotehyper.sty}{\usepackage{footnotehyper}}{\usepackage{footnote}}
\makesavenoteenv{longtable}
\usepackage{graphicx}
\makeatletter
\def\maxwidth{\ifdim\Gin@nat@width>\linewidth\linewidth\else\Gin@nat@width\fi}
\def\maxheight{\ifdim\Gin@nat@height>\textheight\textheight\else\Gin@nat@height\fi}
\makeatother
% Scale images if necessary, so that they will not overflow the page
% margins by default, and it is still possible to overwrite the defaults
% using explicit options in \includegraphics[width, height, ...]{}
\setkeys{Gin}{width=\maxwidth,height=\maxheight,keepaspectratio}
% Set default figure placement to htbp
\makeatletter
\def\fps@figure{htbp}
\makeatother
\setlength{\emergencystretch}{3em} % prevent overfull lines
\providecommand{\tightlist}{%
  \setlength{\itemsep}{0pt}\setlength{\parskip}{0pt}}
\setcounter{secnumdepth}{5}
\newlength{\cslhangindent}
\setlength{\cslhangindent}{1.5em}
\newlength{\csllabelwidth}
\setlength{\csllabelwidth}{3em}
\newlength{\cslentryspacingunit} % times entry-spacing
\setlength{\cslentryspacingunit}{\parskip}
\newenvironment{CSLReferences}[2] % #1 hanging-ident, #2 entry spacing
 {% don't indent paragraphs
  \setlength{\parindent}{0pt}
  % turn on hanging indent if param 1 is 1
  \ifodd #1
  \let\oldpar\par
  \def\par{\hangindent=\cslhangindent\oldpar}
  \fi
  % set entry spacing
  \setlength{\parskip}{#2\cslentryspacingunit}
 }%
 {}
\usepackage{calc}
\newcommand{\CSLBlock}[1]{#1\hfill\break}
\newcommand{\CSLLeftMargin}[1]{\parbox[t]{\csllabelwidth}{#1}}
\newcommand{\CSLRightInline}[1]{\parbox[t]{\linewidth - \csllabelwidth}{#1}\break}
\newcommand{\CSLIndent}[1]{\hspace{\cslhangindent}#1}
\usepackage{float}
\usepackage[version=4]{mhchem}
\usepackage{booktabs}
\usepackage{multirow}
\usepackage{bm}
\usepackage{xr} \externaldocument[main-]{leaching-paper}
\usepackage{lineno}
\linenumbers
\usepackage{tocloft}
\setlength{\cftsecnumwidth}{4.5ex}
\usepackage{booktabs}
\usepackage{longtable}
\usepackage{array}
\usepackage{multirow}
\usepackage{wrapfig}
\usepackage{float}
\usepackage{colortbl}
\usepackage{pdflscape}
\usepackage{tabu}
\usepackage{threeparttable}
\usepackage{threeparttablex}
\usepackage[normalem]{ulem}
\usepackage{makecell}
\usepackage{xcolor}
\ifLuaTeX
  \usepackage{selnolig}  % disable illegal ligatures
\fi

\title{Supporting information to: A Synthesis of \emph{Sphagnum} Litterbag Experiments: Initial Leaching Losses Bias Decomposition Rate Estimates}
\author{Henning Teickner\textsuperscript{1,2,*} \and Edzer Pebesma\textsuperscript{2} \and Klaus-Holger Knorr\textsuperscript{1}}
\date{04 June, 2024}

\begin{document}
\maketitle

{
\setcounter{tocdepth}{2}
\tableofcontents
}
\textsuperscript{1} ILÖK, Ecohydrology \& Biogeochemistry Group, Institute of Landscape Ecology, University of Münster, 48149, Germany\\
\textsuperscript{2} IfGI, Spatiotemporal Modelling Lab, Institute for Geoinformatics, University of Münster, 48149, Germany

\textsuperscript{*} Correspondence: \href{mailto:henning.teickner@uni-muenster.de}{Henning Teickner \textless{}\href{mailto:henning.teickner@uni-muenster.de}{\nolinkurl{henning.teickner@uni-muenster.de}}\textgreater{}}

\renewcommand{\thefigure}{S\arabic{figure}} 
\renewcommand{\thetable}{S\arabic{table}}
\renewcommand{\thesection}{S\arabic{section}}
\renewcommand{\theequation}{S\arabic{equation}}

\hypertarget{sup-1}{%
\section{\texorpdfstring{Initial leaching losses as estimated in Moore et al. (\protect\hyperlink{ref-Moore.2007}{2007})}{Initial leaching losses as estimated in Moore et al. (2007)}}\label{sup-1}}

In Moore et al. (\protect\hyperlink{ref-Moore.2007}{2007}) decomposition rates are estimated from a logarithmic version of the one-pool exponential decomposition model, where the remaining mass at the start is estimated as intercept \(a\):

\[
\ln(m(t)) = a - k_0t\\
\]

Because initial leaching losses happen shortly after the start of the incubation, this intercept is smaller than 100 percent of the initial mass and \(\exp(a)\) is an estimate for initial leaching losses. With data from Tab. 2 in Moore et al. (\protect\hyperlink{ref-Moore.2007}{2007}), initial leaching losses for \emph{S. magellanicum}, \emph{S. fallax}, \emph{S. capillifolium}, and \emph{S. angustifolium} are within the range -1 to 16 percent of the initial mass. Samples in the pond had the lowest initial leaching losses (on average -1 percent of the initial mass) and samples in the fen the largest (on average 3 percent of the initial mass).

\hypertarget{sup-13}{%
\section{Model equations}\label{sup-13}}

Tab. \ref{tab:m-litterbag-synthesis-models} lists the models we computed for this study. Here, we also assign identifiers to the models to make it easier to trace parts of the supporting information back to the specific model used to compute it. In the main text, models 1-2 and 2-2 are used in section \ref{main-methods-bias-real-1}, and model 1-4 in section \ref{main-methods-estimate-real-1}. Models 1-5 and 1-6 were computed for the sensitivity analysis described in section \ref{main-methods-estimate-real-1} in the main text. The other models were computed to analyze the influence of estimating \(\alpha\) from the data and the influence of including or excluding data from Bengtsson et al. (\protect\hyperlink{ref-Bengtsson.2017}{2017}), as described in the main text.











\begin{table}[H]

\caption{\label{tab:m-litterbag-synthesis-models}Overview of models computed in this study on synthesized litterbag data. ``Decomposition equation'' is the equation the models use to describe remaining masses over time for litterbag experiments. Equations for the model components are shown in supporting information \ref{sup-13}.}
\centering
\resizebox{\linewidth}{!}{
\begin{tabular}[t]{lll>{\raggedright\arraybackslash}p{7cm}>{\raggedright\arraybackslash}p{7cm}}
\toprule
Model version & Considers $l_0$? & Decomposition equation & Description & Dataset\\
\midrule
model 1-1 & Yes & \ref{main-eq:decomposition-solution-1-with-leaching-1} & One-pool exponential decomposition model which estimates decomposition rates and initial leaching losses. The model is a hierarchical model and estimates decomposition rates and initial leaching losses for individual litterbag replicates, combinations of species and studies, and species. & Full dataset.\\
model 1-2 & Yes & \ref{main-eq:decomposition-solution-1-with-leaching-1} & Same as model 1-1. & Full dataset excluding data from Bengtsson et al. (\protect\hyperlink{ref-Bengtsson.2017}{2017}).\\
model 1-3 & Yes & \ref{main-eq:decomposition-solution-2-with-leaching-1} & Same as model 1-1, but uses equation \ref{main-eq:decomposition-solution-2-with-leaching-1}. & Full dataset.\\
model 1-4 & Yes & \ref{main-eq:decomposition-solution-2-with-leaching-1} & Same as model 1-3. & Full dataset excluding data from Bengtsson et al. (\protect\hyperlink{ref-Bengtsson.2017}{2017}).\\
model 1-5 & Yes & \ref{main-eq:decomposition-solution-2-with-leaching-1} & Same as model 1-3. & Simulated data.\\
\addlinespace
model 1-6 & Yes & \ref{main-eq:decomposition-solution-2-with-leaching-1} & Same as model 1-3. & Simulated data, created with parameter values sampled from the posterior of model 1-4.\\
model 2-1 & No & \ref{main-eq:decomposition-solution-1-no-leaching-1} & Same as model 1-1, but ignores initial leaching losses. & Full dataset.\\
model 2-2 & No & \ref{main-eq:decomposition-solution-1-no-leaching-1} & Same as model 2-1. & Full dataset, excluding data from Bengtsson et al. (\protect\hyperlink{ref-Bengtsson.2017}{2017}).\\
model 2-3 & No & \ref{main-eq:decomposition-solution-2-no-leaching-1} & Same as model 2-1, but uses equation \ref{main-eq:decomposition-solution-2-no-leaching-1}. & Full dataset.\\
model 2-4 & No & \ref{main-eq:decomposition-solution-2-no-leaching-1} & Same as model 2-3. & Full dataset, excluding data from Bengtsson et al. (\protect\hyperlink{ref-Bengtsson.2017}{2017}).\\
\bottomrule
\end{tabular}}
\end{table}

All models used the following components to model the remaining mass of litterbag replicate \(i\) conditional on the average remaining mass (\(\mu_i\)) , the precision of remaining masses (\(\phi_i\)), and decomposition rates (\(k\_1_i\)):

\begin{equation}
\begin{aligned}
m_i & \sim & \text{Beta}(\mu_i\phi_{i},(1 - \mu_i)\phi_{i})\\
\phi_{i} & = & \begin{cases}
\text{precision}_i & \text{if}~\text{precision}_i~\text{is available}\\
\phi\_1_{i} & \text{otherwise}\\
\end{cases}\\
\phi\_1_{i} & = & \phi\_2_{\text{sample}[i]}\\
\text{precision}_i & \sim & \text{Gamma}\left(\phi\_2\_p1, \frac{\phi\_2\_p1}{\phi\_2\_p2_{\text{sample}}}\right)\\
\phi\_2_{\text{sample}} & \sim & \text{Gamma}\left(\phi\_2\_p1, \frac{\phi\_2\_p1}{\phi\_2\_p2_{\text{sample}}}\right)\\
\phi\_2\_p2_{\text{sample}} & = & \exp(\phi\_2\_p2\_p1 + \phi\_2\_p2\_p2_{\text{species[sample]}} + \\
                                && \phi\_2\_p2\_p3_{\text{species x studies[sample]}})\\
\phi\_2\_p2\_p1 & \sim & \text{Normal}(\phi\_2\_p2\_p1\_p1, \phi\_2\_p2\_p1\_p2)\\
\phi\_2\_p2\_p2_{\text{species}} & \sim & \text{Normal}(\phi\_2\_p2\_p2\_p1, \phi\_2\_p2\_p2\_p2)\\
\phi\_2\_p2\_p3_{\text{species x studies}} & \sim & \text{Normal}(\phi\_2\_p2\_p3\_p1, \phi\_2\_p2\_p3\_p2)\\
\phi\_2\_p1 & \sim & \text{Gamma}(\phi\_2\_p1\_p1, \phi\_2\_p1\_p2)\\
\phi\_2\_p2\_p1\_p2 & \sim & \text{Normal}^+(0, \phi\_2\_p2\_p1\_p2_p1)\\
\phi\_2\_p2\_p2\_p2_{\text{species}} & \sim & \text{Normal}^+(0, \phi\_2\_p2\_p2\_p2\_p1)\\
\phi\_2\_p2\_p3\_p2_{\text{species x studies}} & \sim & \text{Normal}^+(0, \phi\_2\_p2\_p3\_p2\_p1)\\
\phi\_2\_p2\_p4\_p2_{\text{samples}} & \sim & \text{Normal}^+(0, \phi\_2\_p2\_p4\_p2\_p1)\\
k\_1_i & = & k\_2_{\text{sample}[i]}\\
k\_2_{\text{sample}} & = & \text{exp}(k\_2\_p1 + k\_2\_p2_{\text{species[sample]}} + \\
                         && k\_2\_p3_{\text{species x study[sample]}} + \\
                         && k\_2\_p4_{\text{samples}})\\
k\_2\_p1 & \sim & \text{Normal}(k\_2\_p1\_p1, k\_2\_p1\_p2)\\
k\_2\_p2_{\text{species}} & \sim & \text{Normal}(k\_2\_p2\_p1, k\_2\_p2\_p2)\\
k\_2\_p3_{\text{species x studies}} & \sim & \text{Normal}(k\_2\_p3\_p1, k\_2\_p3\_p2)\\
k\_2\_p4_{\text{samples}} & \sim & \text{Normal}(k\_2\_p3\_p1, k\_2\_p3\_p2)\\
k\_2\_p1\_p2 & \sim &  \text{Normal}^+(0, k\_2\_p1\_p2\_p1)\\
k\_2\_p2\_p2_{\text{species}} & \sim &  \text{Normal}^+(0, k\_2\_p2\_p2\_p1)\\
k\_2\_p3\_p2_{\text{species x studies}} & \sim &  \text{Normal}^+(0, k\_2\_p3\_p2\_p1)\\
k\_2\_p4\_p2_{\text{samples}} & \sim &  \text{Normal}^+(0, k\_2\_p4\_p2\_p1)\\
\label{eq:sup-model-1}
\end{aligned}
\end{equation}

Where \(\mu_i\) is the average mass remaining for sample \(i\), \(\sigma_i\) is the reported standard deviation for the mass remaining for sample \(i\), and \(\text{precision}_i = \frac{\mu_i(1 - \mu_i)}{\sigma_i^2} -1\).

The formula for the average remaining mass (\(\mu_i\)) when \(\alpha=1\) and there are no initial leaching losses (models 2-1 and 2-2), according to equation \ref{main-eq:decomposition-solution-1-no-leaching-1} in the main text, are:

\begin{equation}
\begin{aligned}
\mu_i & = & 1 \exp(k t)\\
\label{eq:sup-model-2}
\end{aligned}
\end{equation}

The formula for the average remaining mass (\(\mu_i\)) when \(\alpha=1\) and there are initial leaching losses (models 1-1 and 1-2), according to equation \ref{main-eq:decomposition-solution-1-with-leaching-1} in the main text, are:

\begin{equation}
\begin{aligned}
\mu_i & = & \begin{cases} 
1 & \text{if}~t_i=0\\
(1 - l\_1_i) \exp(k t) & \text{if}~t_i>0
\end{cases}\\
\label{eq:sup-model-3}
\end{aligned}
\end{equation}

The formula for the average remaining mass (\(\mu_i\)) when \(\alpha\) is estimated from the litterbag data and there are no initial leaching losses (models 2-3 and 2-4), according to equation \ref{main-eq:decomposition-solution-2-no-leaching-1} in the main text, are:

\begin{equation}
\begin{aligned}
\mu_i & = & \frac{(1)}{(1 + (\alpha - 1) k t)^{\frac{1}{\alpha - 1}}}\\
\label{eq:sup-model-4}
\end{aligned}
\end{equation}

The formula for the average remaining mass (\(\mu_i\)) when \(\alpha\) is estimated from the litterbag data and there are initial leaching losses (models 1-3, 1-4), according to equation \ref{main-eq:decomposition-solution-2-with-leaching-1} in the main text, are:

\begin{equation}
\begin{aligned}
\mu_i & = & \begin{cases} 
1 & \text{if}~t_i=0\\
\frac{(1 - l\_1_i)}{(1 + (\alpha - 1) k t)^{\frac{1}{\alpha - 1}}} & \text{if}~t_i>0
\end{cases}\\
\label{eq:sup-model-5}
\end{aligned}
\end{equation}

To avoid \(\mu_i = 1\), we subtracted a constant (\ensuremath{10^{-4}}) from \(\mu_i\) when \(\mu_i = 1\).

\(\alpha\) is modeled in the same way as \(\phi\) (models 1-3, 1-4, 2-3, 2-4):

\begin{equation}
\begin{aligned}
\alpha\_1_i & = & \alpha\_2_{\text{sample}[i]}\\
\alpha\_2_{\text{sample}} & = & 1 + \text{exp}(\alpha\_2\_p1 + \alpha\_2\_p2_{\text{species[sample]}} + \\
                         && \alpha\_2\_p3_{\text{species x study[sample]}} + \\
                         && \alpha\_2\_p4_{\text{samples}})\\
\alpha\_2\_p1 & \sim & \text{Normal}(\alpha\_2\_p1\_p1, \alpha\_2\_p1\_p2)\\
\alpha\_2\_p2_{\text{species}} & \sim & \text{Normal}(\alpha\_2\_p2\_p1, \alpha\_2\_p2\_p2)\\
\alpha\_2\_p3_{\text{species x studies}} & \sim & \text{Normal}(\alpha\_2\_p3\_p1, \alpha\_2\_p3\_p2)\\
\alpha\_2\_p4_{\text{samples}} & \sim & \text{Normal}(\alpha\_2\_p3\_p1, \alpha\_2\_p3\_p2)\\
\label{eq:sup-model-6}
\end{aligned}
\end{equation}

Initial leaching losses are modeled in the same way as \(\phi\) (models 1-3, 1-4):

\begin{equation}
\begin{aligned}
l\_1_i & = & l\_2_{\text{sample}[i]}\\
l\_2_{\text{sample}} & = & \text{inv\_logit}(l\_2\_p1 + l\_2\_p2_{\text{species[sample]}} + \\
                        && l\_2\_p3_{\text{species x study[sample]}} + \\                              && l\_2\_p4_{\text{samples}})\\
l\_2\_p1 & \sim & \text{Normal}(l\_2\_p1\_p1, l\_2\_p1\_p2)\\
l\_2\_p2_{\text{species}} & \sim & \text{Normal}(l\_2\_p2\_p1, l\_2\_p2\_p2)\\
l\_2\_p3_{\text{species x study}} & \sim & \text{Normal}(l\_2\_p3\_p1, l\_2\_p3\_p2)\\
l\_2\_p4_{\text{samples}} & \sim & \text{Normal}(l\_2\_p4\_p1, l\_2\_p4\_p2)\\
l\_2\_p1\_p2 & \sim & \text{Normal}^+(0, l\_2\_p1\_p2\_p1)\\
l\_2\_p2\_p2_{\text{species}} & \sim & \text{Normal}^+(0, l\_2\_p2\_p2\_p1)\\
l\_2\_p3\_p2_{\text{species x study}} & \sim & \text{Normal}^+(0, l\_2\_p3\_p2\_p1)\\
l\_2\_p4\_p2_{\text{samples}} & \sim & \text{Normal}^+(0, l\_2\_p4\_p2\_p1)\\
\label{eq:sup-model-7}
\end{aligned}
\end{equation}

\hypertarget{sup-2}{%
\section{\texorpdfstring{Estimates of \(k_0\) and \(\alpha\) from available litterbag data while ignoring initial leaching losses}{Estimates of k\_0 and \textbackslash alpha from available litterbag data while ignoring initial leaching losses}}\label{sup-2}}

In the main text (section \ref{main-out-methods-1}) we mentioned that estimating \(\alpha\) from the litterbag data while ignoring initial leaching losses causes even larger bias of \(k_0\) estimates than when \(\alpha\) is set to \(1\). Here, we present additional analyses to support this claim.

When equation \ref{main-eq:decomposition-solution-2-with-leaching-1} in the main text is used to estimate one-pool decomposition rates from litterbag experiments with large initial mass losses as caused by initial leaching, estimates for \(k_0\) and \(\alpha\) are much larger, indicating that under these conditions parameters intended to describe how decomposition rates decrease over time incorporate the effect of mass losses due to initial leaching. Fig. \ref{fig:sup-out-mm35-1-p5} shows estimates for decomposition rates and \(\alpha\) for the available litterbag data and Fig. \ref{fig:sup-out-mm35-2-p5} shows the same when data from Bengtsson et al. (\protect\hyperlink{ref-Bengtsson.2017}{2017}) are excuded.



\begin{figure}[H]

{\centering \includegraphics[width=1\linewidth]{figures/leaching_plot_1_7} 

}

\caption{Estimated parameter controlling a decrease of decomposition rates over time (\(\alpha\)) (a), and decomposition rates (b) grouped by species and study for model 2-3. Points represent averages and error bars 95\% confidence intervals. The study is indicated by numbers on the x axis: (1) Asada and Warner (\protect\hyperlink{ref-Asada.2005b}{2005}), (2) Bartsch and Moore (\protect\hyperlink{ref-Bartsch.1985}{1985}), (3) Bengtsson et al. (\protect\hyperlink{ref-Bengtsson.2017}{2017}), (4) Breeuwer et al. (\protect\hyperlink{ref-Breeuwer.2008}{2008}), (5) Golovatskaya and Nikonova (\protect\hyperlink{ref-Golovatskaya.2017}{2017}), (6) Hagemann and Moroni (\protect\hyperlink{ref-Hagemann.2015}{2015}), (7) Johnson and Damman (\protect\hyperlink{ref-Johnson.1991}{1991}), (8) Mäkilä et al. (\protect\hyperlink{ref-Makila.2018}{2018}), (9) Prevost et al. (\protect\hyperlink{ref-Prevost.1997}{1997}), (10) Scheffer et al. (\protect\hyperlink{ref-Scheffer.2001}{2001}), (11) Straková et al. (\protect\hyperlink{ref-Strakova.2010}{2010}), (12) Szumigalski and Bayley (\protect\hyperlink{ref-Szumigalski.1996}{1996}), (13) Thormann et al. (\protect\hyperlink{ref-Thormann.2001}{2001}), (14) Trinder et al. (\protect\hyperlink{ref-Trinder.2008}{2008}), (15) Vitt (\protect\hyperlink{ref-Vitt.1990}{1990}). \emph{Sphagnum} spec. are samples that have been identified only to the genus level.}\label{fig:sup-out-mm35-1-p5}
\end{figure}



\begin{figure}[H]

{\centering \includegraphics[width=1\linewidth]{figures/leaching_plot_1_8} 

}

\caption{Estimated parameter controlling a decrease of decomposition rates over time (\(\alpha\)) (a), and decomposition rates (b) grouped by species and study for model 2-4. Points represent averages and error bars 95\% confidence intervals. The study is indicated by numbers on the x axis: (1) Asada and Warner (\protect\hyperlink{ref-Asada.2005b}{2005}), (2) Bartsch and Moore (\protect\hyperlink{ref-Bartsch.1985}{1985}), (3) Breeuwer et al. (\protect\hyperlink{ref-Breeuwer.2008}{2008}), (4) Golovatskaya and Nikonova (\protect\hyperlink{ref-Golovatskaya.2017}{2017}), (5) Hagemann and Moroni (\protect\hyperlink{ref-Hagemann.2015}{2015}), (6) Johnson and Damman (\protect\hyperlink{ref-Johnson.1991}{1991}), (7) Mäkilä et al. (\protect\hyperlink{ref-Makila.2018}{2018}), (8) Prevost et al. (\protect\hyperlink{ref-Prevost.1997}{1997}), (9) Scheffer et al. (\protect\hyperlink{ref-Scheffer.2001}{2001}), (10) Straková et al. (\protect\hyperlink{ref-Strakova.2010}{2010}), (11) Szumigalski and Bayley (\protect\hyperlink{ref-Szumigalski.1996}{1996}), (12) Thormann et al. (\protect\hyperlink{ref-Thormann.2001}{2001}), (13) Trinder et al. (\protect\hyperlink{ref-Trinder.2008}{2008}), (14) Vitt (\protect\hyperlink{ref-Vitt.1990}{1990}). \emph{Sphagnum} spec. are samples that have been identified only to the genus level.}\label{fig:sup-out-mm35-2-p5}
\end{figure}

\hypertarget{sup-3}{%
\section{Difference in initial leaching loss and decomposition rate estimates between models 1-1 and 1-2 and between models 1-3 and 1-4}\label{sup-3}}

Here, we compare estimated initial leaching losses and decomposition rates for all other samples when data from Bengtsson et al. (\protect\hyperlink{ref-Bengtsson.2017}{2017}) are included (models 1-1 and 1-3) or not included (model 1-2 and 1-4).



\begin{figure}[H]

{\centering \includegraphics[width=1\linewidth]{figures/leaching_plot_4_fit_1_vs_fit_2} 

}

\caption{Difference in initial leaching losses (a) and decomposition rate (b) between model 1-1 and model 1-2 for all samples except from Bengtsson et al. (\protect\hyperlink{ref-Bengtsson.2017}{2017}).}\label{fig:sup-out-sdm-mm27-1-mm27-2-parameter-difference-p1-1}
\end{figure}



\begin{figure}[H]

{\centering \includegraphics[width=1\linewidth]{figures/leaching_plot_4_fit_3_vs_fit_4} 

}

\caption{Difference in initial leaching losses (a) and decomposition rate (b) between model 1-3 and model 1-4 for all samples except from Bengtsson et al. (\protect\hyperlink{ref-Bengtsson.2017}{2017}).}\label{fig:sup-out-sdm-mm36-1-mm36-2-parameter-difference-p1-1}
\end{figure}

\hypertarget{sup-4}{%
\section{Comparison of one-pool decomposition rates estimated while considering or ignoring initial leaching losses and also allowing the decomposition rate to decrease with decreasing remaining mass}\label{sup-4}}

In section \ref{main-out-res-2} in the main text, we have compared one-pool decomposition rate estimates and their uncertainties between a model which considers initial leaching losses (model 1-1) and a model which ignores initial leaching losses (model 2-1). Both of these models assume that the decomposition rate remains constant over time, whereas this may in reality not be the case. We therefore repeated the analysis using two models which allow the decomposition rate to decrease over time (see last paragraph in section \ref{main-out-methods-1} in the main text, models 1-3 and 2-3).



\begin{figure}[H]

{\centering \includegraphics[width=0.6\linewidth]{figures/leaching_plot_2_fit_7_and_fit_3} 

}

\caption{(a) Decomposition rate estimates, either considering leaching (black) or ignoring leaching (grey) versus average initial leaching losses estimated by the model considering initial leaching losses. Points are average estimates and error bars are 95\% prediction intervals. (b) Standard deviation of decomposition rate estimates, either considering leaching (black) or ignoring leaching (grey) versus average initial leaching losses estimated by the model considering initial leaching losses. Compare with Fig. \ref{main-fig:out-mm27-1-mm28-1-p2} in the main text.}\label{fig:sup-out-mm36-1-mm35-1-p3-1}
\end{figure}

\hypertarget{sup-5}{%
\section{Prior choices and justification}\label{sup-5}}



\begin{table}[H]

\caption{\label{tab:sup-out-d-sdm-all-models-priors-1}Prior distributions of all Bayesian models and their justifications. ``HPM parameter'' is the name of the corresponding parameter in the Holocene Peatland Model (\protect\hyperlink{ref-Frolking.2010}{Frolking et al., 2010}). When there is no value for ``Justification'', the prior was chosen based on prior predictive checks against the data. This prior predictive check tests whether the models can produce distributions of measured variables we expect based on prior knowledge.}
\centering
\resizebox{\linewidth}{!}{
\begin{tabular}[t]{lll>{\raggedright\arraybackslash}p{7cm}}
\toprule
Parameter & Unit & Prior distribution & Justification\\
\midrule
l\_2\_p1 & (g g$_\text{initial}$) (logit scale) & normal(-3.5, l\_2\_p1\_p2) & Assumes an average initial leaching loss across all available litterbag data within (95\% confidence interval) (0.012, 0.066) g g$_\text{initial}^{-1}$\\
l\_2\_p2 & (g g$_\text{initial}$) (logit scale) & normal(0, l\_2\_p2\_p2) & \\
l\_2\_p3 & (g g$_\text{initial}$) (logit scale) & normal(0, l\_2\_p3\_p2) & \\
l\_2\_p4 & (g g$_\text{initial}$) (logit scale) & normal(0, l\_2\_p4\_p2) & \\
k\_2\_p1 & (yr$^{-1}$) (log scale) & normal(-2.9, k\_2\_p1\_p2) & Assumes an average initial decomposition rate across all available litterbag data within (95\% confidence interval) (0.023, 0.129) yr$^{-1}$\\
\addlinespace
k\_2\_p2 & (yr$^{-1}$) (log scale) & normal(0, k\_2\_p2\_p2) & \\
k\_2\_p3 & (yr$^{-1}$) (log scale) & normal(0, k\_2\_p3\_p2) & \\
k\_2\_p4 & (yr$^{-1}$) (log scale) & normal(0, k\_2\_p4\_p2) & \\
phi\_2\_p2\_p1 & (-) (log scale) & normal(5, phi\_2\_p2\_p1\_p2) & \\
phi\_2\_p2\_p2 & (-) (log scale) & normal(0, phi\_2\_p2\_p2\_p2) & \\
\addlinespace
phi\_2\_p2\_p3 & (-) (log scale) & normal(0, phi\_2\_p2\_p3\_p2) & \\
phi\_2\_p2\_p4 & (-) (log scale) & normal(0, phi\_2\_p2\_p4\_p2) & \\
alpha\_2\_p1 & (-) (log scale) & normal(-0.2, 0.3) & Assumes an average $\alpha$ across all available litterbag data within (95\% confidence interval) (1.458, 2.468)\\
alpha\_2\_p2 & (-) (log scale) & normal(0, 0.3) & \\
alpha\_2\_p3 & (-) (log scale) & normal(0, 0.3) & \\
\addlinespace
alpha\_2\_p4 & (-) (log scale) & normal(0, 0.2) & \\
k\_2\_p1\_p2 & (yr$^{-1}$) (log scale) & half-normal(0, 0.4) & \\
k\_2\_p2\_p2 & (yr$^{-1}$) (log scale) & half-normal(0, 0.4) & \\
k\_2\_p3\_p2 & (yr$^{-1}$) (log scale) & half-normal(0, 0.4) & \\
k\_2\_p4\_p2 & (yr$^{-1}$) (log scale) & half-normal(0, 0.4) & \\
\addlinespace
phi\_2\_p2\_p1\_p2 & (-) (log scale) & half-normal(0, 0.3) & \\
phi\_2\_p2\_p2\_p2 & (-) (log scale) & half-normal(0, 0.3) & \\
phi\_2\_p2\_p3\_p2 & (-) (log scale) & half-normal(0, 0.3) & \\
phi\_2\_p2\_p4\_p2 & (-) (log scale) & half-normal(0, 0.3) & \\
l\_2\_p1\_p2 & (g g$_\text{initial}$) (logit scale) & half-normal(0, 0.4) & \\
\addlinespace
l\_2\_p2\_p2 & (g g$_\text{initial}$) (logit scale) & half-normal(0, 0.4) & \\
l\_2\_p3\_p2 & (g g$_\text{initial}$) (logit scale) & half-normal(0, 0.4) & \\
l\_2\_p4\_p2 & (g g$_\text{initial}$) (logit scale) & half-normal(0, 0.4) & \\
\bottomrule
\end{tabular}}
\end{table}

\hypertarget{sup-6}{%
\section{Prior and posterior predictive checks}\label{sup-6}}



\begin{figure}[H]

{\centering \includegraphics[width=1\linewidth]{figures/leaching_plot_ppc_prior_m} 

}

\caption{Density estimate of 100 sets of remaining masses sampled from the prior distribution of each model (light blue lines) versus density estimate of the measured remaining masses from the litterbag studies.}\label{fig:sup-out-p-sdm-all-models-check-1-1}
\end{figure}



\begin{figure}[H]

{\centering \includegraphics[width=1\linewidth]{figures/leaching_plot_ppc_posterior_m} 

}

\caption{Density estimate of 100 sets of remaining masses sampled from the posterior distribution of each model (light blue lines) versus density estimate of the measured remaining masses from the litterbag studies.}\label{fig:sup-out-p-sdm-all-models-check-2-1}
\end{figure}



\begin{figure}[H]

{\centering \includegraphics[width=1\linewidth]{figures/leaching_plot_ppc_prior_phi} 

}

\caption{Density estimate of 100 sets of remaining mass errors (converted to precision) sampled from the prior distribution of each model (light blue lines) versus density estimate of the measured remaining masses from the litterbag studies. The x axis is log scaled.}\label{fig:sup-out-p-sdm-all-models-check-3-1}
\end{figure}



\begin{figure}[H]

{\centering \includegraphics[width=1\linewidth]{figures/leaching_plot_ppc_posterior_phi} 

}

\caption{Density estimate of 100 sets of remaining mass errors (converted to precision) sampled from the posterior distribution of each model (light blue lines) versus density estimate of the measured remaining masses from the litterbag studies. The x axis is log scaled.}\label{fig:sup-out-p-sdm-all-models-check-4-1}
\end{figure}

\hypertarget{sup-11}{%
\section{Sensitivity of parameter estimates to priors and the experimental design of litterbag experiments}\label{sup-11}}

To check that the model considering initial leaching losses can in principle correctly estimate parameter values for \(l_0\), \(k_0\), and \(\alpha\) under conditions which resemble those in available litterbag experiments, we simulated a dataset with all combinations of different values for these parameters (\(l_0\): 1, 5, or 15 mass-\%, \(k_0\): 0.01, 0.05, or 0.15 yr\(^{-1}\), \(\alpha\): 1 or 3, and constant precision parameter for remaining masses of 200 (near the median precision estimated by the model in the main text when excluding data from Bengtsson et al. (\protect\hyperlink{ref-Bengtsson.2017}{2017}), 241), implying standard deviations for remaining masses of 0.7 to 3.5 mass-\%). From this, we simulated remaining masses according to three litterbag designs which differ in the time points at which litterbags are collected after the incubation started (collection plan) (design 1: after 1 and 2 years, design 2: after \textasciitilde{} 20 days, 1, and 2 years, design 3: after \textasciitilde{} 20 days, 1, 2, 3, and 5 years) and the number of litterbags collected at each time point (5 or 10, for each collection plan).

We then used the same hierarchical Bayesian model as for the model in the main text (see equations \ref{main-eq:leaching-hierarchical-model-l0} and \ref{main-eq:leaching-hierarchical-model-l0-2} and supporting information \ref{sup-13}) and estimated parameters for the simulated data (model 1-5, Tab. \ref{tab:m-litterbag-synthesis-models}), treating samples with the same \(k_0\) and \(\alpha\) as samples from the same species, and samples with the same experimental design and \(l_0\) as samples from the same study.

The estimated parameter values for \(l_0\), \(k_0\), and \(\alpha\) can be compared against the values used to simulate the data and this allowed us to (1) test whether the models can estimate the true parameter values from litterbag data if our model is a good approximation to the data generating process, if \emph{Sphagnum} species have similar \(k_0\) and \(\alpha\) in different studies, but may vary in their \(l_0\), (2) analyze how the true \(k_0\), \(l\), and \(\alpha\) control how accurate any of these parameters can be estimated, and (3) how the litterbag design controls how accurate \(l_0\), \(k_0\), and \(\alpha\) can be estimated.

To provide an additional test that the model used in the main text can in principle provide accurate estimates of \(k_0\), \(l_0\), \(\alpha\) when the simulation is as similar to available litterbag experiments as possible, we sampled parameter values from the posterior of the model in the main text and simulated remaining masses and standard deviations of remaining masses which could be observed in the available litterbag experiments if the model approximates the true data generating process. We then estimated \(k_0\), \(l_0\), \(\alpha\) based on the simulated data (model 1-6) and compared the estimates to the parameter values sampled from the model in the main text to simulate the data.

The results of this analysis indicate that all parameters except \(\alpha\) can be successfully estimated for the simulated data (figures \ref{fig:leaching-sup-plot-5-1} to \ref{fig:leaching-sup-plot-5-3}). The true value for \(k_0\) and \(l_0\) is contained in central 95\% posterior intervals more than 95\% of the simulated litterbag experiments. Also the maximum bias (absolute difference) was comparatively small: 0.043 (0.002, 0.105) yr\(^{-1}\) for \(k_0\) and 4.4 (0.1, 12.7) mass-\% for \(l_0\). Biases and errors for both \(k_0\) and \(l_0\) were smallest when the first litterbags were collected ca. 20 days after the start of the experiment compared to after a year. Importantly, the bias was smallest for all experimental designs for the smallest true \(l_0\) indicating that small initial leaching losses are not overestimated (supporting information \ref{sup-11}).

Estimates for \(\alpha\) were always biased, except when the prior was already similar to the true value, indicating that litterbag data provide little information about this parameter which therefore is dominated by the prior (figure \ref{fig:leaching-sup-plot-5-3}). In our simulations, \(\alpha\) has only a small influence on the estimates of \(k_0\) and \(l_0\) (figures \ref{fig:leaching-sup-plot-5-1} and \ref{fig:leaching-sup-plot-5-2}), indicating that also for available litterbag data, this bias should affect our estimates for \(l_0\) and \(k_0\) not much.







\begin{figure}[H]

{\centering \includegraphics[width=1\linewidth]{figures/leaching_plot_5_plot_1} 

}

\caption{True decomposition rates minus estimated decomposition rates for model 1-5 versus true initial leaching losses (\(l_{0,\text{true}}\)). Columns show values for different experimental designs (see the main text for details). Rows show values for different true decomposition rates (\(k_{0,\text{true}}\)) (yr\(^{-1}\)).}\label{fig:leaching-sup-plot-5-1}
\end{figure}
\begin{figure}[H]

{\centering \includegraphics[width=1\linewidth]{figures/leaching_plot_5_plot_2} 

}

\caption{True initial leaching losses minus estimated initial leaching losses for model 1-5 versus true initial leaching losses (\(l_{0,\text{true}}\)). Columns show values for different experimental designs (see the main text for details). Rows show values for different true decomposition rates (\(k_{0,\text{true}}\)) (yr\(^{-1}\)).}\label{fig:leaching-sup-plot-5-2}
\end{figure}
\begin{figure}[H]

{\centering \includegraphics[width=1\linewidth]{figures/leaching_plot_5_plot_3} 

}

\caption{True \(\alpha\) minus estimated \(\alpha\) for model 1-5 versus true initial leaching losses (\(l_{0,\text{true}}\)). Columns show values for different experimental designs (see the main text for details). Rows show values for different true decomposition rates (\(k_{0,\text{true}}\)) (yr\(^{-1}\)).}\label{fig:leaching-sup-plot-5-3}
\end{figure}

Based on these results we can assess how large the risk is that our model overestimated \(l_0\). If the experimental design causes available litterbag data to provide only few information on \(l_0\) and \(k_0\), their estimates will depend more strongly on the prior and the variability of initial leaching losses and decomposition rates across studies and species, meaning that uncertain estimates are constrained to the global average. In the sensitivity analysis, this caused underestimation of larger initial leaching losses, but no overestimation of smaller initial leaching losses for the sampling design where the first litterbags were collected only after a year because we chose a prior which reflects that previous studies which directly measured initial leaching losses mostly observed small initial leaching losses. Even though this simulation is not directly transferable to the real data, the general pattern that initial leaching losses are constrained to the estimated average still holds and thus even if the data are not informative, our estimates should be conservative because also our prior is conservative.

Moreover, for 5 experiments where the first litterbag was collected a year after the start of the experiment, the estimated average decomposition rate was larger than 0.03 yr\(^{-1}\) and the estimated average initial leaching loss smaller than 8 mass-\%, indicating that there are only few initial leaching loss estimates which are small. In addition, the sensitivity analysis with data simulated from the posterior distribution of the model in the main text suggests that if the posterior is an approximately correct representation of the true parameter values, estimating these parameter values from the simulated data does not result in biased estimates for \(l_0\).

In contrast, the risk of underestimation of large initial leaching losses is probably larger because 42 experiments where the first litterbag was collected a year after the start of the experiment had estimates for average \(k_0\) larger than 0.03 yr\(^{-1}\) and estimates for average \(l_0\) larger than 10 mass-\%, indicating that underestimation of large initial leaching losses may be more common in our analysis than overestimation of small initial leaching losses.

Overall, even though the design of most available litterbag data introduces errors which should certainly be reduced to validate our results, we do not expect that there is a serious overestimation of small initial leaching losses and estimates of large initial leaching losses may be conservative.

\hypertarget{sup-14}{%
\section{Further information on Bayesian data analysis}\label{sup-14}}

None of the models had divergent transitions, the minimum bulk effective sample size was larger than 400, and the largest improved \(\hat{R}\) was 1.01, indicating that all chains converged (\protect\hyperlink{ref-Vehtari.2021}{Vehtari et al., 2021}). Monte Carlo standard errors (MSCE) (\protect\hyperlink{ref-Vehtari.2021}{Vehtari et al., 2021}) for the median were at most 0.081 yr\(^{-1}\) for \(k_0\) (if initial leaching was considered), 0.204 yr\(^{-1}\) for \(k_0\) (if initial leaching was ignored), 1.419 mass-\% for \(l_0\), 0.19 for \(\alpha\), and 0.47 mass-\% for the remaining mass. For the 2.5\% and 97.5\% quantiles, MCSE were at most 0.137 yr\(^{-1}\) for \(k_0\) (if initial leaching was considered), 1.267 yr\(^{-1}\) for \(k_0\) (if initial leaching was ignored), 0.691 mass-\% for \(l_0\), 0.701 for \(\alpha\), and 3.044 mass-\% for the remaining mass.

All other computations were done in R (4.2.0) (\protect\hyperlink{ref-RCoreTeam.2022}{R Core Team, 2022}). We computed prior and posterior predictive checks with the bayesplot package (1.9.0) (\protect\hyperlink{ref-Gabry.2022}{Gabry and Mahr, 2022}) (supporting section \ref{sup-6}). Data were handled with tidyverse packages (\protect\hyperlink{ref-Wickham.2019}{Wickham et al., 2019}), MCMC samples with the posterior (1.5.0) (\protect\hyperlink{ref-Burkner.2023}{Bürkner et al., 2023}) and tidybayes (3.0.2) (\protect\hyperlink{ref-Kay.2022}{Kay, 2022}) packages. Graphics were created with ggplot2 (3.4.4) (\protect\hyperlink{ref-Wickham.2016}{Wickham, 2016}) and patchwork (1.1.1) (\protect\hyperlink{ref-Pedersen.2020}{Pedersen, 2020}).

\hypertarget{sup-7}{%
\section{Initial leaching losses and one-pool decomposition rates as estimated by all models in this study for all species and studies}\label{sup-7}}



















\begin{table}[H]

\caption{\label{tab:sup-out-l-2-all-models}Range of average estimated initial leaching losses (percent of the initial mass) for litterbag replicates grouped by species and study as estimated by all models (see Tab. \ref{tab:m-litterbag-synthesis-models}). Ranges were computed on average estimates and therefore do not consider the uncertainty of initial leaching losses for individual litterbag replicates.}
\centering
\resizebox{\linewidth}{!}{
\begin{tabular}[t]{llrllllllll}
\toprule
\multicolumn{1}{c}{ } & \multicolumn{1}{c}{ } & \multicolumn{1}{c}{ } & \multicolumn{8}{c}{Initial leaching losses (mass-\%)} \\
\cmidrule(l{3pt}r{3pt}){4-11}
Taxon & Study & Sample size & model 1-1 & model 1-2 & model 1-3 & model 1-4 & model 2-1 & model 2-2 & model 2-3 & model 2-4\\
\midrule
 & Bengtsson et al. (\protect\hyperlink{ref-Bengtsson.2017}{2017}) & 10 & 17.8, 28.9 &  & 10.96, 16.42 &  &  &  &  & \\

 & Golovatskaya and Nikonova (\protect\hyperlink{ref-Golovatskaya.2017}{2017}) & 2 & 10.11, 14.88 & 9.95, 14.04 & 7.8, 11.61 & 7.57, 11.72 &  &  &  & \\

 & Mäkilä et al. (\protect\hyperlink{ref-Makila.2018}{2018}) & 2 & 14.1, 16.29 & 13.73, 16.05 & 11.76, 13.55 & 11.66, 13.4 &  &  &  & \\

 & Straková et al. (\protect\hyperlink{ref-Strakova.2010}{2010}) & 9 & 11.45, 22.53 & 10.92, 20.71 & 9.71, 18.55 & 9.56, 17.65 &  &  &  & \\

\multirow[t]{-5}{*}{\raggedright\arraybackslash $S.~angustifolium$} & Vitt (\protect\hyperlink{ref-Vitt.1990}{1990}) & 1 & 16.33, 16.33 & 15.79, 15.79 & 13.93, 13.93 & 13.52, 13.52 &  &  &  & \\
\cmidrule{1-11}
$S.~auriculatum$ & Trinder et al. (\protect\hyperlink{ref-Trinder.2008}{2008}) & 3 & 5.77, 6.03 & 5.15, 5.37 & 5.52, 5.79 & 5.01, 5.22 &  &  &  & \\
\cmidrule{1-11}
 & Bengtsson et al. (\protect\hyperlink{ref-Bengtsson.2017}{2017}) & 9 & 15.26, 18.97 &  & 13.18, 15.96 &  &  &  &  & \\

 & Breeuwer et al. (\protect\hyperlink{ref-Breeuwer.2008}{2008}) & 8 & 11.99, 16.89 & 12.02, 17.08 & 11.81, 16.69 & 11.88, 16.88 &  &  &  & \\

 & Mäkilä et al. (\protect\hyperlink{ref-Makila.2018}{2018}) & 2 & 14.28, 16.72 & 14.72, 17.26 & 13.54, 15.89 & 14.31, 16.8 &  &  &  & \\

\multirow[t]{-4}{*}{\raggedright\arraybackslash $S.~balticum$} & Straková et al. (\protect\hyperlink{ref-Strakova.2010}{2010}) & 2 & 11.21, 14.81 & 11.69, 15.38 & 10.64, 14 & 11.32, 14.91 &  &  &  & \\
\cmidrule{1-11}
$S.~capillifolium$ & Bengtsson et al. (\protect\hyperlink{ref-Bengtsson.2017}{2017}) & 10 & 9.15, 27.38 &  & 8.85, 26.39 &  &  &  &  & \\
\cmidrule{1-11}
$S.~contortum$ & Bengtsson et al. (\protect\hyperlink{ref-Bengtsson.2017}{2017}) & 6 & 19.93, 23.61 &  & 19.5, 23.09 &  &  &  &  & \\
\cmidrule{1-11}
 & Bengtsson et al. (\protect\hyperlink{ref-Bengtsson.2017}{2017}) & 10 & 11.86, 18.62 &  & 9.89, 13.53 &  &  &  &  & \\

\multirow[t]{-2}{*}{\raggedright\arraybackslash $S.~cuspidatum$} & Johnson and Damman (\protect\hyperlink{ref-Johnson.1991}{1991}) & 5 & 12.66, 17.94 & 12.87, 18.2 & 12.32, 17.38 & 12.57, 17.82 &  &  &  & \\
\cmidrule{1-11}
 & Bengtsson et al. (\protect\hyperlink{ref-Bengtsson.2017}{2017}) & 10 & 16.99, 50.6 &  & 12.62, 32.95 &  &  &  &  & \\

\multirow[t]{-2}{*}{\raggedright\arraybackslash $S.~fallax$} & Straková et al. (\protect\hyperlink{ref-Strakova.2010}{2010}) & 4 & 10.56, 19.66 & 11.17, 20.4 & 9.39, 17.11 & 10.2, 18.08 &  &  &  & \\
\cmidrule{1-11}
 & Asada and Warner (\protect\hyperlink{ref-Asada.2005b}{2005}) & 8 & 11.31, 18.94 & 11.27, 18.73 & 11.05, 18.67 & 10.92, 18.28 &  &  &  & \\

 & Bengtsson et al. (\protect\hyperlink{ref-Bengtsson.2017}{2017}) & 16 & 5.91, 13.8 &  & 6.06, 13.6 &  &  &  &  & \\

 & Breeuwer et al. (\protect\hyperlink{ref-Breeuwer.2008}{2008}) & 8 & 5.27, 9.8 & 5.29, 9.87 & 5.25, 9.8 & 5.24, 9.73 &  &  &  & \\

 & Golovatskaya and Nikonova (\protect\hyperlink{ref-Golovatskaya.2017}{2017}) & 2 & 2.34, 2.91 & 2.44, 3.05 & 2.62, 3.36 & 2.81, 3.72 &  &  &  & \\

 & Johnson and Damman (\protect\hyperlink{ref-Johnson.1991}{1991}) & 5 & 8.89, 10.19 & 8.91, 10.18 & 8.87, 10.14 & 8.79, 10.08 &  &  &  & \\

 & Mäkilä et al. (\protect\hyperlink{ref-Makila.2018}{2018}) & 3 & 10.38, 11.69 & 10.28, 11.53 & 10.21, 11.5 & 9.93, 11.22 &  &  &  & \\

 & Straková et al. (\protect\hyperlink{ref-Strakova.2010}{2010}) & 3 & 10.3, 12.64 & 10.27, 12.66 & 10.07, 12.47 & 9.86, 12.16 &  &  &  & \\

 & Szumigalski and Bayley (\protect\hyperlink{ref-Szumigalski.1996}{1996}) & 1 & 10.74, 10.74 & 10.72, 10.72 & 10.67, 10.67 & 10.41, 10.41 &  &  &  & \\

 & Thormann et al. (\protect\hyperlink{ref-Thormann.2001}{2001}) & 1 & 16.4, 16.4 & 16.36, 16.36 & 16.28, 16.28 & 15.97, 15.97 &  &  &  & \\

\multirow[t]{-10}{*}{\raggedright\arraybackslash $S.~fuscum$} & Vitt (\protect\hyperlink{ref-Vitt.1990}{1990}) & 1 & 8.88, 8.88 & 8.88, 8.88 & 8.77, 8.77 & 8.57, 8.57 &  &  &  & \\
\cmidrule{1-11}
$S.~girgensohnii$ & Bengtsson et al. (\protect\hyperlink{ref-Bengtsson.2017}{2017}) & 9 & 20.11, 27.77 &  & 17.29, 23.91 &  &  &  &  & \\
\cmidrule{1-11}
 & Bartsch and Moore (\protect\hyperlink{ref-Bartsch.1985}{1985}) & 2 & 4.73, 5.54 & 5.3, 6.16 & 4.58, 5.33 & 5.07, 5.81 &  &  &  & \\

\multirow[t]{-2}{*}{\raggedright\arraybackslash $S.~lindbergii$} & Bengtsson et al. (\protect\hyperlink{ref-Bengtsson.2017}{2017}) & 10 & 13.8, 22.41 &  & 12.79, 20.9 &  &  &  &  & \\
\cmidrule{1-11}
 & Bengtsson et al. (\protect\hyperlink{ref-Bengtsson.2017}{2017}) & 27 & 8.29, 32.18 &  & 8.06, 29.37 &  &  &  &  & \\

 & Mäkilä et al. (\protect\hyperlink{ref-Makila.2018}{2018}) & 3 & 9.51, 13.33 & 9.29, 13.01 & 9.09, 12.66 & 8.81, 12.29 &  &  &  & \\

 & Straková et al. (\protect\hyperlink{ref-Strakova.2010}{2010}) & 3 & 10.73, 12.62 & 10.46, 12.22 & 9.88, 11.65 & 9.79, 11.33 &  &  &  & \\

\multirow[t]{-4}{*}{\raggedright\arraybackslash $S.~magellanicum~aggr.$} & Vitt (\protect\hyperlink{ref-Vitt.1990}{1990}) & 1 & 10.42, 10.42 & 10.5, 10.5 & 9.79, 9.79 & 9.82, 9.82 &  &  &  & \\
\cmidrule{1-11}
 & Bengtsson et al. (\protect\hyperlink{ref-Bengtsson.2017}{2017}) & 8 & 9.99, 13.71 &  & 7.69, 9.84 &  &  &  &  & \\

\multirow[t]{-2}{*}{\raggedright\arraybackslash $S.~majus$} & Mäkilä et al. (\protect\hyperlink{ref-Makila.2018}{2018}) & 2 & 9.93, 10.04 & 10.76, 11.16 & 8.76, 8.85 & 10.1, 10.45 &  &  &  & \\
\cmidrule{1-11}
 & Bengtsson et al. (\protect\hyperlink{ref-Bengtsson.2017}{2017}) & 10 & 13.85, 21.76 &  & 13.14, 20.58 &  &  &  &  & \\

 & Scheffer et al. (\protect\hyperlink{ref-Scheffer.2001}{2001}) & 2 & 8.95, 9.65 & 8.09, 8.46 & 8.62, 9.18 & 7.8, 8.09 &  &  &  & \\

\multirow[t]{-3}{*}{\raggedright\arraybackslash $S.~papillosum$} & Straková et al. (\protect\hyperlink{ref-Strakova.2010}{2010}) & 4 & 6.59, 10.7 & 7.2, 11.33 & 6.33, 10.36 & 6.86, 10.94 &  &  &  & \\
\cmidrule{1-11}
 & Bengtsson et al. (\protect\hyperlink{ref-Bengtsson.2017}{2017}) & 10 & 7.64, 11.34 &  & 7.02, 9.83 &  &  &  &  & \\

\multirow[t]{-2}{*}{\raggedright\arraybackslash $S.~rubellum$} & Mäkilä et al. (\protect\hyperlink{ref-Makila.2018}{2018}) & 2 & 13.59, 14.26 & 15.59, 16.33 & 12.36, 12.84 & 14.8, 15.55 &  &  &  & \\
\cmidrule{1-11}
$S.~russowii$ & Straková et al. (\protect\hyperlink{ref-Strakova.2010}{2010}) & 3 & 12.94, 15.78 & 13.4, 16.24 & 11.7, 14.33 & 12.29, 15.14 &  &  &  & \\
\cmidrule{1-11}
$S.~russowii$ and $capillifolium$ & Hagemann and Moroni (\protect\hyperlink{ref-Hagemann.2015}{2015}) & 18 & 15.85, 21.57 & 15.98, 21.54 & 8.62, 14.71 & 7.56, 13.08 &  &  &  & \\
\cmidrule{1-11}
$S.~squarrosum$ & Scheffer et al. (\protect\hyperlink{ref-Scheffer.2001}{2001}) & 2 & 9.68, 10.08 & 8.91, 9.61 & 9.18, 9.57 & 8.55, 9.23 &  &  &  & \\
\cmidrule{1-11}
$S.~tenellum$ & Bengtsson et al. (\protect\hyperlink{ref-Bengtsson.2017}{2017}) & 9 & 15.18, 33.95 &  & 13.52, 25.68 &  &  &  &  & \\
\cmidrule{1-11}
$S.~teres$ & Szumigalski and Bayley (\protect\hyperlink{ref-Szumigalski.1996}{1996}) & 1 & 11.89, 11.89 & 11.9, 11.9 & 11.14, 11.14 & 11.32, 11.32 &  &  &  & \\
\cmidrule{1-11}
$S.~warnstorfii$ & Bengtsson et al. (\protect\hyperlink{ref-Bengtsson.2017}{2017}) & 6 & 14.37, 17.52 &  & 13.84, 16.95 &  &  &  &  & \\
\cmidrule{1-11}
 & Bartsch and Moore (\protect\hyperlink{ref-Bartsch.1985}{1985}) & 6 & 3.5, 4.63 & 3.89, 5.21 & 3.41, 4.59 & 3.71, 4.97 &  &  &  & \\

\multirow[t]{-2}{*}{\raggedright\arraybackslash $Sphagnum$ spec.} & Prevost et al. (\protect\hyperlink{ref-Prevost.1997}{1997}) & 10 & 2.6, 5.95 & 2.75, 6.03 & 2.56, 5.77 & 2.64, 5.68 &  &  &  & \\
\bottomrule
\end{tabular}}
\end{table}



\begin{table}[H]

\caption{\label{tab:sup-out-k-2-all-models}Range of average one-pool exponential decomposition rates (yr\(^{-1}\)) for litterbag replicates grouped by species and study as estimated by all models (see Tab. \ref{tab:m-litterbag-synthesis-models}). Ranges were computed on average estimates and therefore do not consider the uncertainty of decomposition rates for individual litterbag replicates.}
\centering
\resizebox{\linewidth}{!}{
\begin{tabular}[t]{llrllllllll}
\toprule
\multicolumn{1}{c}{ } & \multicolumn{1}{c}{ } & \multicolumn{1}{c}{ } & \multicolumn{8}{c}{Decomposition rates (yr$^{-1}$)} \\
\cmidrule(l{3pt}r{3pt}){4-11}
Taxon & Study & Sample size & model 1-1 & model 1-2 & model 1-3 & model 1-4 & model 2-1 & model 2-2 & model 2-3 & model 2-4\\
\midrule
 & Bengtsson et al. (\protect\hyperlink{ref-Bengtsson.2017}{2017}) & 10 & 0.59, 0.82 &  & 0.97, 1.68 &  & 0.86, 1.17 &  & 2.02, 3.6 & \\

 & Golovatskaya and Nikonova (\protect\hyperlink{ref-Golovatskaya.2017}{2017}) & 2 & 0.1, 0.25 & 0.1, 0.25 & 0.14, 0.37 & 0.15, 0.41 & 0.16, 0.35 & 0.16, 0.35 & 0.32, 1.05 & 0.4, 1.19\\

 & Mäkilä et al. (\protect\hyperlink{ref-Makila.2018}{2018}) & 2 & 0.07, 0.08 & 0.07, 0.08 & 0.1, 0.11 & 0.1, 0.12 & 0.18, 0.2 & 0.17, 0.19 & 0.65, 0.78 & 0.69, 0.83\\

 & Straková et al. (\protect\hyperlink{ref-Strakova.2010}{2010}) & 9 & 0.09, 0.2 & 0.09, 0.2 & 0.11, 0.26 & 0.13, 0.31 & 0.18, 0.33 & 0.19, 0.33 & 0.45, 1.35 & 0.49, 1.52\\

\multirow[t]{-5}{*}{\raggedright\arraybackslash $S.~angustifolium$} & Vitt (\protect\hyperlink{ref-Vitt.1990}{1990}) & 1 & 0.06, 0.06 & 0.06, 0.06 & 0.1, 0.1 & 0.11, 0.11 & 0.2, 0.2 & 0.19, 0.19 & 0.8, 0.8 & 0.85, 0.85\\
\cmidrule{1-11}
$S.~auriculatum$ & Trinder et al. (\protect\hyperlink{ref-Trinder.2008}{2008}) & 3 & 0.05, 0.05 & 0.04, 0.04 & 0.05, 0.05 & 0.04, 0.04 & 0.06, 0.06 & 0.05, 0.06 & 0.37, 0.39 & 0.31, 0.32\\
\cmidrule{1-11}
 & Bengtsson et al. (\protect\hyperlink{ref-Bengtsson.2017}{2017}) & 9 & 0.24, 0.48 &  & 0.34, 0.69 &  & 0.48, 0.71 &  & 2.11, 3.46 & \\

 & Breeuwer et al. (\protect\hyperlink{ref-Breeuwer.2008}{2008}) & 8 & 0.04, 0.06 & 0.03, 0.06 & 0.04, 0.06 & 0.04, 0.06 & 0.17, 0.2 & 0.16, 0.2 & 3.59, 5.09 & 4.65, 6.5\\

 & Mäkilä et al. (\protect\hyperlink{ref-Makila.2018}{2018}) & 2 & 0.05, 0.05 & 0.04, 0.04 & 0.06, 0.06 & 0.05, 0.05 & 0.16, 0.18 & 0.15, 0.17 & 1.83, 2.15 & 2.39, 2.77\\

\multirow[t]{-4}{*}{\raggedright\arraybackslash $S.~balticum$} & Straková et al. (\protect\hyperlink{ref-Strakova.2010}{2010}) & 2 & 0.05, 0.06 & 0.04, 0.05 & 0.05, 0.07 & 0.05, 0.07 & 0.14, 0.17 & 0.13, 0.16 & 1.41, 1.86 & 1.92, 2.54\\
\cmidrule{1-11}
$S.~capillifolium$ & Bengtsson et al. (\protect\hyperlink{ref-Bengtsson.2017}{2017}) & 10 & 0.05, 0.08 &  & 0.06, 0.1 &  & 0.15, 0.37 &  & 0.91, 3.41 & \\
\cmidrule{1-11}
$S.~contortum$ & Bengtsson et al. (\protect\hyperlink{ref-Bengtsson.2017}{2017}) & 6 & 0.06, 0.07 &  & 0.07, 0.08 &  & 0.32, 0.35 &  & 1.81, 2.32 & \\
\cmidrule{1-11}
 & Bengtsson et al. (\protect\hyperlink{ref-Bengtsson.2017}{2017}) & 10 & 0.39, 0.79 &  & 0.5, 1.12 &  & 0.57, 0.96 &  & 1.61, 3.45 & \\

\multirow[t]{-2}{*}{\raggedright\arraybackslash $S.~cuspidatum$} & Johnson and Damman (\protect\hyperlink{ref-Johnson.1991}{1991}) & 5 & 0.03, 0.05 & 0.03, 0.05 & 0.03, 0.06 & 0.03, 0.06 & 0.14, 0.19 & 0.14, 0.18 & 1.25, 1.85 & 1.35, 2.08\\
\cmidrule{1-11}
 & Bengtsson et al. (\protect\hyperlink{ref-Bengtsson.2017}{2017}) & 10 & 0.3, 0.64 &  & 0.46, 2.09 &  & 0.55, 1.34 &  & 1.33, 11.05 & \\

\multirow[t]{-2}{*}{\raggedright\arraybackslash $S.~fallax$} & Straková et al. (\protect\hyperlink{ref-Strakova.2010}{2010}) & 4 & 0.07, 0.14 & 0.06, 0.12 & 0.09, 0.2 & 0.08, 0.19 & 0.15, 0.27 & 0.14, 0.26 & 0.52, 1.28 & 0.57, 1.46\\
\cmidrule{1-11}
 & Asada and Warner (\protect\hyperlink{ref-Asada.2005b}{2005}) & 8 & 0.03, 0.05 & 0.03, 0.05 & 0.03, 0.05 & 0.04, 0.06 & 0.12, 0.17 & 0.12, 0.17 & 1.37, 2.47 & 1.5, 2.68\\

 & Bengtsson et al. (\protect\hyperlink{ref-Bengtsson.2017}{2017}) & 16 & 0.03, 0.04 &  & 0.03, 0.05 &  & 0.08, 0.19 &  & 1.1, 2.04 & \\

 & Breeuwer et al. (\protect\hyperlink{ref-Breeuwer.2008}{2008}) & 8 & 0.02, 0.03 & 0.02, 0.03 & 0.02, 0.03 & 0.02, 0.03 & 0.08, 0.11 & 0.08, 0.11 & 1.23, 2.06 & 1.33, 2.2\\

 & Golovatskaya and Nikonova (\protect\hyperlink{ref-Golovatskaya.2017}{2017}) & 2 & 0.03, 0.05 & 0.03, 0.05 & 0.04, 0.05 & 0.04, 0.05 & 0.05, 0.07 & 0.05, 0.07 & 0.75, 0.91 & 0.87, 1.04\\

 & Johnson and Damman (\protect\hyperlink{ref-Johnson.1991}{1991}) & 5 & 0.01, 0.02 & 0.01, 0.02 & 0.01, 0.02 & 0.01, 0.02 & 0.09, 0.09 & 0.09, 0.09 & 1.59, 2.06 & 1.66, 2.08\\

 & Mäkilä et al. (\protect\hyperlink{ref-Makila.2018}{2018}) & 3 & 0.03, 0.03 & 0.03, 0.03 & 0.03, 0.04 & 0.03, 0.04 & 0.11, 0.12 & 0.1, 0.12 & 1.4, 1.55 & 1.49, 1.68\\

 & Straková et al. (\protect\hyperlink{ref-Strakova.2010}{2010}) & 3 & 0.04, 0.05 & 0.04, 0.05 & 0.04, 0.06 & 0.04, 0.06 & 0.11, 0.14 & 0.11, 0.14 & 1.27, 1.64 & 1.38, 1.74\\

 & Szumigalski and Bayley (\protect\hyperlink{ref-Szumigalski.1996}{1996}) & 1 & 0.03, 0.03 & 0.03, 0.03 & 0.04, 0.04 & 0.04, 0.04 & 0.11, 0.11 & 0.11, 0.11 & 1.43, 1.43 & 1.51, 1.51\\

 & Thormann et al. (\protect\hyperlink{ref-Thormann.2001}{2001}) & 1 & 0.05, 0.05 & 0.05, 0.05 & 0.05, 0.05 & 0.06, 0.06 & 0.24, 0.24 & 0.24, 0.24 & 3.32, 3.32 & 3.45, 3.45\\

\multirow[t]{-10}{*}{\raggedright\arraybackslash $S.~fuscum$} & Vitt (\protect\hyperlink{ref-Vitt.1990}{1990}) & 1 & 0.03, 0.03 & 0.03, 0.03 & 0.03, 0.03 & 0.03, 0.03 & 0.1, 0.1 & 0.09, 0.09 & 1.18, 1.18 & 1.3, 1.3\\
\cmidrule{1-11}
$S.~girgensohnii$ & Bengtsson et al. (\protect\hyperlink{ref-Bengtsson.2017}{2017}) & 9 & 0.17, 0.41 &  & 0.25, 0.6 &  & 0.47, 0.73 &  & 1.77, 3.29 & \\
\cmidrule{1-11}
 & Bartsch and Moore (\protect\hyperlink{ref-Bartsch.1985}{1985}) & 2 & 0.05, 0.05 & 0.04, 0.04 & 0.05, 0.06 & 0.04, 0.05 & 0.1, 0.11 & 0.09, 0.11 & 0.25, 0.29 & 0.29, 0.34\\

\multirow[t]{-2}{*}{\raggedright\arraybackslash $S.~lindbergii$} & Bengtsson et al. (\protect\hyperlink{ref-Bengtsson.2017}{2017}) & 10 & 0.1, 0.16 &  & 0.13, 0.2 &  & 0.29, 0.4 &  & 1.09, 2 & \\
\cmidrule{1-11}
 & Bengtsson et al. (\protect\hyperlink{ref-Bengtsson.2017}{2017}) & 27 & 0.06, 0.3 &  & 0.07, 0.57 &  & 0.18, 0.79 &  & 0.66, 3.92 & \\

 & Mäkilä et al. (\protect\hyperlink{ref-Makila.2018}{2018}) & 3 & 0.03, 0.04 & 0.03, 0.04 & 0.03, 0.05 & 0.04, 0.05 & 0.1, 0.13 & 0.1, 0.13 & 0.74, 1.04 & 0.56, 0.82\\

 & Straková et al. (\protect\hyperlink{ref-Strakova.2010}{2010}) & 3 & 0.08, 0.1 & 0.08, 0.1 & 0.1, 0.12 & 0.1, 0.13 & 0.16, 0.19 & 0.16, 0.19 & 0.8, 0.95 & 0.63, 0.77\\

\multirow[t]{-4}{*}{\raggedright\arraybackslash $S.~magellanicum~aggr.$} & Vitt (\protect\hyperlink{ref-Vitt.1990}{1990}) & 1 & 0.05, 0.05 & 0.04, 0.04 & 0.06, 0.06 & 0.05, 0.05 & 0.14, 0.14 & 0.12, 0.12 & 0.76, 0.76 & 0.6, 0.6\\
\cmidrule{1-11}
 & Bengtsson et al. (\protect\hyperlink{ref-Bengtsson.2017}{2017}) & 8 & 0.31, 0.65 &  & 0.41, 0.91 &  & 0.48, 0.79 &  & 0.88, 1.87 & \\

\multirow[t]{-2}{*}{\raggedright\arraybackslash $S.~majus$} & Mäkilä et al. (\protect\hyperlink{ref-Makila.2018}{2018}) & 2 & 0.05, 0.06 & 0.04, 0.05 & 0.06, 0.07 & 0.05, 0.06 & 0.13, 0.14 & 0.12, 0.13 & 0.44, 0.44 & 0.61, 0.64\\
\cmidrule{1-11}
 & Bengtsson et al. (\protect\hyperlink{ref-Bengtsson.2017}{2017}) & 10 & 0.1, 0.2 &  & 0.12, 0.24 &  & 0.29, 0.44 &  & 1.22, 2.19 & \\

 & Scheffer et al. (\protect\hyperlink{ref-Scheffer.2001}{2001}) & 2 & 0.04, 0.04 & 0.03, 0.03 & 0.04, 0.05 & 0.03, 0.04 & 0.12, 0.12 & 0.1, 0.1 & 1.02, 1.08 & 0.66, 0.67\\

\multirow[t]{-3}{*}{\raggedright\arraybackslash $S.~papillosum$} & Straková et al. (\protect\hyperlink{ref-Strakova.2010}{2010}) & 4 & 0.04, 0.09 & 0.04, 0.07 & 0.05, 0.1 & 0.05, 0.08 & 0.1, 0.18 & 0.09, 0.17 & 0.5, 0.91 & 0.43, 0.74\\
\cmidrule{1-11}
 & Bengtsson et al. (\protect\hyperlink{ref-Bengtsson.2017}{2017}) & 10 & 0.26, 0.38 &  & 0.31, 0.45 &  & 0.34, 0.5 &  & 0.75, 1.13 & \\

\multirow[t]{-2}{*}{\raggedright\arraybackslash $S.~rubellum$} & Mäkilä et al. (\protect\hyperlink{ref-Makila.2018}{2018}) & 2 & 0.05, 0.06 & 0.04, 0.04 & 0.07, 0.07 & 0.05, 0.05 & 0.17, 0.17 & 0.15, 0.15 & 0.83, 0.88 & 0.94, 0.98\\
\cmidrule{1-11}
$S.~russowii$ & Straková et al. (\protect\hyperlink{ref-Strakova.2010}{2010}) & 3 & 0.07, 0.11 & 0.06, 0.1 & 0.08, 0.14 & 0.08, 0.13 & 0.18, 0.22 & 0.17, 0.22 & 0.85, 1.17 & 0.71, 1.01\\
\cmidrule{1-11}
$S.~russowii$ and $capillifolium$ & Hagemann and Moroni (\protect\hyperlink{ref-Hagemann.2015}{2015}) & 18 & 0.1, 0.16 & 0.1, 0.16 & 0.65, 0.85 & 0.85, 1.16 & 0.33, 0.38 & 0.33, 0.38 & 3.64, 7.06 & 3.79, 7.21\\
\cmidrule{1-11}
$S.~squarrosum$ & Scheffer et al. (\protect\hyperlink{ref-Scheffer.2001}{2001}) & 2 & 0.04, 0.04 & 0.03, 0.03 & 0.05, 0.05 & 0.04, 0.04 & 0.12, 0.13 & 0.1, 0.11 & 1.07, 1.09 & 0.75, 0.79\\
\cmidrule{1-11}
$S.~tenellum$ & Bengtsson et al. (\protect\hyperlink{ref-Bengtsson.2017}{2017}) & 9 & 0.23, 0.8 &  & 0.29, 1.45 &  & 0.45, 1.1 &  & 1.22, 6.54 & \\
\cmidrule{1-11}
$S.~teres$ & Szumigalski and Bayley (\protect\hyperlink{ref-Szumigalski.1996}{1996}) & 1 & 0.03, 0.03 & 0.03, 0.03 & 0.04, 0.04 & 0.04, 0.04 & 0.13, 0.13 & 0.12, 0.12 & 0.87, 0.87 & 0.71, 0.71\\
\cmidrule{1-11}
$S.~warnstorfii$ & Bengtsson et al. (\protect\hyperlink{ref-Bengtsson.2017}{2017}) & 6 & 0.05, 0.08 &  & 0.06, 0.09 &  & 0.24, 0.29 &  & 1.28, 1.64 & \\
\cmidrule{1-11}
 & Bartsch and Moore (\protect\hyperlink{ref-Bartsch.1985}{1985}) & 6 & 0.04, 0.05 & 0.04, 0.05 & 0.04, 0.06 & 0.04, 0.05 & 0.08, 0.11 & 0.08, 0.11 & 0.13, 0.18 & 0.15, 0.22\\

\multirow[t]{-2}{*}{\raggedright\arraybackslash $Sphagnum$ spec.} & Prevost et al. (\protect\hyperlink{ref-Prevost.1997}{1997}) & 10 & 0.02, 0.03 & 0.02, 0.03 & 0.02, 0.03 & 0.02, 0.03 & 0.03, 0.06 & 0.03, 0.06 & 0.06, 0.17 & 0.08, 0.21\\
\bottomrule
\end{tabular}}
\end{table}



\begin{table}[H]

\caption{\label{tab:sup-out-alpha-2-all-models}Range of average \(\alpha\) (-) for litterbag replicates grouped by species and study as estimated by all models (see Tab. \ref{tab:m-litterbag-synthesis-models}). Ranges were computed on average estimates and therefore do not consider the uncertainty of decomposition rates for individual litterbag replicates.}
\centering
\resizebox{\linewidth}{!}{
\begin{tabular}[t]{llrllllllll}
\toprule
\multicolumn{1}{c}{ } & \multicolumn{1}{c}{ } & \multicolumn{1}{c}{ } & \multicolumn{8}{c}{$\alpha$ (-)} \\
\cmidrule(l{3pt}r{3pt}){4-11}
Taxon & Study & Sample size & model 1-1 & model 1-2 & model 1-3 & model 1-4 & model 2-1 & model 2-2 & model 2-3 & model 2-4\\
\midrule
 & Bengtsson et al. (\protect\hyperlink{ref-Bengtsson.2017}{2017}) & 10 &  &  & 1.87, 1.94 &  &  &  & 2.63, 2.97 & \\

 & Golovatskaya and Nikonova (\protect\hyperlink{ref-Golovatskaya.2017}{2017}) & 2 &  &  & 1.96, 2.02 & 2.44, 2.54 &  &  & 3.87, 4.41 & 4.61, 5.3\\

 & Mäkilä et al. (\protect\hyperlink{ref-Makila.2018}{2018}) & 2 &  &  & 2.11, 2.12 & 2.88, 2.9 &  &  & 7.56, 8.2 & 8.37, 9.17\\

 & Straková et al. (\protect\hyperlink{ref-Strakova.2010}{2010}) & 9 &  &  & 1.94, 1.98 & 2.49, 2.58 &  &  & 4.82, 5.93 & 5.28, 6.64\\

\multirow[t]{-5}{*}{\raggedright\arraybackslash $S.~angustifolium$} & Vitt (\protect\hyperlink{ref-Vitt.1990}{1990}) & 1 &  &  & 2.12, 2.12 & 2.93, 2.93 &  &  & 7.77, 7.77 & 8.82, 8.82\\
\cmidrule{1-11}
$S.~auriculatum$ & Trinder et al. (\protect\hyperlink{ref-Trinder.2008}{2008}) & 3 &  &  & 2.21, 2.21 & 3, 3.01 &  &  & 11.41, 11.46 & 15.03, 15.2\\
\cmidrule{1-11}
 & Bengtsson et al. (\protect\hyperlink{ref-Bengtsson.2017}{2017}) & 9 &  &  & 2.02, 2.11 &  &  &  & 5.36, 7.14 & \\

 & Breeuwer et al. (\protect\hyperlink{ref-Breeuwer.2008}{2008}) & 8 &  &  & 2.18, 2.19 & 2.99, 3.02 &  &  & 17.78, 25.05 & 19.07, 27.31\\

 & Mäkilä et al. (\protect\hyperlink{ref-Makila.2018}{2018}) & 2 &  &  & 2.19, 2.2 & 3.06, 3.07 &  &  & 14.42, 16.4 & 15.56, 17.63\\

\multirow[t]{-4}{*}{\raggedright\arraybackslash $S.~balticum$} & Straková et al. (\protect\hyperlink{ref-Strakova.2010}{2010}) & 2 &  &  & 2.17, 2.18 & 3, 3.02 &  &  & 14.32, 17.86 & 15.65, 19.85\\
\cmidrule{1-11}
$S.~capillifolium$ & Bengtsson et al. (\protect\hyperlink{ref-Bengtsson.2017}{2017}) & 10 &  &  & 2.22, 2.24 &  &  &  & 9.53, 12.98 & \\
\cmidrule{1-11}
$S.~contortum$ & Bengtsson et al. (\protect\hyperlink{ref-Bengtsson.2017}{2017}) & 6 &  &  & 2.24, 2.24 &  &  &  & 9.33, 10.16 & \\
\cmidrule{1-11}
 & Bengtsson et al. (\protect\hyperlink{ref-Bengtsson.2017}{2017}) & 10 &  &  & 1.66, 1.74 &  &  &  & 3.44, 4.34 & \\

\multirow[t]{-2}{*}{\raggedright\arraybackslash $S.~cuspidatum$} & Johnson and Damman (\protect\hyperlink{ref-Johnson.1991}{1991}) & 5 &  &  & 1.97, 1.97 & 3.1, 3.15 &  &  & 12.8, 18.58 & 13.4, 19.36\\
\cmidrule{1-11}
 & Bengtsson et al. (\protect\hyperlink{ref-Bengtsson.2017}{2017}) & 10 &  &  & 1.98, 2.11 &  &  &  & 3.27, 4.16 & \\

\multirow[t]{-2}{*}{\raggedright\arraybackslash $S.~fallax$} & Straková et al. (\protect\hyperlink{ref-Strakova.2010}{2010}) & 4 &  &  & 2.13, 2.15 & 2.83, 2.88 &  &  & 6.27, 7.87 & 7.98, 9.86\\
\cmidrule{1-11}
 & Asada and Warner (\protect\hyperlink{ref-Asada.2005b}{2005}) & 8 &  &  & 2.2, 2.21 & 2.96, 2.99 &  &  & 14.93, 22.32 & 15.11, 22.54\\

 & Bengtsson et al. (\protect\hyperlink{ref-Bengtsson.2017}{2017}) & 16 &  &  & 2.18, 2.19 &  &  &  & 20.22, 28.03 & \\

 & Breeuwer et al. (\protect\hyperlink{ref-Breeuwer.2008}{2008}) & 8 &  &  & 2.19, 2.2 & 2.99, 3.01 &  &  & 30.73, 43.65 & 30.99, 44.18\\

 & Golovatskaya and Nikonova (\protect\hyperlink{ref-Golovatskaya.2017}{2017}) & 2 &  &  & 2.09, 2.11 & 2.78, 2.81 &  &  & 25.92, 29.26 & 26.6, 30.05\\

 & Johnson and Damman (\protect\hyperlink{ref-Johnson.1991}{1991}) & 5 &  &  & 2.2, 2.21 & 3, 3.02 &  &  & 34.47, 39.11 & 34.19, 38.88\\

 & Mäkilä et al. (\protect\hyperlink{ref-Makila.2018}{2018}) & 3 &  &  & 2.2, 2.21 & 2.99, 3.01 &  &  & 22.44, 25.95 & 22.59, 26.11\\

 & Straková et al. (\protect\hyperlink{ref-Strakova.2010}{2010}) & 3 &  &  & 2.17, 2.19 & 2.92, 2.96 &  &  & 17.84, 21.85 & 18.06, 21.95\\

 & Szumigalski and Bayley (\protect\hyperlink{ref-Szumigalski.1996}{1996}) & 1 &  &  & 2.19, 2.19 & 2.96, 2.96 &  &  & 23.08, 23.08 & 23.24, 23.24\\

 & Thormann et al. (\protect\hyperlink{ref-Thormann.2001}{2001}) & 1 &  &  & 2.18, 2.18 & 2.95, 2.95 &  &  & 16.42, 16.42 & 16.73, 16.73\\

\multirow[t]{-10}{*}{\raggedright\arraybackslash $S.~fuscum$} & Vitt (\protect\hyperlink{ref-Vitt.1990}{1990}) & 1 &  &  & 2.21, 2.21 & 3.02, 3.02 &  &  & 28.54, 28.54 & 28.95, 28.95\\
\cmidrule{1-11}
$S.~girgensohnii$ & Bengtsson et al. (\protect\hyperlink{ref-Bengtsson.2017}{2017}) & 9 &  &  & 2.13, 2.22 &  &  &  & 4.77, 6.45 & \\
\cmidrule{1-11}
 & Bartsch and Moore (\protect\hyperlink{ref-Bartsch.1985}{1985}) & 2 &  &  & 2.21, 2.22 & 3.01, 3.01 &  &  & 11.69, 12.14 & 16.6, 17.55\\

\multirow[t]{-2}{*}{\raggedright\arraybackslash $S.~lindbergii$} & Bengtsson et al. (\protect\hyperlink{ref-Bengtsson.2017}{2017}) & 10 &  &  & 2.19, 2.23 &  &  &  & 7.16, 9.03 & \\
\cmidrule{1-11}
 & Bengtsson et al. (\protect\hyperlink{ref-Bengtsson.2017}{2017}) & 27 &  &  & 2.12, 2.15 &  &  &  & 8.33, 11.24 & \\

 & Mäkilä et al. (\protect\hyperlink{ref-Makila.2018}{2018}) & 3 &  &  & 2.21, 2.22 & 3.03, 3.04 &  &  & 16.22, 20.47 & 14.95, 18.52\\

 & Straková et al. (\protect\hyperlink{ref-Strakova.2010}{2010}) & 3 &  &  & 2.11, 2.12 & 2.78, 2.81 &  &  & 9.77, 11 & 8.78, 9.82\\

\multirow[t]{-4}{*}{\raggedright\arraybackslash $S.~magellanicum~aggr.$} & Vitt (\protect\hyperlink{ref-Vitt.1990}{1990}) & 1 &  &  & 2.2, 2.2 & 3, 3 &  &  & 14.54, 14.54 & 13.63, 13.63\\
\cmidrule{1-11}
 & Bengtsson et al. (\protect\hyperlink{ref-Bengtsson.2017}{2017}) & 8 &  &  & 1.72, 1.78 &  &  &  & 3.06, 3.68 & \\

\multirow[t]{-2}{*}{\raggedright\arraybackslash $S.~majus$} & Mäkilä et al. (\protect\hyperlink{ref-Makila.2018}{2018}) & 2 &  &  & 1.98, 1.99 & 3.03, 3.06 &  &  & 8.89, 9.32 & 13.34, 14.53\\
\cmidrule{1-11}
 & Bengtsson et al. (\protect\hyperlink{ref-Bengtsson.2017}{2017}) & 10 &  &  & 2.17, 2.21 &  &  &  & 7.69, 10.28 & \\

 & Scheffer et al. (\protect\hyperlink{ref-Scheffer.2001}{2001}) & 2 &  &  & 2.22, 2.22 & 2.98, 2.99 &  &  & 15.83, 16.39 & 18.4, 18.74\\

\multirow[t]{-3}{*}{\raggedright\arraybackslash $S.~papillosum$} & Straková et al. (\protect\hyperlink{ref-Strakova.2010}{2010}) & 4 &  &  & 2.19, 2.21 & 2.89, 2.93 &  &  & 12.56, 15.67 & 12.54, 15.47\\
\cmidrule{1-11}
 & Bengtsson et al. (\protect\hyperlink{ref-Bengtsson.2017}{2017}) & 10 &  &  & 1.88, 1.92 &  &  &  & 4.24, 4.79 & \\

\multirow[t]{-2}{*}{\raggedright\arraybackslash $S.~rubellum$} & Mäkilä et al. (\protect\hyperlink{ref-Makila.2018}{2018}) & 2 &  &  & 2.11, 2.11 & 3.07, 3.07 &  &  & 10.61, 11.04 & 12.18, 12.6\\
\cmidrule{1-11}
$S.~russowii$ & Straková et al. (\protect\hyperlink{ref-Strakova.2010}{2010}) & 3 &  &  & 2.14, 2.18 & 2.84, 2.9 &  &  & 8.47, 10.66 & 8.07, 9.91\\
\cmidrule{1-11}
$S.~russowii$ and $capillifolium$ & Hagemann and Moroni (\protect\hyperlink{ref-Hagemann.2015}{2015}) & 18 &  &  & 8.42, 10.63 & 9.2, 11.84 &  &  & 11.51, 14.48 & 11.59, 14.58\\
\cmidrule{1-11}
$S.~squarrosum$ & Scheffer et al. (\protect\hyperlink{ref-Scheffer.2001}{2001}) & 2 &  &  & 2.21, 2.22 & 3.02, 3.02 &  &  & 14.16, 14.28 & 16.37, 16.7\\
\cmidrule{1-11}
$S.~tenellum$ & Bengtsson et al. (\protect\hyperlink{ref-Bengtsson.2017}{2017}) & 9 &  &  & 1.98, 2.05 &  &  &  & 4.15, 5.33 & \\
\cmidrule{1-11}
$S.~teres$ & Szumigalski and Bayley (\protect\hyperlink{ref-Szumigalski.1996}{1996}) & 1 &  &  & 2.24, 2.24 & 3.07, 3.07 &  &  & 16.46, 16.46 & 15.73, 15.73\\
\cmidrule{1-11}
$S.~warnstorfii$ & Bengtsson et al. (\protect\hyperlink{ref-Bengtsson.2017}{2017}) & 6 &  &  & 2.22, 2.23 &  &  &  & 10.62, 13.31 & \\
\cmidrule{1-11}
 & Bartsch and Moore (\protect\hyperlink{ref-Bartsch.1985}{1985}) & 6 &  &  & 2.21, 2.22 & 3, 3.03 &  &  & 10.66, 11.77 & 13.6, 15.34\\

\multirow[t]{-2}{*}{\raggedright\arraybackslash $Sphagnum$ spec.} & Prevost et al. (\protect\hyperlink{ref-Prevost.1997}{1997}) & 10 &  &  & 2.23, 2.25 & 3.07, 3.11 &  &  & 14.06, 16.38 & 16.17, 19.3\\
\bottomrule
\end{tabular}}
\end{table}

\clearpage

















\begin{figure}[H]

{\centering \includegraphics[width=1\linewidth]{figures/leaching_plot_1_1} 

}

\caption{Estimated initial leaching losses (a), and decomposition rates (c) grouped by species and study for model 1-1. Points represent averages and error bars 95\% confidence intervals. The study is indicated by numbers on the x axis: (1) Asada and Warner (\protect\hyperlink{ref-Asada.2005b}{2005}), (2) Bartsch and Moore (\protect\hyperlink{ref-Bartsch.1985}{1985}), (3) Bengtsson et al. (\protect\hyperlink{ref-Bengtsson.2017}{2017}), (4) Breeuwer et al. (\protect\hyperlink{ref-Breeuwer.2008}{2008}), (5) Golovatskaya and Nikonova (\protect\hyperlink{ref-Golovatskaya.2017}{2017}), (6) Hagemann and Moroni (\protect\hyperlink{ref-Hagemann.2015}{2015}), (7) Johnson and Damman (\protect\hyperlink{ref-Johnson.1991}{1991}), (8) Mäkilä et al. (\protect\hyperlink{ref-Makila.2018}{2018}), (9) Prevost et al. (\protect\hyperlink{ref-Prevost.1997}{1997}), (10) Scheffer et al. (\protect\hyperlink{ref-Scheffer.2001}{2001}), (11) Straková et al. (\protect\hyperlink{ref-Strakova.2010}{2010}), (12) Szumigalski and Bayley (\protect\hyperlink{ref-Szumigalski.1996}{1996}), (13) Thormann et al. (\protect\hyperlink{ref-Thormann.2001}{2001}), (14) Trinder et al. (\protect\hyperlink{ref-Trinder.2008}{2008}), (15) Vitt (\protect\hyperlink{ref-Vitt.1990}{1990}). \emph{Sphagnum} spec. are samples that have been identified only to the genus level.}\label{fig:sup-out-mm-p5-1-1}
\end{figure}
\begin{figure}[H]

{\centering \includegraphics[width=1\linewidth]{figures/leaching_plot_1_2} 

}

\caption{Estimated initial leaching losses (a), and decomposition rates (c) grouped by species and study for model 1-2. Points represent averages and error bars 95\% confidence intervals. The study is indicated by numbers on the x axis: (1) Asada and Warner (\protect\hyperlink{ref-Asada.2005b}{2005}), (2) Bartsch and Moore (\protect\hyperlink{ref-Bartsch.1985}{1985}), (3) Breeuwer et al. (\protect\hyperlink{ref-Breeuwer.2008}{2008}), (4) Golovatskaya and Nikonova (\protect\hyperlink{ref-Golovatskaya.2017}{2017}), (5) Hagemann and Moroni (\protect\hyperlink{ref-Hagemann.2015}{2015}), (6) Johnson and Damman (\protect\hyperlink{ref-Johnson.1991}{1991}), (7) Mäkilä et al. (\protect\hyperlink{ref-Makila.2018}{2018}), (8) Prevost et al. (\protect\hyperlink{ref-Prevost.1997}{1997}), (9) Scheffer et al. (\protect\hyperlink{ref-Scheffer.2001}{2001}), (10) Straková et al. (\protect\hyperlink{ref-Strakova.2010}{2010}), (11) Szumigalski and Bayley (\protect\hyperlink{ref-Szumigalski.1996}{1996}), (12) Thormann et al. (\protect\hyperlink{ref-Thormann.2001}{2001}), (13) Trinder et al. (\protect\hyperlink{ref-Trinder.2008}{2008}), (14) Vitt (\protect\hyperlink{ref-Vitt.1990}{1990}). \emph{Sphagnum} spec. are samples that have been identified only to the genus level.}\label{fig:sup-out-mm-p5-1-2}
\end{figure}
\begin{figure}[H]

{\centering \includegraphics[width=1\linewidth]{figures/leaching_plot_1_3} 

}

\caption{Estimated initial leaching losses (a), the parameter controlling a decrease of decomposition rates over time (\(\alpha\)) (b), and decomposition rates (c) grouped by species and study for model 1-3. Points represent averages and error bars 95\% confidence intervals. The study is indicated by numbers on the x axis: (1) Asada and Warner (\protect\hyperlink{ref-Asada.2005b}{2005}), (2) Bartsch and Moore (\protect\hyperlink{ref-Bartsch.1985}{1985}), (3) Bengtsson et al. (\protect\hyperlink{ref-Bengtsson.2017}{2017}), (4) Breeuwer et al. (\protect\hyperlink{ref-Breeuwer.2008}{2008}), (5) Golovatskaya and Nikonova (\protect\hyperlink{ref-Golovatskaya.2017}{2017}), (6) Hagemann and Moroni (\protect\hyperlink{ref-Hagemann.2015}{2015}), (7) Johnson and Damman (\protect\hyperlink{ref-Johnson.1991}{1991}), (8) Mäkilä et al. (\protect\hyperlink{ref-Makila.2018}{2018}), (9) Prevost et al. (\protect\hyperlink{ref-Prevost.1997}{1997}), (10) Scheffer et al. (\protect\hyperlink{ref-Scheffer.2001}{2001}), (11) Straková et al. (\protect\hyperlink{ref-Strakova.2010}{2010}), (12) Szumigalski and Bayley (\protect\hyperlink{ref-Szumigalski.1996}{1996}), (13) Thormann et al. (\protect\hyperlink{ref-Thormann.2001}{2001}), (14) Trinder et al. (\protect\hyperlink{ref-Trinder.2008}{2008}), (15) Vitt (\protect\hyperlink{ref-Vitt.1990}{1990}). \emph{Sphagnum} spec. are samples that have been identified only to the genus level.}\label{fig:sup-out-mm-p5-1-3}
\end{figure}
\begin{figure}[H]

{\centering \includegraphics[width=1\linewidth]{figures/leaching_plot_1_4} 

}

\caption{Estimated initial leaching losses (a), the parameter controlling a decrease of decomposition rates over time (\(\alpha\)) (b), and decomposition rates (c) grouped by species and study for model 1-4. Points represent averages and error bars 95\% confidence intervals. The study is indicated by numbers on the x axis: (1) Asada and Warner (\protect\hyperlink{ref-Asada.2005b}{2005}), (2) Bartsch and Moore (\protect\hyperlink{ref-Bartsch.1985}{1985}), (3) Breeuwer et al. (\protect\hyperlink{ref-Breeuwer.2008}{2008}), (4) Golovatskaya and Nikonova (\protect\hyperlink{ref-Golovatskaya.2017}{2017}), (5) Hagemann and Moroni (\protect\hyperlink{ref-Hagemann.2015}{2015}), (6) Johnson and Damman (\protect\hyperlink{ref-Johnson.1991}{1991}), (7) Mäkilä et al. (\protect\hyperlink{ref-Makila.2018}{2018}), (8) Prevost et al. (\protect\hyperlink{ref-Prevost.1997}{1997}), (9) Scheffer et al. (\protect\hyperlink{ref-Scheffer.2001}{2001}), (10) Straková et al. (\protect\hyperlink{ref-Strakova.2010}{2010}), (11) Szumigalski and Bayley (\protect\hyperlink{ref-Szumigalski.1996}{1996}), (12) Thormann et al. (\protect\hyperlink{ref-Thormann.2001}{2001}), (13) Trinder et al. (\protect\hyperlink{ref-Trinder.2008}{2008}), (14) Vitt (\protect\hyperlink{ref-Vitt.1990}{1990}). \emph{Sphagnum} spec. are samples that have been identified only to the genus level.}\label{fig:sup-out-mm-p5-1-4}
\end{figure}
\begin{figure}[H]

{\centering \includegraphics[width=1\linewidth]{figures/leaching_plot_1_5} 

}

\caption{Estimated decomposition rates grouped by species and study for model 2-1. Points represent averages and error bars 95\% confidence intervals. The study is indicated by numbers on the x axis: (1) Asada and Warner (\protect\hyperlink{ref-Asada.2005b}{2005}), (2) Bartsch and Moore (\protect\hyperlink{ref-Bartsch.1985}{1985}), (3) Bengtsson et al. (\protect\hyperlink{ref-Bengtsson.2017}{2017}), (4) Breeuwer et al. (\protect\hyperlink{ref-Breeuwer.2008}{2008}), (5) Golovatskaya and Nikonova (\protect\hyperlink{ref-Golovatskaya.2017}{2017}), (6) Hagemann and Moroni (\protect\hyperlink{ref-Hagemann.2015}{2015}), (7) Johnson and Damman (\protect\hyperlink{ref-Johnson.1991}{1991}), (8) Mäkilä et al. (\protect\hyperlink{ref-Makila.2018}{2018}), (9) Prevost et al. (\protect\hyperlink{ref-Prevost.1997}{1997}), (10) Scheffer et al. (\protect\hyperlink{ref-Scheffer.2001}{2001}), (11) Straková et al. (\protect\hyperlink{ref-Strakova.2010}{2010}), (12) Szumigalski and Bayley (\protect\hyperlink{ref-Szumigalski.1996}{1996}), (13) Thormann et al. (\protect\hyperlink{ref-Thormann.2001}{2001}), (14) Trinder et al. (\protect\hyperlink{ref-Trinder.2008}{2008}), (15) Vitt (\protect\hyperlink{ref-Vitt.1990}{1990}). \emph{Sphagnum} spec. are samples that have been identified only to the genus level.}\label{fig:sup-out-mm-p5-1-5}
\end{figure}
\begin{figure}[H]

{\centering \includegraphics[width=1\linewidth]{figures/leaching_plot_1_6} 

}

\caption{Estimated decomposition rates grouped by species and study for model 2-2. Points represent averages and error bars 95\% confidence intervals. The study is indicated by numbers on the x axis: (1) Asada and Warner (\protect\hyperlink{ref-Asada.2005b}{2005}), (2) Bartsch and Moore (\protect\hyperlink{ref-Bartsch.1985}{1985}), (3) Breeuwer et al. (\protect\hyperlink{ref-Breeuwer.2008}{2008}), (4) Golovatskaya and Nikonova (\protect\hyperlink{ref-Golovatskaya.2017}{2017}), (5) Hagemann and Moroni (\protect\hyperlink{ref-Hagemann.2015}{2015}), (6) Johnson and Damman (\protect\hyperlink{ref-Johnson.1991}{1991}), (7) Mäkilä et al. (\protect\hyperlink{ref-Makila.2018}{2018}), (8) Prevost et al. (\protect\hyperlink{ref-Prevost.1997}{1997}), (9) Scheffer et al. (\protect\hyperlink{ref-Scheffer.2001}{2001}), (10) Straková et al. (\protect\hyperlink{ref-Strakova.2010}{2010}), (11) Szumigalski and Bayley (\protect\hyperlink{ref-Szumigalski.1996}{1996}), (12) Thormann et al. (\protect\hyperlink{ref-Thormann.2001}{2001}), (13) Trinder et al. (\protect\hyperlink{ref-Trinder.2008}{2008}), (14) Vitt (\protect\hyperlink{ref-Vitt.1990}{1990}). \emph{Sphagnum} spec. are samples that have been identified only to the genus level.}\label{fig:sup-out-mm-p5-1-6}
\end{figure}
\begin{figure}[H]

{\centering \includegraphics[width=1\linewidth]{figures/leaching_plot_1_7} 

}

\caption{Estimated parameter controlling a decrease of decomposition rates over time (\(\alpha\)) (a), and decomposition rates (b) grouped by species and study for model 2-3. Points represent averages and error bars 95\% confidence intervals. The study is indicated by numbers on the x axis: (1) Asada and Warner (\protect\hyperlink{ref-Asada.2005b}{2005}), (2) Bartsch and Moore (\protect\hyperlink{ref-Bartsch.1985}{1985}), (3) Bengtsson et al. (\protect\hyperlink{ref-Bengtsson.2017}{2017}), (4) Breeuwer et al. (\protect\hyperlink{ref-Breeuwer.2008}{2008}), (5) Golovatskaya and Nikonova (\protect\hyperlink{ref-Golovatskaya.2017}{2017}), (6) Hagemann and Moroni (\protect\hyperlink{ref-Hagemann.2015}{2015}), (7) Johnson and Damman (\protect\hyperlink{ref-Johnson.1991}{1991}), (8) Mäkilä et al. (\protect\hyperlink{ref-Makila.2018}{2018}), (9) Prevost et al. (\protect\hyperlink{ref-Prevost.1997}{1997}), (10) Scheffer et al. (\protect\hyperlink{ref-Scheffer.2001}{2001}), (11) Straková et al. (\protect\hyperlink{ref-Strakova.2010}{2010}), (12) Szumigalski and Bayley (\protect\hyperlink{ref-Szumigalski.1996}{1996}), (13) Thormann et al. (\protect\hyperlink{ref-Thormann.2001}{2001}), (14) Trinder et al. (\protect\hyperlink{ref-Trinder.2008}{2008}), (15) Vitt (\protect\hyperlink{ref-Vitt.1990}{1990}). \emph{Sphagnum} spec. are samples that have been identified only to the genus level.}\label{fig:sup-out-mm-p5-1-7}
\end{figure}
\begin{figure}[H]

{\centering \includegraphics[width=1\linewidth]{figures/leaching_plot_1_8} 

}

\caption{Estimated parameter controlling a decrease of decomposition rates over time (\(\alpha\)) (a), and decomposition rates (b) grouped by species and study for model 2-4. Points represent averages and error bars 95\% confidence intervals. The study is indicated by numbers on the x axis: (1) Asada and Warner (\protect\hyperlink{ref-Asada.2005b}{2005}), (2) Bartsch and Moore (\protect\hyperlink{ref-Bartsch.1985}{1985}), (3) Breeuwer et al. (\protect\hyperlink{ref-Breeuwer.2008}{2008}), (4) Golovatskaya and Nikonova (\protect\hyperlink{ref-Golovatskaya.2017}{2017}), (5) Hagemann and Moroni (\protect\hyperlink{ref-Hagemann.2015}{2015}), (6) Johnson and Damman (\protect\hyperlink{ref-Johnson.1991}{1991}), (7) Mäkilä et al. (\protect\hyperlink{ref-Makila.2018}{2018}), (8) Prevost et al. (\protect\hyperlink{ref-Prevost.1997}{1997}), (9) Scheffer et al. (\protect\hyperlink{ref-Scheffer.2001}{2001}), (10) Straková et al. (\protect\hyperlink{ref-Strakova.2010}{2010}), (11) Szumigalski and Bayley (\protect\hyperlink{ref-Szumigalski.1996}{1996}), (12) Thormann et al. (\protect\hyperlink{ref-Thormann.2001}{2001}), (13) Trinder et al. (\protect\hyperlink{ref-Trinder.2008}{2008}), (14) Vitt (\protect\hyperlink{ref-Vitt.1990}{1990}). \emph{Sphagnum} spec. are samples that have been identified only to the genus level.}\label{fig:sup-out-mm-p5-1-8}
\end{figure}

\hypertarget{sup-8}{%
\section{Fit of all models to the remaining masses reported in the synthesized litterbag studies}\label{sup-8}}



\begin{figure}[H]

{\centering \includegraphics[width=1\linewidth]{figures/leaching_plot_y_vs_yrep_m} 

}

\caption{Measured versus predicted remaining masses in litterbag studies from all models computed in our study. Points are average values and error bars are 95\% prediction intervals. For a description of each model, see Tab. \ref{tab:m-litterbag-synthesis-models}.}\label{fig:sup-out-sdm-all-models-p1}
\end{figure}

\hypertarget{sup-9}{%
\section{Litterbag experiments with bad fit under model 1-3}\label{sup-9}}



\begin{figure}[H]

{\centering \includegraphics[width=1\linewidth]{figures/leaching_plot_outlier_1_3} 

}

\caption{Remaining masses predicted by model 1-3 (shaded area) or as reported on average in available litterbag studies (red lines) during the litterbag experiments for outlier litterbag replicates. There is a panel for each litterbag replicate, identified by \texttt{id\_sample\_incubation\_start} in the Peatland Decomposition Database (first row), the species (second row), and the study the data are from (third) row. The studies are: (1) Bengtsson et al. (\protect\hyperlink{ref-Bengtsson.2017}{2017}), (2) Breeuwer et al. (\protect\hyperlink{ref-Breeuwer.2008}{2008}), (3) Golovatskaya and Nikonova (\protect\hyperlink{ref-Golovatskaya.2017}{2017}), (4) Hagemann and Moroni (\protect\hyperlink{ref-Hagemann.2015}{2015}), (5) Prevost et al. (\protect\hyperlink{ref-Prevost.1997}{1997}), (6) Scheffer et al. (\protect\hyperlink{ref-Scheffer.2001}{2001}), (7) Straková et al. (\protect\hyperlink{ref-Strakova.2010}{2010}), (8) Thormann et al. (\protect\hyperlink{ref-Thormann.2001}{2001}), (9) Trinder et al. (\protect\hyperlink{ref-Trinder.2008}{2008}). Outlier litterbag replicates are defined as those replicates where the average measured remaining mass significantly different from the average predicted remaining mass (\(\alpha = 0.99\)).}\label{fig:sup-out-sdm-mm36-1-outlier-decomposition-trajectory-p1}
\end{figure}

\hypertarget{sup-10}{%
\section{Effects of considering or ignoring initial leaching losses on decomposition rate estimates for model 1-1 versus model 2-1 and for all species}\label{sup-10}}



\begin{figure}[H]

{\centering \includegraphics[width=0.7\linewidth]{figures/leaching_plot_2_fit_7_and_fit_3} 

}

\caption{(a) Decomposition rate estimates, either considering leaching (black) or ignoring leaching (grey) versus average initial leaching losses estimated by the model considering initial leaching losses. Points are average estimates and error bars are 95\% prediction intervals. (b) Standard deviation of decomposition rate estimates, either considering leaching (black) or ignoring leaching (grey) versus average initial leaching losses estimated by the model considering initial leaching losses.}\label{fig:sup-out-mm27-1-mm28-1-p3}
\end{figure}

\hypertarget{references}{%
\section*{References}\label{references}}
\addcontentsline{toc}{section}{References}

\hypertarget{refs}{}
\begin{CSLReferences}{0}{0}
\leavevmode\vadjust pre{\hypertarget{ref-Asada.2005b}{}}%
Asada, T. and Warner, B. G.: Surface {Peat Mass} and {Carbon Balance} in a {Hypermaritime Peatland}, Soil Science Society of America Journal, 69, 549--562, \url{https://doi.org/10.2136/sssaj2005.0549}, 2005.

\leavevmode\vadjust pre{\hypertarget{ref-Bartsch.1985}{}}%
Bartsch, I. and Moore, T. R.: A preliminary investigation of primary production and decomposition in four peatlands near {Schefferville}, {Qu{é}bec}, Canadian Journal of Botany, 63, 1241--1248, \url{https://doi.org/10.1139/b85-171}, 1985.

\leavevmode\vadjust pre{\hypertarget{ref-Bengtsson.2017}{}}%
Bengtsson, F., Granath, G., and Rydin, H.: Data from: {Photosynthesis}, growth, and decay traits in {\emph{Sphagnum}} -- a multispecies comparison, 83493 bytes, \url{https://doi.org/10.5061/DRYAD.62054}, 2017.

\leavevmode\vadjust pre{\hypertarget{ref-Breeuwer.2008}{}}%
Breeuwer, A., Heijmans, M., Robroek, B. J. M., Limpens, J., and Berendse, F.: The effect of increased temperature and nitrogen deposition on decomposition in bogs, Oikos, 117, 1258--1268, \url{https://doi.org/10.1111/j.0030-1299.2008.16518.x}, 2008.

\leavevmode\vadjust pre{\hypertarget{ref-Burkner.2023}{}}%
Bürkner, P.-C., Gabry, J., Kay, M., and Vehtari, A.: {posterior}: {Tools} for working with posterior distributions, 2023.

\leavevmode\vadjust pre{\hypertarget{ref-Frolking.2010}{}}%
Frolking, S., Roulet, N. T., Tuittila, E., Bubier, J. L., Quillet, A., Talbot, J., and Richard, P. J. H.: A new model of {Holocene} peatland net primary production, decomposition, water balance, and peat accumulation, Earth System Dynamics, 1, 1--21, \url{https://doi.org/10.5194/esd-1-1-2010}, 2010.

\leavevmode\vadjust pre{\hypertarget{ref-Gabry.2022}{}}%
Gabry, J. and Mahr, T.: {bayesplot}: {Plotting} for bayesian models, 2022.

\leavevmode\vadjust pre{\hypertarget{ref-Golovatskaya.2017}{}}%
Golovatskaya, E. A. and Nikonova, L. G.: The influence of the bog water level on the transformation of sphagnum mosses in peat soils of oligotrophic bogs, Eurasian Soil Science, 50, 580--588, \url{https://doi.org/10.1134/S1064229317030036}, 2017.

\leavevmode\vadjust pre{\hypertarget{ref-Hagemann.2015}{}}%
Hagemann, U. and Moroni, M. T.: Moss and lichen decomposition in old-growth and harvested high-boreal forests estimated using the litterbag and minicontainer methods, Soil Biology and Biochemistry, 87, 10--24, \url{https://doi.org/10.1016/j.soilbio.2015.04.002}, 2015.

\leavevmode\vadjust pre{\hypertarget{ref-Johnson.1991}{}}%
Johnson, L. C. and Damman, A. W. H.: Species-controlled {\emph{Sphagnum}} decay on a south {Swedish} raised bog, Oikos, 61, 234, \url{https://doi.org/10.2307/3545341}, 1991.

\leavevmode\vadjust pre{\hypertarget{ref-Kay.2022}{}}%
Kay, M.: {tidybayes}: {Tidy} data and geoms for {Bayesian} models, \url{https://doi.org/10.5281/zenodo.1308151}, 2022.

\leavevmode\vadjust pre{\hypertarget{ref-Makila.2018}{}}%
Mäkilä, M., Säävuori, H., Grundström, A., and Suomi, T.: {\emph{Sphagnum}} decay patterns and bog microtopography in south-eastern {Finland}, Mires and Peat, 1--12, \url{https://doi.org/10.19189/MaP.2017.OMB.283}, 2018.

\leavevmode\vadjust pre{\hypertarget{ref-Moore.2007}{}}%
Moore, T. R., Bubier, J. L., and Bledzki, L.: Litter decomposition in temperate peatland ecosystems: {The} effect of substrate and site, Ecosystems, 10, 949--963, \url{https://doi.org/10.1007/s10021-007-9064-5}, 2007.

\leavevmode\vadjust pre{\hypertarget{ref-Pedersen.2020}{}}%
Pedersen, T. L.: {patchwork}: {The} composer of plots, 2020.

\leavevmode\vadjust pre{\hypertarget{ref-Prevost.1997}{}}%
Prevost, M., Belleau, P., and Plamondon, A. P.: Substrate conditions in a treed peatland: {Responses} to drainage, {É}coscience, 4, 543--554, \url{https://doi.org/10.1080/11956860.1997.11682434}, 1997.

\leavevmode\vadjust pre{\hypertarget{ref-RCoreTeam.2022}{}}%
R Core Team: R: {A} language and environment for statistical computing, Manual, R Foundation for Statistical Computing, Vienna, Austria, 2022.

\leavevmode\vadjust pre{\hypertarget{ref-Scheffer.2001}{}}%
Scheffer, R. A., Van Logtestijn, R. S. P., and Verhoeven, J. T. A.: Decomposition of {\emph{Carex}} and {\emph{Sphagnum}} litter in two mesotrophic fens differing in dominant plant species, Oikos, 92, 44--54, \url{https://doi.org/10.1034/j.1600-0706.2001.920106.x}, 2001.

\leavevmode\vadjust pre{\hypertarget{ref-Strakova.2010}{}}%
Straková, P., Anttila, J., Spetz, P., Kitunen, V., Tapanila, T., and Laiho, R.: Litter quality and its response to water level drawdown in boreal peatlands at plant species and community level, Plant and Soil, 335, 501--520, \url{https://doi.org/10.1007/s11104-010-0447-6}, 2010.

\leavevmode\vadjust pre{\hypertarget{ref-Szumigalski.1996}{}}%
Szumigalski, A. R. and Bayley, S. E.: Decomposition along a bog to rich fen gradient in central {Alberta}, {Canada}, Canadian Journal of Botany, 74, 573--581, \url{https://doi.org/10.1139/b96-073}, 1996.

\leavevmode\vadjust pre{\hypertarget{ref-Thormann.2001}{}}%
Thormann, M. N., Bayley, S. E., and Currah, R. S.: Comparison of decomposition of belowground and aboveground plant litters in peatlands of boreal {Alberta}, {Canada}, Canadian Journal of Botany, 79, 9--22, \url{https://doi.org/10.1139/b00-138}, 2001.

\leavevmode\vadjust pre{\hypertarget{ref-Trinder.2008}{}}%
Trinder, C. J., Johnson, D., and Artz, R. R. E.: Interactions among fungal community structure, litter decomposition and depth of water table in a cutover peatland, FEMS Microbiology Ecology, 64, 433--448, \url{https://doi.org/10.1111/j.1574-6941.2008.00487.x}, 2008.

\leavevmode\vadjust pre{\hypertarget{ref-Vehtari.2021}{}}%
Vehtari, A., Gelman, A., Simpson, D., Carpenter, B., and Bürkner, P.-C.: Rank-{Normalization}, {Folding}, and {Localization}: {An Improved R{\^{}}} for {Assessing Convergence} of {MCMC} (with {Discussion}), Bayesian Analysis, 16, \url{https://doi.org/10.1214/20-BA1221}, 2021.

\leavevmode\vadjust pre{\hypertarget{ref-Vitt.1990}{}}%
Vitt, D. H.: Growth and production dynamics of boreal mosses over climatic, chemical and topographic gradients, Botanical Journal of the Linnean Society, 104, 35--59, \url{https://doi.org/10.1111/j.1095-8339.1990.tb02210.x}, 1990.

\leavevmode\vadjust pre{\hypertarget{ref-Wickham.2016}{}}%
Wickham, H.: {ggplot2}: {Elegant} graphics for data analysis, 2nd ed. 2016., Springer International Publishing : Imprint: Springer, Cham, \url{https://doi.org/10.1007/978-3-319-24277-4}, 2016.

\leavevmode\vadjust pre{\hypertarget{ref-Wickham.2019}{}}%
Wickham, H., Averick, M., Bryan, J., Chang, W., McGowan, L., François, R., Grolemund, G., Hayes, A., Henry, L., Hester, J., Kuhn, M., Pedersen, T., Miller, E., Bache, S., Müller, K., Ooms, J., Robinson, D., Seidel, D., Spinu, V., Takahashi, K., Vaughan, D., Wilke, C., Woo, K., and Yutani, H.: Welcome to the {Tidyverse}, Journal of Open Source Software, 4, 1686, \url{https://doi.org/10.21105/joss.01686}, 2019.

\end{CSLReferences}

\end{document}
